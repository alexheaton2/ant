\chapter{Adic Numbers: The Finite Residue Field Case}

\section{Finite Residue Field Case}
Let $K$ be a field with a non-archimedean valuation $v=\absspc$.
Recall that the set of $a\in K$ with $\abs{a}\leq 1$ forms a ring
$\O$, the ring of integers for $v$.  The set of $u\in K$ with
$\abs{u}=1$ are a group $U$ under multiplication, the group of units
for $v$.  Finally, the set of $a\in K$ with $\abs{a}<1$ is a maximal
ideal $\p$, so the quotient ring $\O/\p$ is a field.  In this section
we consider the case when $\O/\p$ is a finite field of order a prime
power~$q$.  For example, $K$ could be $\Q$ and $\absspc{}$ could be a
$p$-adic valuation, or $K$ could be a number field and $\absspc{}$
could be the valuation corresponding to a maximal ideal of the ring of
integers.  Among other things, we will discuss in more depth the
topological and measure-theoretic nature of the completion of $K$ at
$v$.

Suppose further for the rest of this section that $\absspc{}$ is
discrete.  Then by Lemma~\ref{lem:discrete_principal}, the ideal $\p$
is a principal ideal $(\pi)$, say, and every $a\in K$ is of the form
$a=\pi^n\eps$, where $n\in\Z$ and $\eps\in U$ is a unit. We call
$$
n = \ord(a) = \ord_\pi(a) = \ord_\p(a) = \ord_v(a)
$$
the ord of~$a$ at~$v$.  (Some authors, including me (!) also call
this integer the \defn{valuation} of~$a$ with respect to~$v$.)  If
$\p=(\pi')$, then $\pi/\pi'$ is a unit, and conversely, so $\ord(a)$
is independent of the choice of~$\pi$.

Let $\O_v$ and $\p_v$ be defined with respect to the completion $K_v$
of $K$ at $v$.  
\begin{lemma}\ilem{reduction homomorphism}
There is a natural isomorphism 
$$
\vphi:\O_v/\p_v \to \O/\p,
$$
and $\p_v = (\pi)$ as an $\O_v$-ideal.
\end{lemma}
\begin{proof}
  Since we are assuming that $\abs{\cdot}$ is discrete, we may view $\O_v$ as the set of equivalence classes of Cauchy
  sequences $(a_n)$ in $K$ such that $a_n\in \O$ for $n$ sufficiently
  large, and similarly $\p_v$ as those such that $a_n \in \p$ for $n$ sufficiently large.  For any $\eps$, given such a sequence $(a_n)$, there is $N$
  such that for $n,m\geq N$, we have $\abs{a_n-a_m}<\eps$.  In
  particular, we can choose $N$ such that $n,m\geq N$ implies that
  $a_n\con a_m\pmod{\p}$.  Let $\vphi((a_n)) = a_N\pmod{\p}$, which is
  well-defined.  The map $\vphi$ is surjective because the constant
  sequences are in $\O_v$.  Its kernel is the set of Cauchy sequences
  whose elements are eventually all in $\p$, which is exactly $\p_v$.
  This proves the first part of the lemma.  The second part is true
  because any element of $\p_v$ is a sequence all of whose terms are
  eventually in $\p$, hence all a multiple of $\pi$ (we can set to $0$
  a finite number of terms of the sequence without changing the
  equivalence class of the sequence).
\end{proof}

{\em Assume for the rest of this section that $K$ is complete with
  respect to $\absspc$.}
\begin{lemma}\ilem{adic-expansion}
Then ring $\O$ is precisely the set of infinite sums 
\begin{equation}\label{eq:infsum}
  a = \sum_{j=0}^{\infty} a_j \cdot \pi^j
\end{equation}
where the $a_j$ run independently through some set $\cR$ of
representatives of $\O$ in $\O/\p$.
\end{lemma}
By (\ref{eq:infsum}) is meant the limit of the Cauchy sequence
$\sum_{j=0}^n a_j\cdot \pi^j$ as $j\to\infty$.
\begin{proof}
There is a uniquely defined $a_0\in \cR$ such that $\abs{a-a_0}<1$.
Then $a' = \pi^{-1}\cdot (a-a_0) \in \O$.  Now define
$a_1\in \cR$ by $\abs{a'-a_1}<1$.  And so on.
\end{proof}
\begin{example}
  Suppose $K=\Q$ and $\absspc=\absspc_p$ is the $p$-adic valuation,
  for some prime~$p$.  We can take $\cR=\{0,1,\ldots, p-1\}$.
  The lemma asserts that 
  $$\O=\Z_p = \left\{ \sum_{j=0}^{\infty} a_n p^n : 0\leq a_n\leq
    p-1\right\}.$$
  Notice that $\O$ is uncountable since there are $p$
  choices for each $p$-adic ``digit''.  We can do arithmetic with
  elements of $\Z_p$, which can be thought of ``backwards'' as numbers
  in base $p$.  For example, with $p=3$ we have
  \begin{align*}
&   (1+2\cdot 3 + 3^2 + \cdots ) + (2 + 2\cdot 3 + 3^2 + \cdots ) \\
& = 3+4\cdot 3 + 2\cdot 3^2 + \cdots   \qquad \text{not in canonical form}\\
& = 0 + 2\cdot 3 + 3\cdot 3 + 2\cdot 3^2 + \cdots \qquad\text{still not canonical}\\
& = 0 + 2\cdot 3 + 0\cdot 3^2 + \cdots 
\end{align*}

% Basic arithmetic with the $p$-adics in \magma{} is really weird (even
% weirder than it was a year ago...  There are presumably efficiency
% advantages to using the \magma{} formalization, and it's supposed to be
% better for working with extension fields.  But I can't get it to do
% even the calculation below in a way that is clear.)  
Here is an example of doing basic arithmetic with $p$-adic
numbers in Sage:
\begin{lstlisting}
sage: a = 1 + 2*3 + 3^2 + O(3^3)
sage: b = 2 + 2*3 + 3^2 + O(3^3)
sage: a + b
2*3 + O(3^3)
sage: sqrt(a)
1 + 3 + O(3^3)
sage: sqrt(a)^2
1 + 2*3 + 3^2 + O(3^3)
sage: a * b
2 + O(3^3)
\end{lstlisting}
Type {\tt Zp?} and {\tt Qp?} in Sage for much more information
about the various computer models of $p$-adic arithmetic that
are available.

% In PARI (gp) the
% $p$-adics work as expected:
% \begin{verbatim}
%    ? a = 1 + 2*3 + 3^2 + O(3^3);
%    ? b = 2 + 2*3 + 3^2 + O(3^3); 
%    ? a+b
%    %3 = 2*3 + O(3^3)
%    ? sqrt(1+2*3+O(3^20))  
%    %5 = 1 + 3 + 3^2 + 2*3^4 + 2*3^7 + 3^8 + 3^9 + 2*3^10 + 2*3^12 
%           + 2*3^13 + 2*3^14 + 3^15 + 2*3^17 + 3^18 + 2*3^19 + O(3^20)
%    ? 1/sqrt(1+2*3+O(3^20))   
%    %6 = 1 + 2*3 + 2*3^2 + 2*3^7 + 2*3^10 + 2*3^11 + 2*3^12 + 2*3^13 
%           + 2*3^14 + 3^15 + 2*3^16 + 2*3^17 + 3^18 + 3^19 + O(3^20)
% \end{verbatim}
\end{example}

\begin{theorem}\label{thm:compact}\ithm{compactness of ring of integers}
Under the conditions of the preceding lemma, $\O$ is compact with
respect to the $\absspc{}$-topology. 
\end{theorem}
\begin{proof}
  Let $V_\lambda$, for $\lambda$ running through some index set
  $\Lambda$, be some family of open sets that cover $\O$.  We must
  show that there is a finite subcover.  We suppose not.
  
  Let $\cR$ be a set of representatives for $\O/\p$.  Then $\O$ is the
  union of the finite number of cosets $a+\pi\O$, for $a\in \cR$.
    Hence for at lest one $a_0\in \cR$ the set $a_0+\pi \O$
    is not covered by finitely many of the $V_\lambda$.  Then similarly
  there is an $a_1\in \cR$ such that $a_0 + a_1\pi + \pi^2\O$ is not
  finitely covered. And so on.  Let 
  $$
  a = a_0 + a_1\pi + a_2 \pi^2 + \cdots \in \O.
  $$
  Then $a\in V_{\lambda_0}$ for some $\lambda_0\in\Lambda$.  Since
  $V_{\lambda_0}$ is an open set, $a+\pi^J\cdot \O\subset
  V_{\lambda_0}$ for some~$J$ (since those are exactly the open balls
  that form a basis for the topology). This is a contradiction because
  we constructed $a$ so that none of the sets $a+\pi^n\cdot \O$, for
  each $n$, are not covered by any finite subset of the $V_{\lambda}$.
\end{proof}

\begin{definition}[Locally compact]
  A topological space $X$ is \defn{locally compact} at a point $x$ if
  there is some compact subset $C$ of $X$ that contains a neighborhood
  of~$x$.  The space $X$ is locally compact if it is locally compact
  at each point in $X$.
\end{definition}
\begin{corollary}\label{cor:locally_compact}\icor{complete local field locally compact}
The complete local field $K$ is locally compact.
\end{corollary}
\begin{proof}
  If $x\in K$, then $x \in C=x+\O$, and $C$ is a compact subset of $K$
  by Theorem~\ref{thm:compact}.  Also $C$ contains the neighborhood
  $x+\pi\O = B(x,1)$ of $x$.  Thus $K$ is locally compact at $x$.
\end{proof}

\begin{remark}\label{rem:locally_compact}
The converse is also true.  If $K$ is locally compact with respect to
a non-archimedean valuation $\absspc{}$, then
\begin{enumerate}
\item $K$ is complete,
\item the residue field is finite, and
\item the valuation is discrete.
\end{enumerate}
For there is a compact neighbourhood $C$ of $0$.  
Let $\pi$ be any nonzero with $\abs{\pi}<1$.
Then $\pi^n\cdot
\O\subset C$ for sufficiently large $n$, so $\pi^n\cdot \O$ is
compact, being closed.  Hence $\O$ is compact.  Since $\absspc$ is a
metric, $\O$ is sequentially compact, i.e., every fundamental sequence
in $\O$ has a limit, which implies (1).  Let $a_\lambda$ (for
$\lambda\in\Lambda$) be a set of representatives in $\O$ of $\O/\p$.
Then $\O_{\lambda} = \{z : \abs{z-a_{\lambda}}<1\}$ is an open
covering of $\O$.  Thus (2) holds since $\O$ is compact.  Finally,
$\p$ is compact, being a closed subset of $\O$.  Let $S_n$ be the set
of $a\in K$ with $\abs{a}<1-1/n.$  Then $S_n$ (for $1\leq n < \infty$)
is an open covering of $\p$, so $\p=S_n$ for some $n$, i.e., (3) is
true. 

If we allow $\absspc{}$ to be archimedean the only further
possibilities are $k=\R$ and $k=\C$ with $\absspc{}$ equivalent to the
usual absolute value.
\end{remark}

We denote by $K^+$ the commutative topological group whose points are
the elements of $K$, whose group law is addition and whose topology is
that induced by $\absspc$.  General theory tells us that there is an
invariant Haar measure defined on $K^+$ and that this
measure is unique up to a multiplicative constant.  

\begin{definition}[Haar Measure]\label{defn:haar}
A \defn{Haar measure} on a locally compact topological group 
$G$ is a translation invariant measure such that every open
set can be covered by open sets with finite measure. 
\end{definition}

\begin{lemma}\ilem{Haar measure on compact}
  Haar measure of any compact subset $C$ of $G$ is finite.  
\end{lemma}
\begin{proof}
The whole group $G$ is open, so there is a covering $U_\alpha$
of $G$ by open sets each of which has finite measure. 
Since $C$ is compact, there is a finite subset of the $U_\alpha$
that covers $C$.  The measure of $C$ is at most the sum of
the measures of these finitely many $U_\alpha$, hence finite.
\end{proof}

\begin{remark}
  Usually one defined Haar measure to be a translation invariant
  measure such that the measure of compact sets is finite.  Because of
  local compactness, this definition is equivalent to
  Definition~\ref{defn:haar}.  We take this alternative viewpoint
  because Haar measure is constructed naturally on the topological
  groups we will consider by defining the measure on each member of a
  basis of open sets for the topology.
\end{remark}

We now deduce what any such measure $\mu$ on $G=K^+$ must be.  Since
$\O$ is compact (Theorem~\ref{thm:compact}), the measure of $\O$ is
finite.  Since $\mu$ is translation invariant,
$$
  \mu_n = \mu(a + \pi^n \O)
$$
is independent of $a$.  Further, 
$$
a + \pi^n\O = \bigcup_{1\leq j\leq q} a + \pi^n a_j + \pi^{n+1}\O,
\qquad\text{(disjoint union)}
$$
where $a_j$ (for $1\leq j \leq q$) is a set of representatives of
$\O/\p$. Hence
$$
 \mu_n = q\cdot \mu_{n+1}.
$$
If we normalize $\mu$ by putting 
$$
 \mu(\O) = 1
 $$
 we have $\mu_0 = 1$, hence $\mu_1 = q^{-1}$, and in general $$\mu_n =
 q^{-n}.$$
 
 Conversely, without the theory of Haar measure, we could {\em define}
 $\mu$ to be the necessarily unique measure on $K^+$ such that
 $\mu(\O)=1$ that is translation invariant.  This would have to be the
 $\mu$ we just found above.
 
 Everything so far in this section has depended not on the valuation
 $\absspc$ but only on its equivalence class.  The above
 considerations now single out one valuation in the equivalence class
 as particularly important.
\begin{definition}[Normalized valuation]\label{defn:normalized}
Let $K$ be a field equipped with a discrete valuation $\absspc$
and residue class field with $q<\infty$ elements.  We say that
$\absspc$ is \defn{normalized} if 
$$
\abs{\pi} = \frac{1}{q},
$$
where $\p=(\pi)$ is the maximal ideal of $\O$.
\end{definition}
\begin{example}
The normalized valuation on the $p$-adic numbers $\Q_p$ is 
$\abs{u\cdot p^n} = p^{-n}$, where $u$ is a rational number
whose numerator and denominator are coprime to $p$.

Next suppose $K=\Q_p(\sqrt{p})$.  Then the $p$-adic valuation on 
$\Q_p$ extends uniquely to one on $K$ such that 
$\abs{\sqrt{p}}^2 = \abs{p} = 1/p$.  Since $\pi=\sqrt{p}$
for $K$, this valuation is not normalized.  (Note that
the ord of $\pi=\sqrt{p}$ is $1/2$.)
The normalized valuation is $v=\absspc' = \absspc^2$.  Note that 
$\abs{p}' = 1/p^2$, or $\ord_v(p)=2$ instead of $1$.

Finally suppose that $K=\Q_p(\sqrt{q})$ where $x^2-q$
has no root mod $p$.  Then the residue class field 
degree is $2$, and the normalized valuation must
satisfy $\abs{\sqrt{q}} = 1/p^2$.
\end{example}


The following proposition makes clear why this is the best choice of
normalization.
\begin{theorem}\ithm{properties of Haar measure}
Suppose further that $K$ is complete with respect to the normalized
valuation $\absspc{}$.  Then 
$$
\mu(a + b\O) = \abs{b},
$$
where $\mu$ is the Haar measure on $K^+$ normalized so
that $\mu(\O)=1$.
\end{theorem}
\begin{proof}
Since $\mu$ is translation invariant, $\mu(a+b\O) = \mu(b\O)$.
Write $b=u\cdot \pi^n$, where $u$ is a unit. Then since $u\cdot
\O=\O$, we have
$$\mu(b\O) 
  = \mu(u\cdot \pi^n\cdot \O) = \mu(\pi^n \cdot u\cdot\O)
    = \mu(\pi^n\cdot \O) = q^{-n} = \abs{\pi^n} = \abs{b}.
$$
Here we have $\mu(\pi^n\cdot \O) = q^{-n}$ by the discussion
before Definition~\ref{defn:normalized}.
\end{proof}

\begin{exercise}\label{ex:adics1}
\begin{enumerate}
\item Find the normalized Haar measure of the following subset of 
$\Q_7^+$:
$$
U = B\left(28,\frac{1}{50}\right) = 
\left\lbrace x\in \Q_7 : \abs{x-28} < \frac{1}{50}\right\rbrace.
$$
\item
Find the normalized Haar measure of the subset $\Z_7^*$ of
$\Q_7^*$.
\end{enumerate} 
\end{exercise} 

We can express the result of the theorem in a more suggestive way.
Let $b\in K$ with $b\neq 0$, and let $\mu$ be a Haar measure on $K^+$
(not necessarily normalized as in the theorem).  Then we can define a
new Haar measure $\mu_b$ on $K^+$ by putting $\mu_b(E) = \mu(bE)$ for
$E\subset K^+$.  But Haar measure is unique up to a multiplicative
constant and so $\mu_b(E) = \mu(bE) = c\cdot \mu(E)$ for all
measurable sets $E$, where the factor~$c$ depends only on~$b$.
Putting $E=\O$, shows that the theorem implies that~$c$ is just
$\abs{b}$, when $\absspc$ is the normalized valuation.

\begin{remark}
  The theory of locally compact topological groups leads to the
  consideration of the dual (character) group of $K^+$.  It turns out
  that it is isomorphic to $K^+$.  We do not need this fact for class
  field theory, so do not prove it here.  For a proof and applications
  see Tate's thesis or Lang's {\em Algebraic Numbers}, and for
  generalizations see Weil's {\em Adeles and Algebraic Groups} and
  Godement's Bourbaki seminars 171 and 176.  The determination of the
  character group of $K^*$ is local class field theory.
\end{remark} 

The set of nonzero elements of~$K$ is a group $K^*$ under
multiplication.  Multiplication and inverses are continuous with
respect to the topology induced on $K^*$ as a subset of $K$, so $K^*$
is a topological group with this topology.  We have 
$$
  U_1 \subset U \subset K^*
  $$
  where $U$ is the group of units of $\O\subset K$ and $U_1$ is
  the group of $1$-units, i.e., those units $\eps\in U$ with
  $\abs{\eps-1}<1$, so
  $$U_1 = 1 + \pi\O.$$
  The set $U$ is open because of the discreteness of the metric, and $U$ is closed because $U = \O \setminus \p$ and we already proved that $\O$ is closed and $\p$ is open in this case.  Likewise, $U_1$ is both open and closed.

  The quotient $K^*/U = \{ \pi^n \cdot U : n \in \Z\}$ is isomorphic to the additive group $\Z^+$
of integers with the discrete topology, where the map is 
$$
 \pi^n\cdot U \mapsto n  \qquad \text{ for } n\in\Z.
 $$

The quotient
 $U/U_1$ is isomorphic to the multiplicative group $\F^*$ of the
 nonzero elements of the residue class field $\F = \O/\p$, where the finite group
 $\F^*$ has the discrete topology.  
Note that $\F^*$ is cyclic
 of order $q-1$, and Hensel's lemma implies that $K^*$ contains a
 primitive $(q-1)$th root of unity $\zeta$.  Thus $K^*$ has
the following structure:
$$
K^* = \{ \pi^n\zeta^m\eps : n\in \Z, m\in\Z/(q-1)\Z, \eps\in U_1\}
\isom \Z\,\, \cross\,\, \Z/(q-1)\Z \,\,\cross\,\, U_1.
$$
(How to apply Hensel's lemma: Let $f(x) = x^{q-1}-1$ and let
$a\in\O$ be such that $a\mod\p$ generates $K^*$.  Then $\abs{f(a)}<1$
and $\abs{f'(a)}=1$.  By Hensel's lemma  there is a
$\zeta\in K$ such that $f(\zeta)=0$ and $\zeta\con a\pmod{\p}$.)

Since $U$ is compact and the cosets of $U$ cover $K$, we see that
$K^*$ is locally compact.  
\begin{lemma}\ilem{Haar measure on $K^*$}
The additive Haar measure $\mu$ on $K^+$,
when restricted to $U_1$ gives a measure on $U_1$ that is also
invariant under multiplication, so gives a Haar measure on $U_1$.
\end{lemma}
\begin{proof}
It suffices to show that
$$\mu(1+\pi^n\O) = \mu(u\cdot(1+\pi^n\O)),$$ 
for any $u\in U_1$ and $n>0$.
Write $u=1+a_1\pi + a_2\pi^2 +\cdots$.
We have
\begin{align*}
 u\cdot (1+\pi^n\O) &= (1+a_1\pi + a_2\pi^2 +\cdots)\cdot (1+\pi^n\O)\\
  &= 1+a_1\pi + a_2\pi^2 + \cdots + \pi^n\O\\
  &= a_1\pi + a_2\pi^2 + \cdots + ( 1+\pi^n\O),
\end{align*}
which is an additive translate of $1+\pi^n\O$, hence has the
same measure. 
\end{proof}
Thus $\mu$ gives a Haar measure on $K^*$ by translating $U_1$ around
to cover $K^*$.

\begin{lemma}\ilem{$K^+$ and $K^*$ are totally disconnected}
The topological spaces $K^+$ and $K^*$ are totally disconnected (the
only connected sets are points).
\end{lemma}
\begin{proof}
  The proof is the same as that of
  Proposition~\ref{prop:disconnected}.  The point is that the
  non-archimedean triangle inequality forces the complement of an open
  disc to be open, hence any set with at least two distinct elements
  ``falls apart'' into a disjoint union of two disjoint open subsets.
\end{proof}

\begin{remark}
Note that $K^*$ and $K^+$ are locally isomorphic if $K$ has
characteristic~$0$.   We have the exponential map
$$
a \mapsto \exp(a) = \sum_{n=0}^{\infty} \frac{a^n}{n!}
$$
defined for all sufficiently small $a$ with its inverse
$$
\log(a) = \sum_{n=1}^{\infty} \frac{(-1)^{n-1}(a-1)^n}{n},
$$
which is defined for all $a$ sufficiently close to $1$.
\end{remark}

%%% Local Variables: 
%%% mode: latex
%%% TeX-master: "ant"
%%% End: 
