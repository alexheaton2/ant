\chapter{Basic Commutative Algebra}

The commutative algebra in this chapter provides a
foundation for understanding the more refined number-theoretic
structures associated to number fields.

First we prove the structure theorem for finitely generated abelian
groups.  Then we establish the standard properties of Noetherian rings
and modules, including a proof of the Hilbert basis theorem.  We also
observe that finitely generated abelian groups are Noetherian
$\Z$-modules.  After establishing
properties of Noetherian rings, we consider rings of algebraic
integers and discuss some of their properties.

\section{Finitely Generated Abelian Groups}\label{sec:fg}
Finitely generated abelian groups arise all over algebraic number
theory.  For example, they will appear in this book as class groups,
unit groups, and the underlying additive groups of rings of integers,
and as Mordell-Weil groups of elliptic curves.

In this section, we prove the structure theorem for finitely generated
abelian groups, since it will be crucial for much of what we will do
later.  \i{abelian groups!structure theorem} \i{structure theorem}

Let $\Z=\{0,\pm 1, \pm 2, \ldots\}$ denote the ring of (rational)
integers, and for each positive integer~$n$, let $\Z/n\Z$ denote the
ring of integers modulo~$n$, which is a cyclic abelian group of
order~$n$ under addition.

\begin{definition}[Finitely Generated]
  A group $G$ is \defn{finitely generated} if there exists
  $g_1,\ldots, g_n \in G$ such that every element of $G$ can be
  expressed as a finite product (or sum, if we write $G$ additively)
  of positive or negative powers of the $g_i$.
\end{definition}
For example, the group $\Z$ is finitely generated, since it is generated
by~$1$.

\begin{theorem}[Structure Theorem for Finitely Generated Abelian Groups]\label{thm:struc}
\ithm{structure of abelian groups}
Let $G$ be a finitely generated abelian group.  Then there is an isomorphism
$$
  G \ncisom (\Z/n_1\Z) \oplus (\Z/n_2\Z) \oplus \cdots \oplus
    (\Z/n_s\Z) \oplus \Z^{r},
$$
where $r, s\geq 0$, $n_1>1$ and $n_1\mid{}n_2\mid{}\cdots \mid{}n_s$. 
Furthermore, the $n_i$ and~$r$ are uniquely determined by~$G$.
\end{theorem}

\begin{exercise}
	Quick! Guess how many abelian groups there are of order less than $12$. Use Theorem~\ref{thm:struc} to classify all abelian groups of order less than $12$. How many do you think there are? How many are there?
\end{exercise}

We will prove the theorem as follows.  We first remark that any
subgroup of a finitely generated free abelian group is finitely
generated.  Then we see how to represent finitely generated abelian groups 
as quotients of finite rank free abelian groups, and how to
reinterpret such a presentation in terms of matrices over the
integers.  Next we describe how to use row and column operations over
the integers to show that every matrix over the integers is equivalent
to one in a canonical diagonal form, called the Smith normal form.  We
obtain a proof of the theorem by reinterpreting the \ii{Smith normal form} in
terms of groups.  Finally, we observe that
the representation in the theorem is necessarily unique. 

\begin{proposition}\label{prop:subfin}\iprop{subgroup of free group}
If $H$ is a subgroup of a finitely generated abelian group, then $H$
is finitely generated.
\end{proposition}
The key reason that this is true is that~$G$ is a finitely generated
module over the principal ideal domain $\Z$.  We defer the
proof of Proposition~\ref{prop:subfin} to Section~\ref{sec:noetherian},
where we will give a complete proof of a beautiful generalization 
in the context of Noetherian rings (the Hilbert basis theorem).

\begin{corollary}\label{cor:presentation}\icor{group as quotient of free
groups}
Suppose $G$ is a finitely generated abelian group.  Then there are
finitely generated free abelian groups $F_1$ and $F_2$ and there is
a homomorphism
$\psi:F_2 \to F_1$ such that
$G \ncisom F_1/\psi(F_2)$.
\end{corollary}
\begin{proof}
  Let $x_1,\ldots, x_m$ be generators for $G$.  Let $F_1=\Z^m$ and let
  $\vphi:F_1\to G$ be the homomorphism that sends the $i$th generator
  $(0,0,\ldots,1,\ldots,0)$ of $\Z^m$ to $x_i$.  Then $\vphi$ is
  surjective, and by Proposition~\ref{prop:subfin} the
  kernel $\ker(\vphi)$ of $\vphi$ is a finitely generated abelian
  group.
Suppose there are $n$ generators for $\ker(\vphi)$, let 
$F_2 = \Z^n$ and fix a surjective homomorphism $\psi:F_2
  \to \ker(\vphi)$.  Then $F_1 / \psi(F_2)$ is isomorphic to $G$.
\end{proof}

 An \defn{sequence} of homomorphisms of abelian groups
$$H \xra{f} G \xra{g} K$$
is exact if  $\im(f) = \ker(g)$.
Given a finitely generated abelian group $G$, Corollary~\ref{cor:presentation} 
provides an exact sequence
$$
  F_2 \xra{\psi} F_1 \to G \to 0.
$$

Suppose $G$ is a nonzero finitely generated abelian group.  By the
corollary, there are free abelian groups $F_1$ and $F_2$ and there is a
homomorphism $\psi:F_2 \to F_1$ such that $G\ncisom F_1/\psi(F_2)$.
Upon choosing a basis for $F_1$ and $F_2$, we obtain isomorphisms
$F_1\ncisom \Z^n$ and $F_2\ncisom \Z^m$ for integers $n$ and $m$.
Just as in linear algebra, we
 view $\psi:F_2\to F_1$ as being given by left multiplication
by the $n\times m$ matrix $A$ whose columns are the images of the
generators of $F_2$ in $\Z^n$.  We visualize this as follows:

$$
\Z^m \xra{A} \Z^n \to G \to 0
$$

The \defn{cokernel} of the
homomorphism defined by $A$ is the quotient of $\Z^n$ by the image of $A$ (i.e., the
$\Z$-span of the columns of $A$), and this cokernel is isomorphic to
$G$.  

The following proposition implies that we may choose a bases for $F_1$
and $F_2$ such that the matrix of $A$ only has nonzero entries along
the diagonal, so that the structure of the cokernel of $A$ is
trivial to understand.

\begin{proposition}[Smith normal form]\label{prop:smith}\iprop{Smith normal form}
Suppose~$A$ is an $n\times m$ integer matrix.  Then there exist
invertible integer matrices $P$ and $Q$ such that $A'=PAQ$ only
has nonzero entries along the diagonal, and these entries are
$n_1, n_2,\ldots, n_s,0,\ldots,0$, where
$s\geq 0$, $n_1\geq 1$ and $n_1\mid n_2 \mid{} \cdots \mid{} n_s$.  
Here $P$ and $Q$ are invertible as integer matrices, so $\det(P)$
and $\det(Q)$ are $\pm 1$.
The matrix $A'$ is called the \defn{Smith normal form} of $A$.
\end{proposition}
We will see in the proof of Theorem~\ref{thm:struc} that
$A'$ is uniquely determined by $A$.
An example of a matrix in Smith normal form is 
$$
A=\left(
        \begin{matrix}2&0&0&0\\0&6&0&0\\0&0&0&0
        \end{matrix}\right).
$$

\begin{proof}[Proof of Proposition~\ref{prop:smith}]  The matrix $P$ will be a product of matrices that define elementary
  row operations and $Q$ will be a product corresponding to elementary
  column operations.  The elementary row and column operations over
  $\ZZ$ are as follows:
\begin{enumerate}
\item{}{[\bf Add multiple]} Add an integer multiple of one row to another (or a multiple
of one column to another).
\item{}{[\bf Swap]} Interchange two rows or two columns.
\item{}{[\bf Rescale]} Multiply a row by $-1$.
\end{enumerate}
Each of these operations is given by left or right multiplying by an
invertible matrix~$E$ with integer entries, where~$E$ is the result of
applying the given operation to the identity matrix, and~$E$ is
invertible because each operation can be reversed using another row or
column operation over the integers.

To see that the proposition must be true, assume $A\neq 0$ and perform
the following steps (compare \cite[pg. 459]{artin:algebra}):
\begin{enumerate}
\item By permuting rows and columns, move a nonzero entry of $A$ with
  smallest absolute value to the upper left corner of $A$.  Now
  ``attempt'' (as explained in detail below) to make all other entries
  in the first row and column $0$ by adding multiples of the top row
  or first column to other rows or columns, as follows:
\begin{quote}
Suppose $a_{i1}$ is a nonzero entry in the first column, with
$i>1$.  Using the division algorithm, write $a_{i1} = a_{11}q + r$,
with $0\leq r < a_{11}$.  Now add $-q$ times the first row to the
$i$th row.  If $r>0$, then go to step~1 (so that an entry with
absolute value at most $r$ is the upper left corner).  
\end{quote}
  If at any point this operation produces a nonzero entry in the
  matrix with absolute value smaller than $|a_{11}|$, start the
  process over by permuting rows and columns to move that entry to the
  upper left corner of $A$.  Since the integers $|a_{11}|$ are a
  decreasing sequence of positive integers, we will not have to move
  an entry to the upper left corner infinitely often, so when this
  step is done the upper left entry of the matrix is nonzero, and all
  entries in the first row and column are 0.


\item We may now assume that $a_{11}$ is the only nonzero entry in the
first row and column.  If some entry $a_{ij}$ of $A$ is not divisible
by $a_{11}$, add the column of $A$ containing $a_{ij}$ to the first
column, thus producing an entry in the first column that is nonzero.
When we perform step~2, the remainder $r$ will be greater than $0$.
Permuting rows and columns results in a smaller $|a_{11}|$.  Since
$|a_{11}|$ can only shrink finitely many times, eventually we will get
to a point where every $a_{ij}$ is divisible by $a_{11}$.  If $a_{11}$
is negative, multiple the first row by $-1$.
\end{enumerate}
After performing the above operations, the first row and column
of $A$ are zero except for $a_{11}$ which is positive and divides
all other entries of $A$.  We repeat the above steps for the 
matrix $B$ obtained from $A$ by deleting the first row and column.
The upper left entry of the resulting matrix will be divisible by
$a_{11}$, since every entry of $B$ is.  Repeating the argument
inductively proves the proposition.
\end{proof}

\begin{example}
The matrix $\mtwo{-1}{2}{-3}{4}$ has Smith normal form
$\mtwo{1}{0}{0}{2}$,
and the matrix 
$\mthree{1}{4}{9}{16}{25}{36}{49}{64}{81}$
has Smith normal form
$\mthree{1}{0}{0}{0}{3}{0}{0}{0}{72}.$
As a double check, note that the determinants of a matrix and its Smith
normal form match, up to sign. This is because
$$\det(PAQ) = \det(P)\det(A)\det(Q) = \pm \det(A).$$

We compute each of the above Smith forms using \sage,
along with the corresponding transformation matrices.
First the $2\times 2$ matrix.
\begin{sagecode}
\begin{sagecell}
A = matrix(ZZ, 2, [-1,2, -3,4])
S, U, V = A.smith_form(); S
\end{sagecell}
\begin{sageout}
[1 0]
[0 2]
\end{sageout}
\begin{sagecell}
U*A*V
\end{sagecell}
\begin{sageout}
[1 0]
[0 2]
\end{sageout}
\begin{sagecell}
U
\end{sagecell}
\begin{sageout}
[ 0  1]
[ 1 -1]
\end{sageout}
\begin{sagecell}
V
\end{sagecell}
\begin{sageout}
[1 4]
[1 3]
\end{sageout}
\end{sagecode}
The \sage{} {\tt matrix} command takes as input the base ring, the number
of rows, and the entries.  Next we compute with a $3\times3$ matrix.
\begin{sagecode}
\begin{sagecell}
A = matrix(ZZ, 3, [1,4,9,  16,25,36, 49,64,81])
S, U, V = A.smith_form(); S
\end{sagecell}
\begin{sageout}
[ 1  0  0]
[ 0  3  0]
[ 0  0 72]
\end{sageout}
\begin{sagecell}
U*A*V
\end{sagecell}
\begin{sageout}
[ 1  0  0]
[ 0  3  0]
[ 0  0 72]
\end{sageout}
\begin{sagecell}
U
\end{sagecell}
\begin{sageout}
[  0   0   1]
[  0   1  -1]
[  1 -20 -17]
\end{sageout}
\begin{sagecell}
V
\end{sagecell}
\begin{sageout}
[  47   74   93]
[ -79 -125 -156]
[  34   54   67]
\end{sageout}
\end{sagecode}

Finally we compute the Smith form of a matrix of rank $2$:
\begin{sagecode}
\begin{sagecell}
m = matrix(ZZ, 3, [2..10]); m
\end{sagecell}
\begin{sageout}
[ 2  3  4]
[ 5  6  7]
[ 8  9 10]
\end{sageout}
\begin{sagecell}
m.smith_form()[0]
\end{sagecell}
\begin{sageout}
[1 0 0]
[0 3 0]
[0 0 0]
\end{sageout}
\end{sagecode}
\end{example}


\begin{proof}[Proof of Theorem~\ref{thm:struc}] 
Suppose $G$ is a finitely generated abelian group, which we may assume
is nonzero.  As in the paragraph before Proposition~\ref{prop:smith},
we use Corollary~\ref{cor:presentation} to write $G$ as the cokernel
of an $n\times m$ integer matrix $A$.  By Proposition~\ref{prop:smith}
there are isomorphisms $Q:\Z^m\to \Z^m$ and $P:\Z^n\to \Z^n$ such that
$A'=PAQ$ has diagonal entries $n_1, n_2,\ldots,
n_s,0,\ldots,0$, where $n_1>1$ and $n_1\mid n_2 \mid{} \ldots \mid{}
n_s$.  Then $G$ is isomorphic to the cokernel of the diagonal matrix
$A'$, so
\begin{equation}
\label{eqn:gprod}
  G \isom (\Z/n_1\Z) \oplus (\Z/n_2\Z)
 \oplus \cdots \oplus (\Z/n_s\Z) \oplus \Z^{r},
\end{equation}
as claimed.  The $n_i$ are determined by $G$, because $n_i$ is the
smallest positive integer~$n$ such that $nG$ requires at most $s+r-i$
generators. We see from the representation (\ref{eqn:gprod}) of $G$ as
a product that $n_i$ has this property and that no smaller positive
integer does.
\end{proof}

\begin{exercise}
	Recall Smith normal form defined in Proposition~\ref{prop:smith}. With only minor modifications, then the proposition and proof will work over any principle ideal domain. Find and apply these modifications then find the Smith normal form of the matrix $\begin{pmatrix} 1 & 2 & 3 \\ 0 & 1+i & 2 \\ 0 & 1 & 5 \end{pmatrix}$.
	
	\begin{hint}
		You can use \sage{} to verify your answer. However, you will need to make explicitly construct the Gaussian integers in order to input the matrix. You can do this by the following:
		\begin{sagecode}
		\begin{sagecell}
K.<i> = QuadraticField(-1)
R = K.maximal_order()
M = matrix(R, 3, [1,2,3,0,1+i,2,0,1,5]); show(M)
#show(M.smith_form()[0]) #uncomment for the answer
		\end{sagecell}
		\end{sagecode}
	\end{hint}
\end{exercise}

\begin{exercise}
	Let $A=\left(
	        \begin{matrix}1&2&3\\4&5&6\\7&8&9
	        \end{matrix}\right)$. 
	\begin{enumerate}
	\item Find the Smith normal form of $A$.
	\item Prove that 
	the cokernel of the map $\Z^3\to \Z^3$ given by multiplication by~$A$ 
	is isomorphic to $\Z/3\Z \oplus \Z$.
	\end{enumerate}
\end{exercise}

\section{Noetherian Rings and Modules}\label{sec:noetherian}

A module $M$ over a commutative ring $R$ with unit element is much
like a vector space, but with more subtle structure.  In this book,
most of the modules we encounter will be noetherian, which is a
generalization of the ``finite dimensional'' property of vector
spaces.  This section is about properties of noetherian modules (and
rings), which are crucial to much of this book.  We thus
give complete proofs of these properties, so you will have a solid
foundation on which to learn algebraic number theory.

We first define noetherian rings and modules, then introduce several
equivalent characterizations of them.  We prove that when the
base ring is noetherian, a module is finitely generated if and only if
it is noetherian.  Next we define short exact sequences, and prove
that the middle module in a sequence is noetherian if and only if the
first and last modules are noetherian.  Finally, we prove the Hilbert
basis theorem, which asserts that adjoining finitely many elements
to a noetherian ring results in a noetherian ring. 

Let $R$ be a commutative ring with unity.
An \defn{$R$-module} is an additive abelian
group $M$ equipped with a map $R\times M \to M$ such that for all~$r,
r'\in R$ and all $m, m'\in M$ we have $(r r')m = r(r' m )$, $(r + r')m
= rm + r' m$, $r(m+m') = rm + rm'$, and $1m=m$.  A \defn{submodule} of $M$
is a subgroup of $M$ that is preserved by the action of $R$.
For example, $R$ is a module over itself, and 
any ideal $I$ in $R$ is an $R$-submodule of $R$.

\begin{example}
Abelian groups are the same as $\Z$-modules, and vector spaces
over a field $K$ are the same as $K$-modules.
\end{example}

An $R$-module $M$ is finitely generated if there are elements $m_1, \ldots, m_n\in M$
such that every element of $M$ is an $R$-linear combination of the $m_i$.  The noetherian
property is stronger than just being finitely generated:

\begin{definition}[Noetherian] An $R$-module~$M$ is \defn{noetherian} if every
submodule of $M$ is finitely generated.  A ring~$R$ is
\defn{noetherian} if~$R$ is noetherian as a module over itself, i.e.,
if every ideal of~$R$ is finitely generated.
\end{definition}

Any submodule~$M'$ of a noetherian module~$M$ is also noetherian.
Indeed, if every submodule of~$M$ is finitely generated then so is
every submodule of $M'$, since submodules of $M'$ are also submodules
of $M$.

\begin{example}
Let $R=M=\Q[x_1,x_2,\ldots]$ be a polynomial ring over $\Q$ in
infinitely many indeterminants $x_i$.  Then $M$ is finitely generated
{\em as an $R$-module} (!), since it is generated by $1$.  Consider
the submodule $I=(x_1,x_2,\ldots)$ of polynomials with $0$ constant
term, and suppose it is generated by polynomials $f_1, \ldots, f_n$.
Let $x_i$ be an indeterminant that does not appear in any $f_j$, and
suppose there are $h_k\in R$ such that $\sum_{k=1}^n h_k f_k = x_i$.
Setting $x_i=1$ and all other $x_j=0$ on both sides of this equation
and using that the $f_k$ all vanish (they have 0 constant term),
yields $0=1$, a contradiction.  We conclude that the ideal $I$ is not
finitely generated, hence $M$ is not a noetherian $R$-module, despite
being finitely generated.
\end{example}

\begin{definition}[Ascending chain condition]
An $R$-module $M$ satisfies the \defn{ascending chain condition} if
every sequence $M_1\subset M_2 \subset M_3 \subset \cdots$ of
submodules of~$M$ eventually stabilizes, i.e., there is some $n$ such
that $M_n=M_{n+1}=M_{n+2}=\cdots$.
\end{definition}
We will use the notion of maximal element below.  If $\mathcal{X}$ is
a set of subsets of a set $S$, ordered by inclusion, then a
\defn{maximal element} $A\in \mathcal{X}$ is a set such that no superset
of $A$ is contained in $\mathcal{X}$.  Note that $\mathcal{X}$ may 
contain many different maximal elements.

\begin{proposition}\iprop{characterization of noetherian}
If $M$ is an $R$-module, then the following are equivalent:
\begin{enumerate}
\item $M$ is noetherian,
\item $M$ satisfies the ascending chain condition, and
\item Every nonempty set of submodules of $M$ contains at least one
maximal element.  
\end{enumerate}
\end{proposition}
\begin{proof}
{\bf $1\implies 2$:} Suppose $M_1\subset M_2\subset \cdots$ is a
sequence of submodules of $M$.  Then $M_\infty=\union_{n=1}^{\infty}
M_n$ is a submodule of $M$.  Since $M$ is noetherian and $M_\infty$
is a submodule of~$M$, there is a
finite set $a_1,\ldots, a_m$ of generators for $M_{\infty}$.  Each $a_i$
must be contained in some $M_j$, so there is an $n$ such that
$a_1,\ldots, a_m\in M_n$.  But then $M_{k}=M_n$ for all $k\geq n$,
which proves that the chain of $M_i$ stabilizes, 
so the ascending chain condition holds for~$M$.

\noindent{}{\bf $2\implies 3$:} Suppose 3 were false, so there exists
a nonempty set~$S$ of submodules of~$M$ that does not contain a
maximal element.  We will use~$S$ to construct an infinite ascending
chain of submodules of~$M$ that does not stabilize.  Note that~$S$ is
infinite, otherwise it would contain a maximal element.  Let $M_1$ be
any element of~$S$.  Then there is an $M_2$ in $S$ that contains
$M_1$, otherwise $S$ would contain the maximal element $M_1$.
Continuing inductively in this way we find an $M_3$ in $S$ that
properly contains $M_2$, etc., and we produce an infinite ascending
chain of submodules of $M$, which contradicts the ascending chain
condition.

\noindent{}{\bf $3\implies 1$:} Suppose 1 is false, so there is a
submodule $M'$ of~$M$ that is not finitely generated.  We will show
that the set~$S$ of all finitely generated submodules of $M'$ does not
have a maximal element, which will be a contradiction.  Suppose~$S$
does have a maximal element~$L$.  Since~$L$ is finitely generated and
$L\subset M'$, and $M'$ is not finitely generated, there is an $a\in
M'$ such that $a\not\in L$.  Then $L'=L+Ra$ is an element of~$S$ that
strictly contains the presumed maximal element~$L$, a contradiction.
\end{proof}

A \defn{homomorphism} of $R$-modules $\vphi:M\to N$ is an abelian group
homomorphism such that for any $r\in R$ and $m\in M$ we have
$\vphi(rm) = r\vphi(m)$.  
A sequence 
$$
  L \xra{f} M \xra{g} N,
$$ where $f$ and $g$ are homomorphisms of $R$-modules,
is \defn{exact} if $\im(f)=\ker(g)$.
A \defn{short exact sequence} of $R$-modules is a sequence
$$0 \to L \xra{f} M \xra{g} N \to 0$$
that is exact at each point; thus
$f$ is injective, $g$ is surjective, and $\im(f) = \ker(g)$.

\begin{example}
The sequence
$$
  0 \to \Z \xra{2} \Z\ra \Z/2\Z \to 0
$$ 
is an exact sequence, where the first map sends $1$ to $2$, and the second
is the natural quotient map.
\end{example}


\begin{lemma}\label{lem:noetherianexact}\ilem{exactness and noetherian}
If 
$$
 0 \to L \xra{f} M \xra{g} N \to 0
$$ 
is a short exact sequence of~$R$-modules, then~$M$ is noetherian if and 
only if both~$L$ and~$N$ are noetherian.
\end{lemma}
\begin{proof}
First suppose that~$M$ is noetherian.  Then~$L$ is a submodule of~$M$, so~$L$
is noetherian.  Let $N'$ be a submodule of~$N$; then the inverse image of $N'$
in~$M$ is a submodule of~$M$, so it is finitely generated, hence its image
$N'$ is also finitely generated.  Thus~$N$ is noetherian as well.

Next assume nothing about~$M$, but suppose that both~$L$ and~$N$ are
noetherian.  Suppose $M'$ is a submodule of $M$; then $M_0=f(L)\cap M'$
is isomorphic to a submodule of the noetherian module $L$, so $M_0$ is
generated by finitely many elements $a_1,\ldots, a_n$.  The quotient
$M'/M_0$ is isomorphic (via $g$) to a submodule of the noetherian
module~$N$, so $M'/M_0$ is generated by finitely many elements
$b_1,\ldots, b_m$. For each $i\leq m$, let $c_i$ be a lift of $b_i$ to
$M'$, modulo $M_0$.  Then the elements $a_1,\ldots, a_n, c_1,\ldots,
c_m$ generate $M'$, for if $x\in M'$, then there is some element $y\in
M_0$ such that $x-y$ is an $R$-linear combination of the $c_i$,
and~$y$ is an $R$-linear combination of the $a_i$.
\end{proof}


\begin{proposition}\label{prop:noethfg}\iprop{noetherian equals finitely generated}
Suppose~$R$ is a noetherian ring.  Then an $R$-module~$M$ is 
noetherian if and only if it is finitely generated. 
\end{proposition}
\begin{proof}
If~$M$ is noetherian then every submodule of~$M$ is finitely generated
so~$M$ itself is finitely generated.  Conversely, suppose~$M$ is finitely
generated, say by elements $a_1,\ldots, a_n$.  Then there is a
surjective homomorphism from $R^n=R\oplus \cdots \oplus R$ to~$M$ that
sends $(0,\ldots,0,1,0,\ldots,0)$ ($1$ in the $i$th factor) to $a_i$.
Using Lemma~\ref{lem:noetherianexact} and exact sequences of
$R$-modules such as $0\to R\to R\oplus R\to R\to 0$, we see
inductively that $R^n$ is noetherian.  Again by
Lemma~\ref{lem:noetherianexact}, homomorphic images of noetherian
modules are noetherian, so~$M$ is noetherian.
\end{proof}

\begin{lemma}\label{lem:surjnoetherian}\ilem{surjection and noetherian}
Suppose $\vphi:R\to S$ is a surjective homomorphism of rings 
and $R$ is noetherian.   Then $S$ is noetherian.
\end{lemma}
\begin{proof}
The kernel of $\vphi$ is an ideal $I$ in $R$, and
we have an exact sequence 
$$ 
  0 \to I \to R \to S \to 0
$$
with~$R$ noetherian.  
This is an exact sequence of $R$-modules, where $S$ has the
$R$-module structure induced from $\vphi$ (if $r\in R$
and $s\in S$, then we define $rs = \vphi(r)s$).
By Lemma~\ref{lem:noetherianexact}, it follows
that~$S$ is a noetherian $R$-modules.  Suppose~$J$ is an ideal of~$S$.
Since~$J$ is an $R$-submodule of~$S$, if we view~$J$ as an $R$-module,
then~$J$ is finitely generated.  Since~$R$ acts on~$J$ through~$S$,
the $R$-generators of~$J$ are also $S$-generators of~$J$, so~$J$ 
is finitely generated as an ideal.  Thus~$S$ is noetherian.
\end{proof}

\begin{theorem}[Hilbert Basis Theorem]\label{thm:hilbert}
\ithm{Hilbert Basis}
If $R$ is a noetherian ring and $S$ is finitely generated as a ring
over~$R$, then~$S$ is noetherian.  In particular, for any~$n$ the
polynomial ring $R[x_1,\ldots, x_n]$ and any of its quotients are
noetherian.
\end{theorem}
\begin{proof}
Assume first that we have already shown that for any $n$ the
polynomial ring $R[x_1,\ldots, x_n]$ is noetherian.  Suppose $S$ is
finitely generated as a ring over $R$, so there are generators
$s_1,\ldots, s_n$ for $S$.  Then the map $x_i\mapsto s_i$ extends
uniquely to a surjective homomorphism $\pi: R[x_1,\ldots, x_n] \onto S$,
and Lemma~\ref{lem:surjnoetherian} implies that $S$ is noetherian.

The rings $R[x_1,\ldots, x_n]$ and $(R[x_1,\ldots,x_{n-1}])[x_n]$ are
isomorphic, so it suffices to prove that if~$R$ is noetherian then
$R[x]$ is also noetherian.  (Our proof follows
\cite[\S12.5]{artin:algebra}.)
Thus suppose $I$ is an ideal of $R[x]$ and that~$R$ is
noetherian.  We will show that $I$ is finitely generated.

Let $A$ be the set of leading coefficients of polynomials in $I$.  
(The leading coefficient of a polynomial is the coefficient of
the highest degree monomial, or $0$ if the polynomial is $0$; thus 
$3x^7 + 5x^2  - 4$ has leading coefficient $3$.)
We will first show that~$A$ is an ideal of~$R$.
Suppose $a,b\in A$ are nonzero with $a+b\neq 0$.  Then there are
polynomials~$f$ and~$g$ in~$I$ with leading coefficients~$a$ and~$b$.
If $\deg(f)\leq \deg(g)$, then $a+b$ is the leading coefficient of
$x^{\deg(g)-\deg(f)}f + g$, so $a+b\in A$; the argument when 
$\deg(f)> \deg(g)$ is analogous.  Suppose $r\in R$ and $a\in A$
with $ra\neq 0$. Then $ra$ is the leading coefficient of $rf$, so
$ra\in A$.  Thus $A$ is an ideal in $R$. 

Since $R$ is noetherian and $A$ is an ideal of $R$, there exist nonzero
$a_1,\ldots, a_n\in A$ that generate~$A$ as an ideal.  Since~$A$ is the set
of leading coefficients of elements of~$I$, and the $a_j$ are in~$A$,
we can choose for each $j\leq n$ an element $f_j\in I$ with leading
coefficient $a_j$.  By multipying the $f_j$ by some power of~$x$, we
may assume that the $f_j$ all have the same degree $d\geq 1$.

Let $S_{<d}$ be the set of elements of~$I$ that have degree strictly
less than~$d$.  This set is closed under addition and under
multiplication by elements of~$R$, so $S_{<d}$ is a module over~$R$.
The module $S_{<d}$ is the submodule of the $R$-module of polynomials
of degree less than~$n$, which is noetherian by
Proposition~\ref{prop:noethfg} because it is generated by $1,x,\ldots,
x^{n-1}$.  Thus $S_{<d}$ is finitely generated, and we may choose
generators $h_1,\ldots, h_m$ for $S_{<d}$.

We finish by proving using induction on the degree that every $g\in I$ is an
$R[x]$-linear combination of $f_1,\ldots, f_n, h_1,\ldots, h_m$.
If $g\in I$ has degree $0$, then $g \in S_{<d}$, since $d\geq 1$, so
$g$ is a linear combination of $h_1,\ldots, h_m$.  Next suppose
$g\in I$ has degree $e$, and that we have proven the statement
for all elements of $I$ of degree $<e$.
If $e\leq d$, then $g\in S_{<d}$, so~$g$ is
in the $R[x]$-ideal generated by $h_1,\ldots, h_m$.  Next suppose
that $e\geq d$.  Then the leading coefficient~$b$
of~$g$ lies in the ideal~$A$ of leading coefficients of elements of~$I$, so there
exist $r_i\in R$ such that $b=r_1 a_1 + \cdots + r_n a_n$.  Since
$f_i$ has leading coefficient $a_i$, the difference $g- x^{e-d} r_i
f_i$ has degree less than the degree~$e$ of~$g$.  By induction $g-
x^{e-d} r_i f_i$ is an $R[x]$ linear combination of $f_1,\ldots, f_n,
h_1,\ldots, h_m$, so $g$ is also an $R[x]$ linear combination of
$f_1,\ldots, f_n, h_1,\ldots, h_m$.  Since each $f_i$ and $h_j$ lies in
$I$, it follows that $I$ is generated by $f_1,\ldots, f_n, h_1,\ldots,
h_m$, so $I$ is finitely generated, as required.
\end{proof}

\subsection{The Ring $\Z$ is noetherian}\label{sec:Znoeth}

The ring $\Z$ is noetherian since every ideal of $\Z$ is
generated by one element. 
\begin{proposition}\label{prop:zpid}\iprop{$\Z$ is a PID}
Every ideal of the ring $\Z$ is principal.
\end{proposition}
\begin{proof}
Suppose~$I$ is a nonzero ideal in~$\Z$.  Let~$d$ be the least positive
element of~$I$.  Suppose that $a\in I$ is any nonzero element of~$I$.
Using the division algorithm, we write  $a=dq + r$, where~$q$ is an integer
and $0\leq r < d$.  We have $r=a-dq\in I$ and $r<d$, so our assumption
that $d$ is minimal implies that $r=0$, hence $a=dq$ is in the ideal generated
by~$d$.   Thus~$I$ is the principal ideal generated by~$d$.
\end{proof}

\begin{example}
Let $I=(12,18)$ be the ideal of $\Z$ generated by $12$ and $18$.
If $n=12a+18b\in I$, with $a,b\in\Z$, 
then $6\mid n$, since $6\mid 12$ and $6\mid 18$.
Also, $6=18-12\in I$, so $I=(6)$.

The ring $\Z$
in \sage{} is {\tt ZZ}, which is Noetherian.
\begin{sagecode}
\begin{sagecell}
ZZ.is_noetherian()
\end{sagecell}
\begin{sageout}
True
\end{sageout}
\end{sagecode}
We create the ideal $I$ in \sage{} as follows, and note that
it is principal:
\begin{sagecode}
\begin{sagecell}
I = ideal(12,18); I
\end{sagecell}
\begin{sageout}
Principal ideal (6) of Integer Ring
\end{sageout}
\begin{sagecell}
I.is_principal()
\end{sagecell}
\begin{sageout}
True
\end{sageout}
\end{sagecode}
We could also create $I$ as follows:
\begin{sagecode}
\begin{sagecell}
ZZ.ideal(12,18)
\end{sagecell}
\begin{sageout}
Principal ideal (6) of Integer Ring
\end{sageout}
\end{sagecode}
\end{example}

Propositions~\ref{prop:noethfg} and \ref{prop:zpid} together imply that
any finitely generated abelian group is noetherian.  This means that
subgroups of finitely generated abelian groups are finitely generated,
which provides the missing step in our proof of the structure theorem
for finitely generated abelian groups.

\begin{exercise}
	There is another way to show every principle ideal domain (for example $\Z$) is noetherian (contrast to the proof in Section~\ref{sec:Znoeth}). Let $R$ be a PID and $(a)$ an arbitrary ideal. Use the facts that $(b)\supseteq(a)$ if and only if $b \mid a$ and $R$ is a UFD to show that ascending chain of ideals starting with $(a)$ must stabilize.
\end{exercise}

\section{Rings of Algebraic Integers}
In this section we introduce the central objects of this book, which
are the rings of algebraic integers.  These are noetherian rings with
an enormous amount of structure.  We also introduce a function field
analogue of these rings.


An \defn{algebraic number} is a root of some nonzero polynomial $f(x) \in \Q[x]$.
For example, $\sqrt{2}$ and $\sqrt{5}$ are both algebraic numbers, being 
roots of $x^2-2$ and $x^2-5$, respectively.
But is $\sqrt{2} + \sqrt{5}$ necessarily the root of some polynomial in $\Q[x]$?
This isn't quite so obvious.

\begin{proposition}\label{prop:fdalg}
An element $\alpha$ of a field extension of $\Q$ is an algebraic
number if and only if the ring $\Q[\alpha]$ generated by $\alpha$ is
finite dimensional as a $\Q$ vector space.
\end{proposition}
\begin{proof}
Suppose $\alpha$ is an algebraic number, so there is a nonzero polynomial $f(x)\in\Q[x]$,
so that $f(\alpha)=0$.  The equation $f(\alpha)=0$ implies that
$\alpha^{\deg(f)}$ can be written in terms of smaller powers of $\alpha$, so $\Q[\alpha]$
is spanned by the finitely many numbers $1,\alpha,\ldots,\alpha^{\deg(f)-1}$, hence finite dimensional.
Conversely, suppose $\Q[\alpha]$ is finite dimensional.  Then for some $n\geq 1$, 
we have that $\alpha^n$ is in the $\Q$-vector space spanned by $1,\alpha,\ldots, \alpha^{n-1}$.
Thus $\alpha$ satisfies a polynomial $f(x) \in \Q[x]$ of degree $n$.
\end{proof}

\begin{proposition}\label{prop:algnumfield}
Suppose $K$ is a field and $\alpha, \beta\in K$ are two algebraic
numbers.  Then $\alpha\beta$ and $\alpha+\beta$ are also algebraic numbers.
\end{proposition}
\begin{proof}
Let $f, g\in\Q[x]$ be polynomials that are satisfied by $\alpha,\beta$, respectively.
The subring $\Q[\alpha,\beta]\subset K$ is a $\Q$-vector space that
is spanned by the numbers $\alpha^i\beta^j$, where $0\leq i<\deg(f)$
and $0\leq j<\deg(g)$.  Thus $\Q[\alpha,\beta]$ is finite dimensional, and
since $\alpha+\beta$ and $\alpha\beta$ are both in $\Q[\alpha,\beta]$,
we conclude by Proposition~\ref{prop:fdalg} that both
are algebraic numbers. 
\end{proof}

Suppose $C$ is a field extension of $\Q$ such that every polynomial
$f(x)\in \Q[x]$ factors completely in $C$.  
The algebraic closure $\Qbar$ of $\Q$ inside $C$
is the  field generated by all roots in $C$ of polynomials
in $\Q[x]$.
The fundamental theorem of algebra tells us that
$C=\C$ is one choice of field $C$ as above.
There are other fields $C$, e.g., constructed using $p$-adic numbers.
One can show that any two choices of~$\Qbar$ are isomorphic; however,
there will be {\em many} isomorphisms between them.   

\begin{definition}[Algebraic Integer]
An element $\alpha\in\Qbar$ is an \defn{algebraic integer} if it is a
root of some monic polynomial with coefficients in~$\Z$.
\end{definition}
For example, $\sqrt{2}$ is an algebraic integer, since it is a root
of the monic integral polynomial $x^2-2$. As we will see below,
$1/2$ is not an algebraic integer.

The following two propositions are analogous to
Propositions~\ref{prop:fdalg}--\ref{prop:algnumfield} above, with
the proofs replacing basic facts about vector spaces with facts we
proved above about noetherian rings and modules.

\begin{proposition}\label{prop:intfg}\iprop{characterization of integrality}
An element $\alpha\in\Qbar$ is an algebraic integer if and only if $\Z[\alpha]$ is
finitely generated as a $\Z$-module.  
\end{proposition}
\begin{proof}
Suppose~$\alpha$ is integral and let $f\in\Z[x]$ be a monic integral polynomial
such that $f(\alpha)=0$.  Then, as a $\Z$-module, $\Z[\alpha]$
is generated by $1,\alpha,\alpha^2,\ldots,\alpha^{d-1}$, where~$d$ is
the degree of~$f$.   Conversely, suppose $\alpha\in\Qbar$ is such that
$\Z[\alpha]$ is finitely generated as a module over $\Z$, say by elements 
$f_1(\alpha), \ldots, f_n(\alpha)$.  Let~$d$ be any integer bigger
than the degrees of all~$f_i$.  Then there exist integers $a_i$ such
that $\alpha^d = \sum_{i=1}^n a_i f_i(\alpha)$, hence~$\alpha$ satisfies
the monic polynomial $x^d - \sum_{i=1}^n a_i f_i(x) \in \Z[x]$, so~$\alpha$
is an algebraic integer.
\end{proof}

The proof of the following proposition uses repeatedly that any
submodule of a finitely generated $\Z$-module is finitely generated,
which uses that $\Z$ is noetherian and that finitely generated modules
over a noetherian ring are noetherian.
\begin{proposition}\iprop{$\Zbar$ is a ring}\label{prop:zbarring}
Suppose $K$ is a field and $\alpha, \beta\in K$ are two algebraic
integers.  Then $\alpha\beta$ and $\alpha+\beta$ are also algebraic integers.
\end{proposition}
\begin{proof}
Let $m, n$ be the degrees of monic integral polynomials that have
$\alpha, \beta$ as roots, respectively.  Then we can write $\alpha^m$
in terms of smaller powers of $\alpha$ and likewise for $\beta^n$, so 
the elements $\alpha^i\beta^j$ for $0\leq i < m$ and $0\leq j< n$ span
the $\Z$-module $\Z[\alpha, \beta]$.  Since $\Z[\alpha + \beta]$ is a submodule of the
finitely-generated $\Z$-module $\Z[\alpha, \beta]$, it is finitely
generated, so $\alpha+\beta$ is integral.  Likewise, $\Z[\alpha\beta]$
is a submodule of $\Z[\alpha, \beta]$, so it is also finitely
generated, and $\alpha\beta$ is integral.
\end{proof}



\subsection{Minimal Polynomials}

\begin{definition}[Minimal Polynomial]\label{defn:minpoly}
The \defn{minimal polynomial} of $\alpha\in\Qbar$ is the monic polynomial
$f\in\Q[x]$ of least positive degree such that $f(\alpha)=0$.
\end{definition}
It is a consequence of Lemma~\ref{lem:mindiv} below that
``the'' minimal polynomial of~$\alpha$ is unique.
The minimal polynomial of $1/2$ is $x-1/2$, and
the minimal polynomial of $\sqrt[3]{2}$ is $x^3-2$.

\begin{example}
We compute the minimal polynomial of a number expressed
in terms of $\sqrt[4]{2}$:
\begin{sagecode}
\begin{sagecell}
k.<a> = NumberField(x^4 - 2)
a^4
\end{sagecell}
\begin{sageout}
2
\end{sageout}
\begin{sagecell}
(a^2 + 3).minpoly()
\end{sagecell}
\begin{sageout}
x^2 - 6*x + 7
\end{sageout}
\end{sagecode}
\end{example}

\begin{exercise}
	Find the minimal polynomial of $\sqrt{2} + \sqrt{3}$ by hand. Check your result with \sage{}.
\end{exercise}

\begin{lemma}\label{lem:mindiv}
  Suppose $\alpha \in\Qbar$.  Then the minimal polynomial of~$\alpha$
  divides any polynomial~$h$ such that $h(\alpha)=0$.
\end{lemma}
\begin{proof}
  Let~$f$ be a choice of minimal polynomial of~$\alpha$, as in
  Definition~\ref{defn:minpoly}, and let $h$ be a polynomial with
  $h(\alpha)=0$.  Use the division algorithm to write $h=qf + r$,
  where $0\leq \deg(r) < \deg(f)$.  We have $$r(\alpha) = h(\alpha) -
  q(\alpha) f(\alpha) = 0,$$ so $\alpha$ is a root of~$r$.
  However,~$f$ is a polynomial of least positive degree with
  root~$\alpha$, so $r=0$.
\end{proof}

\begin{exercise}
	Show that the minimal polynomial of an algebraic number $\alpha\in\Qbar$ is unique. 
\end{exercise}

\begin{lemma}\label{lem:minpolint}\ilem{minimal polynomial of algebraic integer}
Suppose~$\alpha \in\Qbar$. Then $\alpha$ is an algebraic integer if and only
if the minimal polynomial $f$ of~$\alpha$ has coefficients in~$\Z$.
\end{lemma}
\begin{proof}
($\Longleftarrow$) Since $f\in \Z[x]$ is monic and $f(\alpha)=0$, we see immediately
that $\alpha$ is an algebraic integer.

($\Longrightarrow$) Since $\alpha$ is an algebraic integer, there is
  some nonzero monic $g\in\Z[x]$ such that $g(\alpha)=0$.
  By Lemma~\ref{lem:mindiv}, we have $g=fh$, for some $h\in\Q[x]$, and
  $h$ is monic because $f$ and $g$ are.  If $f\not\in\Z[x]$, then some
  prime~$p$ divides the denominator of some coefficient of $f$.  Let
  $p^i$ be the largest power of~$p$ that divides some denominator of
  some coefficient~$f$, and likewise let $p^j$ be the largest power
  of~$p$ that divides some denominator of a coefficient of~$h$.  Then
  $p^{i+j}g = (p^if)(p^j h)$, and if we reduce both sides modulo $p$,
  then the left hand side is $0$ but the right hand side is a product
  of two nonzero polynomials in $\F_p[x]$, hence nonzero, a
  contradiction.
\end{proof}

\begin{exercise}
	Which of the following numbers are algebraic integers?
	\begin{enumerate}
		\item The number $(1+\sqrt{5})/2$.
		\item The number $(2+\sqrt{5})/2$.
		\item The value of the infinite sum $\sum_{n=1}^{\infty} 1/n^2$.
		\item The number $\alpha/3$, where $\alpha$ is a root of 
		$x^4 + 54x + 243$.
	\end{enumerate}
\end{exercise}

\begin{example}
We compute some minimal polynomials in \sage.
The minimal polynomial of $1/2$:
\begin{sagecode}
\begin{sagecell}
(1/2).minpoly()
\end{sagecell}
\begin{sageout}
x - 1/2
\end{sageout}
We construct a root $a$ of $x^2-2$ and compute its minimal polynomial:
\begin{sagecell}
k.<a> = NumberField(x^2 - 2)
a^2 - 2
\end{sagecell}
\begin{sageout}
0
\end{sageout}
\begin{sagecell}
a.minpoly()
\end{sagecell}
\begin{sageout}
x^2 - 2
\end{sageout}
\end{sagecode}
%link
Finally we compute the minimal polynomial of $\alpha=\sqrt{2}/2 + 3$, which
is not integral, hence Proposition~\ref{prop:intfg} implies that $\alpha$
is not an algebraic integer:
%link
\begin{sagecode}
\begin{sagecell}
(a/2 + 3).minpoly()
\end{sagecell}
\begin{sageout}
x^2 - 6*x + 17/2
\end{sageout}
\end{sagecode}
\end{example}

The only elements of $\Q$ that are algebraic
integers are the usual integers $\Z$, since $\Z[1/d]$ is not finitely
generated as a $\Z$-module.  Watch out since there are elements of
$\Qbar$ that seem to {\em appear} to have denominators when written down, but are still
algebraic integers.  
This is an artifact of how we write them down, e.g., if we wrote
our integers as a multiple of $\alpha=2$, then we would write $1$
as $\alpha/2$.
For example,
$$
  \alpha = \frac{1+\sqrt{5}}{2}
$$
is an algebraic integer, since it is a root of the monic integral
polynomial $x^2 - x - 1$.  We verify this using \sage below,
though of course this is easy to do by hand (you should try
much more complicated examples in \sage).
\begin{sagecode}
\begin{sagecell}
k.<a> = QuadraticField(5)
a^2
\end{sagecell}
\begin{sageout}
5
\end{sageout}
\begin{sagecell}
alpha = (1 + a)/2
alpha.minpoly()
\end{sagecell}
\begin{sageout}
x^2 - x - 1
\end{sageout}
\begin{sagecell}
alpha.is_integral()
\end{sagecell}
\begin{sageout}
True
\end{sageout}
\end{sagecode}

Since $\sqrt{5}$ can be expressed in terms of radicals, we can also
compute this minimal polynomial using the symbolic functionality in
Sage.
\begin{sagecode}
\begin{sagecell}
alpha = (1+sqrt(5))/2
alpha.minpoly()
\end{sagecell}
\begin{sageout}
x^2 - x - 1
\end{sageout}
Here is a more complicated example using a similar approach:
\begin{sagecell}
alpha = sqrt(2) + 3^(1/4)
alpha.minpoly()
\end{sagecell}
\begin{sageout}
x^8 - 8*x^6 + 18*x^4 - 104*x^2 + 1
\end{sageout}
\end{sagecode}

\begin{example}
  We illustrate an example of a sum and product of two algebraic
  integers being an algebraic integer.  We first make the relative number
  field obtained by adjoining a root of $x^3 - 5$ to the field
  $\Q(\sqrt{2})$:
\begin{sagecode}
\begin{sagecell}
k.<a, b> = NumberField([x^2 - 2, x^3 - 5])
k
\end{sagecell}
\begin{sageout}
Number Field in a with defining polynomial x^2 + -2 over its base field
\end{sageout}
\end{sagecode}
%link
\noindent Here $a$ and $b$ are roots of $x^2-2$ and $x^3-5$, respectively.
%link
\begin{sagecode}
\begin{sagecell}
a^2
\end{sagecell}
\begin{sageout}
2
\end{sageout}
\begin{sagecell}
b^3
\end{sagecell}
\begin{sageout}
5
\end{sageout}
\end{sagecode}

\noindent We compute the minimal polynomial of the sum and product of
$\sqrt[3]{5}$ and $\sqrt{2}$.  The command {\tt absolute\_minpoly}
gives the minimal polynomial of the element over the rational numbers.
%link
\begin{sagecode}
\begin{sagecell}
(a+b).absolute_minpoly()
\end{sagecell}
\begin{sageout}
x^6 - 6*x^4 - 10*x^3 + 12*x^2 - 60*x + 17
\end{sageout}
\begin{sagecell}
(a*b).absolute_minpoly()
\end{sagecell}
\begin{sageout}
x^6 - 200
\end{sageout}
\end{sagecode}
The minimal polynomial of the product is $\sqrt[3]{5} \sqrt{2}$ is
trivial to compute by hand.  In light of the Cayley-Hamilton theorem,
we can compute the minimal polynomial of $\alpha = \sqrt[3]{5} +
\sqrt{2}$ by hand by computing the determinant of the matrix given by
left multiplication by $\alpha$ on the basis
$$
 1,\sqrt{2}, \sqrt[3]{5}, \sqrt[3]{5}\sqrt{2}, \sqrt[3]{5}^2, \sqrt[3]{5}^2\sqrt{2}.
$$

The following is an alternative, more symbolic way to compute the
minimal polynomials above, though it is not provably correct.  We
compute $\alpha$ to 100 bits precision (via the {\tt n} command), then
use the LLL algorithm (via the {\tt algdep} command) to heuristically
find a linear relation between the first $6$ powers of $\alpha$ (see
Section~\ref{sec:LLL} below for more about LLL).
\begin{sagecode}
\begin{sagecell}
a = 5^(1/3); b = sqrt(2)
c = a+b; c
\end{sagecell}
\begin{sageout}
5^(1/3) + sqrt(2)
\end{sageout}
\begin{sagecell}
(a+b).n(100).algdep(6)
\end{sagecell}
\begin{sageout}
x^6 - 6*x^4 - 10*x^3 + 12*x^2 - 60*x + 17
\end{sageout}
\begin{sagecell}
(a*b).n(100).algdep(6)
\end{sagecell}
\begin{sageout}
x^6 - 200
\end{sageout}
\end{sagecode}
\end{example}

\begin{exercise}
	Let $\alpha = \sqrt{2} + \frac{1+\sqrt{5}}{2}$. 
	\begin{enumerate}
	\item Is $\alpha$ an algebraic integer?
	\item Explicitly write down the minimal polynomial of $\alpha$
	as an element of $\QQ[x]$.
	\end{enumerate}
\end{exercise}




\subsection{Number fields, rings of integers, and orders}
\begin{definition}[Number field]
  A \defn{number field} is a field~$K$ that contains the rational
  numbers $\QQ$ such that the degree $[K:\Q] = \dim_\Q(K)$ is finite.
\end{definition}

If $K$ is a number field, then by the primitive element theorem there
is an $\alpha \in K$ so that $K = \QQ(\alpha)$.  Let $f(x) \in \QQ[x]$
be the minimal polynomial of $\alpha$.  Fix a choice of algebraic
closure $\Qbar$ of $\Q$.  Associated to each of the $\deg(f)$ roots
$\alpha'\in\Qbar$ of $f$, we obtain a field embedding $K\hra \Qbar$
that sends $\alpha$ to $\alpha'$.  Thus any number field can be
embedded in $[K:\Q]=\deg(f)$ distinct ways in $\Qbar$.

\begin{definition}[Ring of Integers]
The \defn{ring of integers} of a number field~$K$ is the ring 
$$
  \O_K = \{x \in K : \text{ $x$ satisfies a monic polynomial with integer coefficients}\}.
$$
\end{definition}
Proposition~\ref{prop:zbarring} implies that $\O_K$ is a ring.

\begin{example}
The field $\Q$ of rational numbers is a number field of degree $1$,
and the ring of integers of $\Q$ is $\Z$.  The field $K=\Q(i)$ of
Gaussian integers has degree $2$ and $\O_K = \Z[i]$.  
\end{example}

\begin{example}\label{example:Qsqrt5ringofints}
The golden ratio $\varphi =(1+\sqrt{5})/2$ is in the quadratic number field
$K=\Q(\sqrt{5})=\Q(\varphi)$; notice that
$\varphi$ satisfies $x^2-x-1$, so $\varphi\in\O_K$.
To see that $\O_K = \Z[\varphi]$ directly, we proceed as follows.
By Proposition~\ref{prop:intfg}, the algebraic integers $K$
are exactly the elements $a+b\sqrt{5} \in K$, with $a,b\in\Q$
that have integral minimal polynomial.   The matrix of $a+b\sqrt{5}$  with respect to the 
basis $1,\sqrt{5}$ for $K$ is
$m=\abcd{a}{5b}{b}{a}$.
The characteristic polynomial of $m$ is $f = (x-a)^2 - 5b^2 = x^2 - 2ax + a^2 - 5b^2$,
which is in $\Z[x]$ if and only if $2a\in\Z$ and $a^2-5b^2\in\Z$.
Thus $a=a'/2$ with $a'\in\Z$, and 
$(a'/2)^2 - 5b^2 \in\Z$, so $5b^2 \in \frac{1}{4}\Z$, so $b\in\frac{1}{2}\Z$ as well.
If $a$ has a denominator of $2$, then $b$ must also have a denominator of $2$ to ensure that the difference
$a^2-5b^2$ is an integer.  This proves that $\O_K = \Z[\varphi]$.  
\end{example}

\begin{example}
The
ring of integers of $K=\Q(\sqrt[3]{9})$ is $\Z[\sqrt[3]{3}]$, where
$\sqrt[3]{3}=\frac{1}{3}(\sqrt[3]{9})^2 \not \in \sqrt[3]{9}$.  As we
will see, in general the problem of computing $\O_K$ given $K$ may be
very hard, since it requires factoring a certain potentially large
integer.
\end{example}

\begin{exercise}
	From basic definitions, find the rings of integers of the fields
	$\Q(\sqrt{11})$ and $\Q(\sqrt{-6})$.
\end{exercise}

\begin{definition}[Order]\label{defn:order}
An \defn{order} in $\O_K$ is any subring $R$ of $\O_K$ such that the
quotient $\O_K/R$ of abelian groups is finite.  
(By definition $R$ must contain $1$ because it is a ring.)
\end{definition}
As noted above, $\Z[i]$ is the ring of integers of $\Q(i)$.  For every
nonzero integer~$n$, the subring $\Z+ni\Z$ of $\Z[i]$ is an order.
The subring $\Z$ of $\Z[i]$ is not an order, because $\Z$ does not
have finite index in $\Z[i]$.  Also the subgroup $2\Z + i\Z$ of
$\Z[i]$ is not an order because it is not a ring.  

We define the number field $\Q(i)$ and compute its
ring of integers.
\begin{sagecode}
\begin{sagecell}
K.<i> = NumberField(x^2 + 1)
OK = K.ring_of_integers(); OK
\end{sagecell}
\begin{sageout}
Order with module basis 1, i in Number Field in i with 
defining polynomial x^2 + 1
\end{sageout}
\end{sagecode}
%link

\noindent Next we compute the order $\Z + 3i \Z$.
%link
\begin{sagecode}
\begin{sagecell}
O3 = K.order(3*i); O3
\end{sagecell}
\begin{sageout}
Order with module basis 1, 3*i in Number Field in i with 
defining polynomial x^2 + 1
\end{sageout}
\begin{sagecell}
O3.gens()
\end{sagecell}
\begin{sageout}
[1, 3*i]
\end{sageout}
\end{sagecode}
%link


\noindent We test whether certain elements are in the order.
%link
\begin{sagecode}
\begin{sagecell}
5 + 9*i in O3
\end{sagecell}
\begin{sageout}
True
\end{sageout}
\begin{sagecell}
1 + 2*i in O3
\end{sagecell}
\begin{sageout}
False
\end{sageout}
\end{sagecode}


We will frequently consider orders because they are often much easier
to write down explicitly than $\O_K$.  For example, if $K=\Q(\alpha)$
and $\alpha$ is an algebraic integer, then $\Z[\alpha]$ is an order in
$\O_K$, but frequently $\Z[\alpha]\neq \O_K$.

\begin{example}
In this example $[\O_K : \Z[a]] = 2197$.  First we define
the number field $K=\Q(a)$ where $a$ is a root of $x^3 - 15 x^2 - 94 x - 3674$,
then we compute the order $\Z[a]$ generated by $a$.
\begin{sagecode}
\begin{sagecell}
K.<a> = NumberField(x^3 - 15*x^2 - 94*x - 3674)
Oa = K.order(a); Oa
\end{sagecell}
\begin{sageout}
Order with module basis 1, a, a^2 in Number Field in a with defining 
polynomial x^3 - 15*x^2 - 94*x - 3674
\end{sageout}
\begin{sagecell}
Oa.basis()
\end{sagecell}
\begin{sageout}
[1, a, a^2]
\end{sageout}
\end{sagecode}
%link

\noindent Next we compute a $\Z$-basis for the maximal order $\O_K$ of $K$, and
compute that the index of $\Z[a]$ in $\O_K$ is $2197=13^3$.
%link
\begin{sagecode}
\begin{sagecell}
OK = K.maximal_order()
OK.basis()
\end{sagecell}
\begin{sageout}
[25/169*a^2 + 10/169*a + 1/169, 5/13*a^2 + 1/13*a, a^2]
\end{sageout}
\begin{sagecell}
Oa.index_in(OK)
\end{sagecell}
\begin{sageout}
2197
\end{sageout}
\end{sagecode}
\end{example}

\begin{lemma}\label{lem:intq}\ilem{$\O_K$ span and $\O_K\cap \Q=\Z$}
Let $\O_K$ be the ring of integers of a number field.  Then 
$\O_K\cap \Q = \Z$ and $\Q\O_K = K$.
\end{lemma}
\begin{proof}
  Suppose $\alpha\in \O_K\cap\Q$ with $\alpha=a/b \in \Q$ in lowest
  terms and $b>0$.  Since $\alpha$ is integral, $\Z[a/b]$ is finitely
  generated as a module, so $b=1$.

To prove that $\Q\O_K=K$, suppose $\alpha\in K$, and let
$f(x)\in\Q[x]$ be the minimal monic polynomial of~$\alpha$.  For any
positive integer~$d$, the minimal monic polynomial of $d\alpha$ is
$d^{\deg(f)}f(x/d)$, i.e., the polynomial obtained from $f(x)$ by
multiplying the coefficient of $x^{\deg(f)}$ by~$1$, multiplying the
coefficient of $x^{\deg(f)-1}$ by~$d$, multiplying the coefficient of
$x^{\deg(f)-2}$ by $d^2$, etc.  If~$d$ is the least common multiple of
the denominators of the coefficients of~$f$, then the minimal monic
polynomial of $d\alpha$ has integer coefficients, so $d\alpha$ is
integral and $d\alpha\in \O_K$.  This proves that $\Q\O_K = K$.
\end{proof}

\begin{exercise}
	Which are the following rings are orders in the given
	number field, i.e. orders in the ring of integers of the
	given number field.
	\begin{enumerate}
	\item The ring $R = \ZZ[i]$ in the number field $\QQ(i)$.
	\item The ring $R = \ZZ[i/2]$ in the number field $\QQ(i)$.
	\item The ring $R = \ZZ[17i]$ in the number field $\QQ(i)$.
	\item The ring $R = \ZZ[i]$ in the number field $\QQ(\sqrt[4]{-1})$.
	\end{enumerate}
\end{exercise}

\subsection{Function fields}
Let $k$ be any field.  We can also make the same definitions, but with $\Q$
replaced by the field $k(t)$ of rational functions in an indeterminate
$t$, and $\Z$ replaced by $k[t]$.
The analogue of a number field is called a {\em function field}; it is
a finite algebraic extension field $K$ of $k(t)$.  Elements of $K$
have a unique minimal polynomial as above, and the ring of integers of
$K$ consists of those elements whose monic minimal polynomial has
coefficients in the polynomial ring $k[t]$.  

Geometrically, if $F(x,t)=0$ is an affine equation that defines (via
projective closure) a nonsingular projective curve $C$, then
$K=k(t)[x]/(F(x,t))$ is a function field.  We view the field $K$ as
the field of all rational functions on the projective closure of the
curve $C$.  The ring of integers $\O_K$ is the subring of rational
functions that have no poles on the affine curve $F(x,t)=0$, though
they may have poles at infinity, i.e., at the extra points we
introduce when passing to the projective closure $C$.  The algebraic
arguments we gave above prove that $\O_K$ is a ring.  This is also
geometrically intuitive, since the sum and product of two functions
with no poles also have no poles.

\begin{exercise}
	Let $k = \F_p$ be the finite field with $p$ elements where $p$ is some prime. Find all automorphisms of $k(t)$. Note that an automorphism is completely characterized by its value on $t$. How many such automorphisms are there?
	
	\begin{hint}
		For some people, it is easier to think about the equivalent question: What rational functions $f\in k(t)$ is the map $k(t)\to k(t)$ given by $t\mapsto f(t)$ an automorphism?
	\end{hint}
\end{exercise}

\section{Norms and Traces}
%\setcounter{section}{1}
In this section we develop some basic properties of norms, traces, and
discriminants, and give more properties of rings of integers in the
general context of Dedekind domains.

Before discussing norms and traces we introduce some notation for
field extensions.  If $K\subset L$ are number fields, we let $[L:K]$
denote the dimension of~$L$ viewed as a $K$-vector space.  If~$K$ is a
number field and $a\in \Qbar$, let $K(a)$ be the extension of~$K$
generated by~$a$, which is the smallest number field that contains
both~$K$ and~$a$.  If $a\in\Qbar$ then~$a$ has a minimal polynomial
$f(x)\in\Q[x]$, and the \defn{Galois conjugates} of~$a$ are the roots
of~$f$. These are called the Galois conjugates because they are the orbit
of $a$ under the action of $\Gal(\Qbar/\Q)$.  

\begin{example}
The element $\sqrt{2}$ has minimal polynomial $x^2-2$ and the Galois
conjugates of $\sqrt{2}$ are $\sqrt{2}$ and $-\sqrt{2}$.  The cube root $\sqrt[3]{2}$
has minimial polynomial $x^3 - 2$ and three Galois conjugates 
$\sqrt[3]{2}, \zeta_3\sqrt[3]{2}, \zeta_3^2\sqrt[3]{2}$, where
$\zeta_3$ is a cube root of unity. 

We create the extension $\Q(\zeta_3)(\sqrt[3]{2})$ in \sage.
\begin{sagecode}
\begin{sagecell}
L.<cuberoot2> = CyclotomicField(3).extension(x^3 - 2)
cuberoot2^3
\end{sagecell}
\begin{sageout}
2
\end{sageout}
\end{sagecode}
%link
\noindent Then we list the Galois conjugates of $\sqrt[3]{2}$. 
%link
\begin{sagecode}
\begin{sagecell}
cuberoot2.galois_conjugates(L)
\end{sagecell}
\begin{sageout}
[cuberoot2, (-zeta3 - 1)*cuberoot2, zeta3*cuberoot2]
\end{sageout}
\end{sagecode}
%link
\noindent Note that $\zeta_3^2 = -\zeta_3 - 1$:
%link
\begin{sagecode}
\begin{sagecell}
zeta3 = L.base_field().0
zeta3^2
\end{sagecell}
\begin{sageout}
-zeta3 - 1
\end{sageout}
\end{sagecode}
\end{example}

Suppose $K\subset L$ is an inclusion of number fields and let $a\in
L$.  Then left multiplication by~$a$ defines a $K$-linear
transformation $\ell_a:L\to L$.  (The transformation $\ell_a$ is
$K$-linear because $L$ is commutative.)

\begin{definition}[Norm and Trace]\label{defn:normtrace}
The \defn{norm} and \defn{trace} of~$a$ from~$L$ to~$K$ are
$$\Norm_{L/K}(a)=\det(\ell_a) \quad\text{ and }\quad
 \tr_{L/K}(a)=\tr(\ell_a).$$
\end{definition}
We know from linear algebra that 
determinants are multiplicative
and traces are additive, so for $a,b\in L$ we have
$$\Norm_{L/K}(ab) = \Norm_{L/K}(a)\cdot \Norm_{L/K}(b)$$
and
$$\tr_{L/K}(a+b) = \tr_{L/K}(a) + \tr_{L/K}(b).$$

Note that if $f\in\Q[x]$ is the characteristic polynomial of~$\ell_a$,
then the constant term of $f$ is $(-1)^{\deg(f)}\det(\ell_a)$, and the
coefficient of $x^{\deg(f)-1}$ is $-\tr(\ell_a)$.

\begin{proposition}\label{prop:normtracesigma}\iprop{norm and trace}
Let $a\in L$ and let $\sigma_1,\ldots, \sigma_d$, where $d=[L:K]$, be
the distinct field embeddings $L\hra \Qbar$ that fix every element
of~$K$.  Then
$$
\Norm_{L/K}(a) = \prod_{i=1}^d \sigma_i(a)
\quad\text{ and }\quad
\tr_{L/K}(a) = \sum_{i=1}^d \sigma_i(a).
$$
\end{proposition}
\begin{proof}
  We prove the proposition by computing the characteristic
  polynomial of~$a$.  Let $f\in K[x]$ be the minimal polynomial
  of~$a$ over~$K$, and note that~$f$ has distinct roots and is
  irreducible, since it is the polynomial in $K[x]$ of least degree
  that is satisfied by~$a$ and~$K$ has characteristic~$0$.  Since~$f$
  is irreducible, we have $K(a) \isom K[x]/(f)$, so $[K(a):K]=\deg(f)$.
  Also~$a$ satisfies a polynomial if and only if~$\ell_a$ does, so the
  characteristic polynomial of $\ell_a$ acting on $K(a)$ is~$f$.  Let
  $b_1,\ldots,b_n$ be a basis for $L$ over $K(a)$ and note that
  $1,\ldots, a^m$ is a basis for $K(a)/K$, where $m=\deg(f)-1$.  Then
  $a^i b_j$ is a basis for $L$ over $K$, and left multiplication by
  $a$ acts the same way on the span of $b_j, a b_j, \ldots, a^m b_j$
  as on the span of $b_k, a b_k, \ldots, a^m b_k$, for any pair $j,
  k\leq n$.  Thus the matrix of $\ell_a$ on $L$ is a block direct sum
  of copies of the matrix of $\ell_a$ acting on $K(a)$, so the
  characteristic polynomial of $\ell_a$ on~$L$ is $f^{[L:K(a)]}$.  The
  proposition follows because the roots of $f^{[L:K(a)]}$ are exactly
  the images $\sigma_i(a)$, with multiplicity $[L:K(a)]$, since each
  embedding of $K(a)$ into $\Qbar$ extends in exactly $[L:K(a)]$ ways
  to $L$.
\end{proof}

It is important in Proposition~\ref{prop:normtracesigma} that
the product and sum be over {\em all} the images $\sigma_i(a)$,
not over just the distinct images.  For example, if $a=1\in L$, then
$\Tr_{L/K}(a) = [L:K]$, whereas the sum of the distinct conjugates
of $a$ is $1$.

The following corollary asserts that the norm and trace behave well in
towers.
\begin{corollary}\icor{norm, trace compatible with towers}\label{cor:compatower}
Suppose $K\subset L \subset M$ is a tower of number fields, and
let $a\in M$.  Then 
$$
\Norm_{M/K}(a) = \Norm_{L/K}(\Norm_{M/L}(a))
\quad\text{ and }\quad
\tr_{M/K}(a) = \tr_{L/K}(\tr_{M/L}(a)).
$$
\end{corollary}
\begin{proof}
The proof uses that every embedding $L\hra \Qbar$ extends in exactly
$[M:L]$ way to an embedding $M\hra \Qbar$.  This is clear
if we view $M$ as $L[x]/(h(x))$ for some irreducicble
polynomial $h(x) \in L[x]$ of degree $[M:L]$, and note that 
the extensions of $L\hra \Qbar$ to $M$ correspond to
the roots of $h$, of which there are $\deg(h)$, since $\Qbar$
is algebraically closed.

  For the first equation, both sides are the product of $\sigma_i(a)$,
  where $\sigma_i$ runs through the embeddings of~$M$ into $\Qbar$
  that fix $K$.  To see this, suppose $\sigma:L\to \Qbar$ fixes $K$.
  If $\sigma'$ is an extension of $\sigma$ to $M$, and $\tau_1,\ldots,
  \tau_d$ are the embeddings of $M$ into $\Qbar$ that fix $L$, then
  $\sigma'\tau_1,\ldots,\sigma'\tau_d$ are exactly the extensions
  of~$\sigma$ to~$M$.  For the second statement, both sides are the
  sum of the $\sigma_i(a)$.
\end{proof}

Let $K\subset L\subset M$ be as in Corollary~\ref{cor:compatower}.  If
$\alpha\in M$, then the formula of
Proposition~\ref{prop:normtracesigma} implies that the norm and trace
down to~$L$ of $\alpha$ is an element of~$\O_L$, because the sum and
product of algebraic integers is an algebraic integer.

\begin{proposition}\label{prop:ok_lattice}\iprop{$\O_K$ is a lattice}
Let~$K$ be a number field.  The ring of integers $\O_K$ is a lattice
in~$K$, i.e., $\Q\O_K=K$ and $\O_K$ is an abelian group of rank $[K:\Q]$.
\end{proposition}
\begin{proof}
We saw in Lemma~\ref{lem:intq} that $\Q\O_K=K$.  Thus there exists a
basis $a_1,\ldots, a_n$ for~$K$, where each~$a_i$ is in $\O_K$.
Suppose that as $x=\sum_{i=1}^n c_i a_i\in \O_K$ varies over all
elements of $\O_K$ the denominators of the coefficients $c_i$ are not
all uniformly bounded.  Then subtracting off integer multiples of the
$a_i$, we see that as $x=\sum_{i=1}^n c_i a_i\in \O_K$ varies over
elements of $\O_K$ with $c_i$ between $0$ and $1$, the denominators of
the $c_i$ are also arbitrarily large.  This implies that there are
infinitely many elements of $\O_K$ in the bounded subset
$$S = \left\{c_1 a_1 +\cdots + c_n a_n : c_i \in \Q,\, 0\leq c_i \leq 1\right\}\subset K.$$
Thus for any $\eps>0$, there are elements $a,b\in \O_K$ such that the
coefficients of $a-b$ are all less than $\eps$ (otherwise the elements
of $\O_K$ would all be a ``distance'' 
of least~$\eps$ from each other, so only finitely
many of them would fit in~$S$).

As mentioned above, the norms of elements of $\O_K$ are integers.
Since the norm of an element is the determinant of left multiplication
by that element, the norm is a homogenous polynomial of degree $n$ in
the indeterminate coefficients $c_i$, which is $0$ only on the
element~$0$, so the constant term of this polynomial is $0$.  If the $c_i$ get arbitrarily small for elements of
$\O_K$, then the values of the norm polynomial get arbitrarily small,
which would imply that there are elements of $\O_K$ with positive norm
too small to be in $\Z$, a contradiction.  So the set $S$ contains
only finitely many elements of $\O_K$.  Thus the denominators of the
$c_i$ are bounded, so for some $d$, we have that $\O_K$ has finite
index in $A=\frac{1}{d}\Z a_1 + \cdots + \frac{1}{d}\Z a_n$.  Since
$A$ is isomorphic to $\Z^n$, it follows from the structure theorem for
finitely generated abelian groups that $\O_K$ is isomorphic as a
$\Z$-module to $\Z^n$, as claimed.
\end{proof}

\begin{corollary}\label{prop:intnoetherian}\icor{$\O_K$
is noetherian}
The ring of integers $\O_K$ of a number field is noetherian.
\end{corollary}
\begin{proof}
By Proposition~\ref{prop:ok_lattice}, the ring $\O_K$ is
finitely generated as a module over $\Z$, so it is certainly
finitely generated as a ring over $\Z$.  By Theorem~\ref{thm:hilbert},
 $\O_K$ is noetherian.
\end{proof}


\section{Recognizing Algebraic Numbers using LLL}\label{sec:LLL}

Suppose we somehow compute a decimal approximation $\alpha$
to some rational number $\beta \in \QQ$ and from this wish
to recover $\beta$.  For concreteness, say
$$\beta= 22/389 = 0.05655526992287917737789203084832904884318766066838046\ldots$$
and we compute
$$
  \alpha = 0.056555.
$$
Now suppose given only $\alpha$ that you would like to recover
$\beta$.  A standard technique is to use continued fractions, which
yields a sequence of good rational approximations
for $\alpha$; by truncating right before a surprisingly big
partial quotient, we obtain $\beta$:
\begin{sagecode}
\begin{sagecell}
v = continued_fraction(0.056555)
continued_fraction(0.056555)
\end{sagecell}
\begin{sageout}
[0, 17, 1, 2, 6, 1, 23, 1, 1, 1, 1, 1, 2]
\end{sageout}
\begin{sagecell}
convergents([0, 17, 1, 2, 6, 1])
\end{sagecell}
\begin{sageout}
[0, 1/17, 1/18, 3/53, 19/336, 22/389]
\end{sageout}
\end{sagecode}

   
Generalizing this, suppose next that somehow you numerically approximate
an algebraic number, e.g., by evaluating a special
function and get a decimal approximation $\alpha \in \C$
to an algebraic number $\beta \in \Qbar$.  For concreteness,
suppose $\beta = \frac{1}{3} + \sqrt[4]{3}$:
\begin{sagecode}
\begin{sagecell}
N(1/3 + 3^(1/4), digits=50)
\end{sagecell}
\begin{sageout}
1.64940734628582579415255223513033238849340192353916
\end{sageout}
\end{sagecode}
Now suppose you very much want to find the (rescaled) 
minimal polynomial $f(x) \in \ZZ[x]$
of $\beta$ just given this numerical approximation $\alpha$.
This is of great value even without proof, since often in practice
once you know a potential minimal polynomial you can
verify that it is in fact right.  Exactly this situation
arises in the explicit construction of class fields (a
more advanced topic in number theory) and in the construction
of Heegner points on elliptic curves.   As we will see, the
LLL algorithm provides a polynomial time way to solve this
problem, assuming $\alpha$ has been computed to
sufficient precision. 

\subsection{LLL Reduced Basis}
Given a basis $b_1,\ldots, b_n$ for $\R^n$, the \defn{Gramm-Schmidt
orthogonalization} process produces an orthogonal basis 
$b_1^*,\ldots, b_n^*$ for $\R^n$ as follows.  Define inductively
$$
   b_i^* = b_i - \sum_{j < i} \mu_{i,j} b_j^*
$$
where
$$
\mu_{i,j} = \frac{b_i \cdot b_j^*}{b_j^* \cdot b_j^*}.
$$

\begin{example}
We compute the Gramm-Schmidt orthogonal basis of the rows
of a matrix.  Note that no square roots are introduced
in the process; there would be square roots if we
constructed an orthonormal basis. 
\begin{sagecode}
\begin{sagecell}
A = matrix(ZZ, 2, [1,2, 3,4]); A
\end{sagecell}
\begin{sageout}
[1 2]
[3 4]
\end{sageout}
\begin{sagecell}
Bstar, mu = A.gramm_schmidt()
\end{sagecell}
\end{sagecode}
%link

\noindent{}The rows of the matrix $B^*$ are obtained
from the rows of $A$ by the Gramm-Schmidt procedure. 
%link
\begin{sagecode}
\begin{sagecell}
Bstar
\end{sagecell}
\begin{sageout}
[   1    2]
[ 4/5 -2/5]
\end{sageout}
\begin{sagecell}
mu
\end{sagecell}
\begin{sageout}
[   0    0]
[11/5    0]
\end{sageout}
\end{sagecode}
\end{example}

A \defn{lattice} $L\subset \R^n$ is a subgroup that
is free of rank $n$ such that $\R L = \R^n$. 

\begin{definition}[LLL-reduced basis]
The basis $b_1,\ldots, b_n$ for a lattice $L\subset \R^n$
is \defn{LLL reduced} if for all $i,j$, 
$$
   |\mu_{i,j}| \leq \frac{1}{2}
$$   
and for each $i\geq 2$, 
$$
  |b_i^*|^2 \geq \left( \frac{3}{4} - \mu_{i,i-1}^2\right) | b_{i-1}^*|^2
$$

\end{definition}

For example, the basis $b_1 = (1,2)$, $b_2 = (3,4)$ for a lattice $L$
 is {\em not}
LLL reduced because $b_1^*=b_1$ and 
$$
   \mu_{2,1} = \frac{b_2 \cdot b_1^*}{b_1^* \cdot b_1^*}
       = \frac{11}{5} > \frac{1}{2}.
$$
However, the basis $b_1 = (1,0)$, $b_2 = (0,2)$ for $L$ is
LLL reduced, since
$$
 \mu_{2,1} = \frac{b_2 \cdot b_1^*}{b_1^* \cdot b_1^*}
       = 0,
$$
and
$$
   2^2 \geq (3/4) \cdot 1^2.
$$
\begin{sagecode}
\begin{sagecell}
A = matrix(ZZ, 2, [1,2, 3,4])
A.LLL()
\end{sagecell}
\begin{sageout}
[1 0]
[0 2]
\end{sageout}
\end{sagecode}

\subsection{What LLL really means}
The following theorem is not too difficult to prove.

Let $b_1, \ldots, b_n$ be an LLL reduced basis
for a lattice $L\subset \R^n$.  Let $d(L)$ denote
the absolute value of the determinant of any matrix
whose rows are basis for $L$.   Then the vectors
$b_i$ are ``nearly orthogonal'' and ``short'' in 
the sense of the following theorem:
\begin{theorem}
We have
\begin{enumerate}
\item $d(L) \leq \prod_{i=1}^n |b_i| \leq 2^{n(n-1)/4} d(L)$.
\item For $1\leq j \leq i \leq n$, we have
$$
 |b_j| \leq 2^{(i-1)/2} |b_i^*|.
$$
\item The vector $b_1$ is very short in the sense that
$$
  |b_1| \leq 2^{(n-1)/4} d(L)^{1/n}
$$
and for every nonzero $x \in L$ we have
$$
  |b_1| \leq 2^{(n-1)/2} |x|.
$$
\item More generally, for any linearly independent $x_1,\ldots, x_t \in L$, we have 
$$
  |b_j| \leq 2^{(n-1)/2} \max(|x_1|, \ldots, |x_t|)
$$
for $1\leq j \leq t$. 
\end{enumerate}
\end{theorem}

Perhaps the most amazing thing about the idea of an LLL
reduced basis is that there is an algorithm (in fact many)
that given a basis for a lattice $L$ produce an LLL reduced
basis for $L$, and do so {\em quickly}, i.e., in polynomial
time in the number of digits of the input.   The current
optimal implementation (and practically optimal algorithms)
for computing LLL reduced basis are due to Damien Stehle,
and are included standard in Magma in Sage.   Stehle's code
is amazing -- it can LLL reduce a random lattice in $\R^{n}$
for $n<1000$ in a matter of minutes!
\begin{sagecode}
\begin{sagecell}
A = random_matrix(ZZ, 200)
t = cputime()
B = A.LLL()     
cputime(t)     # random output
\end{sagecell}
\begin{sageout}
3.0494159999999999
\end{sageout}
\end{sagecode}
%link

\noindent{}There is even a very fast variant of Stehle's implementation that computes a basis for $L$ that is very likely LLL reduced but may in rare
cases fail to be LLL reduced.

%link
\begin{sagecode}
\begin{sagecell}
t = cputime()
B = A.LLL(algorithm="fpLLL:fast")   # not tested
cputime(t)      # random output
\end{sagecell}
\begin{sageout}
0.96842699999999837
\end{sageout}
\end{sagecode}

\subsection{Applying LLL}
The LLL definition and algorithm has many application in 
number theory, e.g., to cracking lattice-based cryptosystems,
to enumerating all short vectors in a lattice, to finding relations
between decimal approximations to complex numbers, to very
fast univariate polynomial factorization in $\ZZ[x]$ and more
generally in $K[x]$ where $K$ is a number fields, and to
computation of kernels and images of integer matrices.  LLL
can also be used to solve the problem of recognizing algebraic
numbers mentioned at the beginning of Section~\ref{sec:LLL}.

Suppose as above that $\alpha$ is a decimal approximation
to some algebraic number $\beta$, and to for simplicity
assume that $\alpha\in\R$ (the general case of $\alpha \in \C$
is described in \cite{cohen:course_ant}). 
We finish by explaining how to use LLL to find a polynomial 
$f(x) \in \ZZ[x]$ such that $f(\alpha)$ is small, hence
has a shot at being the minimal polynomial of $\beta$. 

Given a real number decimal approximation $\alpha$, an
integer $d$ (the degree), and an integer $K$ (a function
of the precision to which $\alpha$ is known), the following
steps produce a polynomial $f(x) \in \ZZ[x]$ of degree 
at most $d$ such that $f(\alpha)$ is small.  
\begin{enumerate}
\item Form the lattice in $\R^{d+2}$ with basis the rows
of the matrix $A$ whose first $(d+1) \times (d+1)$ part is the
identity matrix, and whose last column has entries
\begin{equation}\label{eqn:K}
K, \lfloor K\alpha \rfloor, \lfloor K\alpha^2 \rfloor, 
\ldots, \lfloor K\alpha^{d} \rfloor.
\end{equation}
(Note this matrix is $(d+1) \times (d+2)$ so the lattice
is not of full rank in $\R^{d+2}$, which isn't a problem,
since the LLL definition also makes sense for less vectors.)
\item Compute an LLL reduced basis for the $\ZZ$-span of the rows
of $A$, and let $B$ be the corresponding matrix. 
Let $b_1 = (a_0, a_1, \ldots, a_{d+1})$ be the first
row of $B$ and notice that $B$ is obtained from $A$
by left multiplication by an invertible integer matrix.
Thus $a_0,\ldots, a_d$ are the linear combination of the
(\ref{eqn:K}) that equals $a_{d+1}$. Moreover, since $B$
is LLL reduced we expect that $a_{d+1}$ is relatively small. 

\item Output $f(x) = a_0 + a_1 x + \cdots + a_{d} x^d$.
We have that $f(\alpha) \sim a_{d+1}/K$, which is small.
Thus $f(x)$ may be a very good candidate for the minimal
polynomial of $\beta$ (the algebraic number we are approximating),
assuming $d$ was chosen minimally and $\alpha$ was computed
to sufficient precision.
\end{enumerate}

The following is a complete implementation of the above algorithm
in Sage:
\begin{sagecode}
\begin{sagecell}
def myalgdep(a, d, K=10^6):
    aa = [floor(K*a^i) for i in range(d+1)]
    A = identity_matrix(ZZ, d+1)
    B = matrix(ZZ, d+1, 1, aa)
    A = A.augment(B)
    L = A.LLL()
    v = L[0][:-1].list()
    return ZZ['x'](v)
\end{sagecell}
\end{sagecode}
%link

Here is an example of using it:
%link
\begin{sagecode}
\begin{sagecell}
R.<x> = RDF[]
f = 2*x^3 - 3*x^2 + 10*x - 4
a = f.roots()[0][0]; a
myalgdep(a, 3, 10^6)       # not tested
\end{sagecell}
\begin{sageout}
2*x^3 - 3*x^2 + 10*x - 4
\end{sageout}
\end{sagecode}


%%% Local Variables: 
%%% mode: latex
%%% TeX-master: "ant"
%%% End: 