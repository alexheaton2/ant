\chapter{Valuations}
The rest of this book is a partial rewrite of \cite{cassels:global}
meant to make it more accessible.  I have attempted to add examples
and details of the implicit exercises and remarks that are left to the
reader.

\section{Valuations}

\begin{definition}[Valuation]
A \defn{valuation} $\absspc{}$ on a field $K$ is a function
defined on $K$ with values in $\R_{\geq 0}$ satisfying
the following axioms:
\begin{enumerate}
\item[(1)] $\abs{a} = 0$ if and only if $a = 0$,
\item[(2)] $\abs{ab}=\abs{a}\abs{b}$, and
\item[(3)] there is a constant $C\geq 1$ such that 
$\abs{1+a}\leq C$ whenever $\abs{a}\leq 1$.
\end{enumerate}
\end{definition}

The \defn{trivial valuation} is the valuation for which
$\abs{a}=1$ for all $a\neq 0$.  We will often tacitly
exclude the trivial valuation from consideration.

From (2) we have 
$$
  \abs{1} = \abs{1}\cdot \abs{1},
$$
so $\abs{1} = 1$ by (1).
If $w\in K$ and $w^n=1$, then $|w|=1$ by (2).
In particular, the only valuation of a finite field
is the trivial one.    The same argument shows that $|-1|=|1|$,
so 
$$
  |-a| = |a|\qquad \text{all }a \in K.
$$

\begin{definition}[Equivalent]
Two valuations $\absspc{}_1$ and $\absspc{}_2$ on the
same field are \defn{equivalent}\i{valuation!equivalence of} 
if there exists $c>0$ such
that $$\abs{a}_2 = \abs{a}_1^c$$
for all $a\in K$.
\end{definition}
Note that if $\absspc{}_1$ is a valuation, then 
$\absspc{}_2=\absspc{}_1^c$ is also a valuation.
Also, equivalence of valuations is an equivalence relation.

If $\absspc{}$ is a valuation and $C>1$ is the constant from Axiom
(3), then there is a $c>0$ such that $C^c=2$ (i.e.,
$c=\log(2)/\log(C)$).  Then we can take $2$ as constant for the
equivalent valuation $\absspc{}^c$.  Thus every valuation is
equivalent to a valuation with $C=2$. Note that if $C=1$, e.g.,
if $\absspc{}$ is the trivial valuation, then we could
simply take $C=2$ in Axiom (3).
\begin{proposition}\iprop{triangle inequality}
Suppose $\absspc{}$ is a valuation with $C\leq 2$.
Then for all $a, b\in K$ we have 
\begin{equation}\label{val3p}
  |a + b| \leq |a| + |b|\qquad \text{(triangle inequality)}.
\end{equation}
\end{proposition}
\begin{proof}
Suppose $a_1, a_2\in K$ with $|a_1|\geq|a_2|$.  Then $a=a_2/a_1$
satisfies $|a|\leq 1$.  By Axiom (3) we have $|1+a|\leq 2$, so
multiplying by $a_1$ we see that
$$|a_1+ a_2|\leq 2|a_1| = 2\cdot\max\{|a_1|,|a_2|\}.$$
Also we have 
$$|a_1+ a_2 + a_3 + a_4|\leq 2\cdot\max\{|a_1+a_2|,|a_3+a_4|\}
   \leq 4\cdot \max\{|a_1|,|a_2|,|a_3|,|a_4|\},
$$
and inductively we have for any $r>0$ that
$$|a_1 + a_2 + \cdots  + a_{2^r}| \leq 2^r\cdot\max{|a_j|}.$$
If $n$ is any positive integer, let $r$ be such
that $2^{r-1}\leq n\leq 2^r$. Thenn
$$|a_1 + a_2 + \cdots + a_{n}| \leq 2^r\cdot \max\{|a_j|\} 
 \leq 2n\cdot \max\{|a_j|\},$$
since $2^r\leq 2n$.  In particular,
\begin{equation}\label{eqn:absn}
  |n| \leq 2n\cdot |1| = 2n \qquad\text{(for $n>0$)}.
\end{equation}
Applying (\ref{eqn:absn}) to $\ds\abs{\binom{n}{j}}$ and using
the binomial expansion, we have for any $a,b\in K$ that
\begin{align*}
|a+b|^n &= \abs{\sum_{j=0}^n \binom{n}{j} a^j b^{n-j}}\\
   &\leq  2(n+1)\max_j\left\{ \abs{\binom{n}{j}} \abs{a}^j\abs{b}^{n-j}\right\}\\
   &\leq  2(n+1)\max_j\left\{ 2 \binom{n}{j} \abs{a}^j\abs{b}^{n-j}\right\}\\
   &\leq  4(n+1)\max_j\left\{ \binom{n}{j} \abs{a}^j\abs{b}^{n-j}\right\}\\
   &\leq  4(n+1)(\abs{a}+\abs{b})^n.
\end{align*}
Now take $n$th roots of both sides to obtain
$$
|a+b| \leq \sqrt[n]{4(n+1)}\cdot (|a| + |b|).
$$
We have by elementary calculus that
$$
  \lim_{n\to \infty} \sqrt[n]{4(n+1)} = 1,
$$ so $|a+b| \leq |a|+|b|$.  
(The ``elementary calculus'': We instead prove that $\sqrt[n]{n}\to 1$, since
the argument is the same and the notation is simpler.  First, for any
$n\geq 1$ we have $\sqrt[n]{n}\geq 1$, since upon taking $n$th powers
this is equivalent to $n\geq 1^n$, which is true by hypothesis.
Second, suppose there is an $\eps>0$ such that $\sqrt[n]{n}\geq
1+\eps$ for all $n\geq 1$.  Then taking logs of boths sides we see
that $\frac{1}{n}\log(n)\geq \log(1+\eps) > 0$.  But 
 $\log(n)/n\to 0$, so there is no such $\eps$.  Thus
$\sqrt[n]{n}\to 1$ as $n\to \infty$.)
\end{proof}
Note that Axioms (1), (2) and Equation (\ref{val3p}) imply Axiom (3)
with $C=2$.  We take Axiom (3) instead of Equation (\ref{val3p}) for
the technical reason that we will want to call the square of the
absolute value of the complex numbers a valuation.

\begin{lemma}\ilem{$\Bigl||a| - |b|\Bigr| \leq \abs{a-b}$}
Suppose $a, b \in K$, and $\absspc{}$ is a valuation on $K$
with $C\leq 2$.
Then 
$$
  \Bigl||a| - |b|\Bigr| \leq \abs{a-b}.
$$
(Here the big absolute value on the outside of the left-hand
side of the inequality is the usual absolute value on 
real numbers, but the other absolute values are a valuation
on an arbitrary field~$K$.)
\end{lemma}
\begin{proof}
We have 
$$|a| = |b + (a-b)| \leq |b| + |a-b|,$$
so $|a|-|b|\leq \abs{a-b}$.  The same argument
with $a$ and $b$ swapped implies that
$|b|-|a|\leq \abs{a-b}$, which proves the lemma.
\end{proof}


\section{Types of Valuations}
We define two important properties of valuations, both of which
apply to  equivalence classes of valuations (i.e., the property
holds for $\absspc{}$ if and only if it holds for a valuation
equivalent to $\absspc{}$).
\begin{definition}[Discrete]
A valuation $\absspc{}$ is \defn{discrete}\i{valuation!discrete} 
if there is a $\delta>0$
such that for any $a\in K$ 
$$
   1-\delta < \abs{a} < 1+\delta \implies |a|=1.
$$
Thus the absolute values are bounded away from $1$.
%then $|a|=1$.
\end{definition}
To say that $\absspc{}$ is discrete is the same as saying
that the set
$$G=\bigl\{
  \log\abs{a} : a \in K, a\neq 0
\bigr\} \subset \R
$$
forms a discrete subgroup of the reals under addition (because
the elements of the group $G$ are bounded away from $0$).
\begin{proposition}\label{prop:discrete}\iprop{discrete subgroup of $\R$}
A nonzero discrete subgroup $G$ of $\R$ is free on one generator.
\end{proposition}
\begin{proof}
Since $G$ is discrete there is a positive  $m\in G$
such that for any positive $x\in G$ we have $m\leq x$.
Suppose $x\in G$ is an arbitrary positive element.
By subtracting off integer multiples of~$m$, we
find that there is a unique $n$ such that
$$
   0\leq x-nm <m.
$$
Since $x-nm\in G$ and $0<x-nm<m$, it follows
that $x-nm=0$, so $x$ is a multiple of $m$.
\end{proof}
By Proposition~\ref{prop:discrete}, the set 
of $\log\abs{a}$ for nonzero $a\in K$
is free on one generator, so there
is a $c<1$ such that $\abs{a}$, for $a\neq 0$,
runs precisely through the set $$c^\Z = \{c^m : m\in \Z\}$$
(Note: we can replace $c$ by $c^{-1}$ to see that we
can assume that $c<1$).

\begin{definition}[Order]
If $\abs{a} = c^m$, we call $m=\ord(a)$ the \defn{order} 
of $a$. 
\end{definition}
Axiom (2) of valuations
translates into
$$
  \ord(ab) = \ord(a) + \ord(b).
$$

\begin{definition}[Non-archimedean]
A valuation $\absspc{}$ is \defn{non-archimedean} 
if we can take $C=1$ in Axiom (3), i.e., if
\begin{equation}\label{eqn:na}
   |a + b| \leq \max\bigl\{|a|,|b|\bigr\}.
\end{equation}
If $\absspc{}$ is not non-archimedean then
it is \defn{archimedean}.
\end{definition}
Note that if we can take $C=1$ for $\absspc{}$
then we can take $C=1$ for any valuation equivalent to
$\absspc{}$.
To see that (\ref{eqn:na}) is equivalent to Axiom (3) with
$C=1$, suppose $|b|\leq |a|$.  Then $|b/a|\leq 1$, so 
Axiom (3) asserts that $|1+b/a|\leq 1$, which implies
that $|a+b| \leq |a| = \max\{|a|,|b|\}$, and conversely.

We note at once the following consequence:
\begin{lemma}\ilem{$|a+b|=|a|$}
Suppose $\absspc{}$ is a non-archimedean valuation.
If $a,b\in K$ with $|b|<|a|$, then
$
 |a+b|=|a|.
$
\end{lemma}
\begin{proof}
Note that $|a+b|\leq \max\{|a|,|b|\} = |a|$, which
is true even if $|b|=|a|$.  Also,
$$
  |a| = |(a+b) - b| \leq \max\{|a+b|, |b|\} = |a+b|,
$$
where for the last equality we have used that $|b|<|a|$
(if $\max\{|a+b|,|b|\} = |b|$, then $|a|\leq |b|$,
a contradiction).

\end{proof}

\begin{definition}[Ring of Integers]
Suppose $\absspc$ is a non-archimedean absolute
value on a  field $K$.  Then
$$
  \O = \{a\in K : |a|\leq 1\}
$$ is a ring called the \defn{ring of integers} of $K$
with respect to $\absspc{}$.
\end{definition}

\begin{lemma}\ilem{equivalent non-archimedean valuations and $\O$'s}
Two non-archimedean valuations $\absspc{}_1$ and 
$\absspc{}_2$ are equivalent if and only if they
give the same $\O$.
\end{lemma}
We will prove this modulo the claim (to 
be proved later in Section~\ref{sec:topology}) that 
valuations are equivalent if (and only if) they induce the 
same topology.
\begin{proof}
Suppose suppose $\absspc{}_1$ is equivalent to
$\absspc{}_2$, so $\absspc{}_1 = \absspc{}_2^c$,
for some $c>0$.  Then $\abs{c}_1 \leq 1$ if and only if
$\abs{c}_2^c \leq 1$, i.e., if $\abs{c}_2 \leq 1^{1/c}=1$.
Thus $\O_1 = \O_2$.

Conversely, suppose $\O_1 = \O_2$. 
Then $|a|_1<|b|_1$ if and only if $a/b\in \O_1$
and $b/a\not\in \O_1$, so 
\begin{equation}\label{eqn:ineqiff}
  |a|_1<|b|_1 \iff |a|_2 < |b|_2.
\end{equation}
The topology induced by $|\mbox{ }|_1$ has as basis
of open neighborhoods the set of open balls
$$
 B_1(z,r) = \{x \in K : |x-z|_1<r \},
$$
for $r>0$, and likewise for $|\mbox{ }|_2$.  Since 
the absolute values $|b|_1$ get arbitrarily close
to $0$, the set $\mathcal{U}$ of open balls $B_1(z,|b|_1)$ also 
forms a  basis of the topology induced
by $|\mbox{ }|_1$ (and similarly for $|\mbox{ }|_2$).
By (\ref{eqn:ineqiff}) we have
$$
 B_1(z,|b|_1) = B_2(z,|b|_2),
$$
so the two topologies both have $\mathcal{U}$ as
a basis, hence are equal.  That equal topologies
imply equivalence of the corresponding valuations
will be proved in Section~\ref{sec:topology}.
\end{proof}

The set of $a\in \O$ with $|a|<1$ forms an ideal $\p$ in $\O$.  The
ideal $\p$ is maximal, since if $a\in\O$ and $a\not\in\p$ then
$|a|=1$, so $|1/a| = 1/|a| = 1$, hence $1/a\in \O$, so $a$ is a unit.

\begin{lemma}\label{lem:discrete_principal}\ilem{characterization of discrete}
A non-archimedean valuation $\absspc{}$ is
discrete if and only if $\p$ is a principal ideal.
\end{lemma}
\begin{proof}
First suppose that $\absspc{}$ is discrete.
Choose $\pi \in \p$ with $|\pi|$ maximal, which
we can do since 
$$
  S=\{\log|a| : a \in \p\} \subset (-\infty,1],
$$
so the discrete set~$S$ is bounded above.
Suppose $a\in \p$.   Then 
$$
  \abs{\frac{a}{\pi}} = \frac{\abs{a}}{\abs{\pi }} \leq 1,
$$
so $a/\pi\in \O$.
Thus $$a = \pi \cdot \frac{a}{\pi} \in \pi \O.$$

Conversely, suppose $\p=(\pi)$ is principal.  For any $a\in \p$
we have $a=\pi b$ with $b\in\O$.  Thus 
$$
  |a| = |\pi|\cdot |b| \leq |\pi| < 1.
$$
Thus $\{|a| : |a|<1\}$ is bounded away from $1$,
which is exactly the definition of discrete.
\end{proof}

\begin{example}\label{ex:padic_valuation}
For any prime $p$, define the $p$-adic valuation
$\absspc{}_p:\Q\to\R$ as follows.  Write a nonzero $\alpha\in K$
as $p^n\cdot \frac{a}{b}$, where $\gcd(a,p)=\gcd(b,p)=1$.  Then
$$\abs{p^n\cdot \frac{a}{b}}_p := p^{-n} = \left(\frac{1}{p}\right)^{n}.$$
This valuation is both discrete and non-archimedean.
The ring $\O$ is the local ring
$$
  \Z_{(p)} = \left\{\frac{a}{b}\in\Q : p\nmid b\right\},
$$
which has maximal ideal generated by $p$.  Note that
$\ord(p^n\cdot \frac{a}{b}) = n.$
\end{example}

\begin{exercise} \label{ex:valuations1}
Give an example of a non-archimedean valuation on a field that
is not discrete.
\end{exercise} 

We will use the following lemma later (e.g., in
the proof of Corollary~\ref{cor:valna} and Theorem~\ref{thm:ostrowski}).
\begin{lemma}\label{lem:nonarch}\ilem{non-archimedean valuation characterization}
  A valuation $\absspc{}$ is non-archimedean if and only if $|n|\leq
  1$ for all $n$ in the ring generated by $1$ in $K$.
\end{lemma}
Note that we cannot identify the ring generated by $1$ with~$\Z$ 
in general, because~$K$ might have characteristic $p>0$.
\begin{proof}
If $\absspc{}$ is non-archimedean, then $|1|\leq 1$,
so by Axiom (3) with $a=1$, we have  $|1+1|\leq 1$.  By
induction it follows that $|n|\leq 1$.

Conversely, suppose $|n|\leq 1$ for all integer multiples~$n$ of~$1$.
This condition is also true if we replace $\absspc{}$ by
any equivalent valuation, so replace $\absspc{}$ by
one with $C\leq 2$, so that the triangle inequality holds.
Suppose $a\in K$ with $|a|\leq 1$.  Then 
by the triangle inequality,
\begin{align*}
  \abs{1+a}^n &= \abs{(1+a)^n} \\
     \leq& \sum_{j=0}^n \abs{\binom{n}{j}} \abs{a}^j\\
     \leq& 1 + 1 + \cdots + 1 = n+1.
\end{align*}
Now take $n$th roots of both sides to get
$$\abs{1+a} \leq \sqrt[n]{n},$$
and take the limit as $n\to \infty$ to see
that $\abs{1+a} \leq 1$.  This proves that one
can take $C=1$ in Axiom (3), hence that $\absspc{}$
is non-archimedean.
\end{proof}

\section{Examples of Valuations}
The archetypal example of an archimedean valuation is the absolute
value on the complex numbers.  It is essentially the only one:
\begin{theorem}[Gelfand-Tornheim]\ithm{Gelfand-Tornheim} Any field~$K$ with
an archimedean valuation is isomorphic to a subfield of~$\C$,
the valuation being equivalent to that induced by the usual
absolute value on~$\C$.
\end{theorem}
We do not prove this here as we do not need it.  For a proof,
see \cite[pg. 45, 67]{artin:ant}.

There are many non-archimedean valuations.  On the rationals $\Q$
there is one for every prime $p>0$, the $p$-adic valuation, as
in Example~\ref{ex:padic_valuation}.

\begin{theorem}[Ostrowski]\label{thm:ostrowski}\ithm{Ostrowski}
\ithm{valuations on $\Q$}
The nontrivial valuations
on $\Q$ are those equivalent to $\absspc_p$, for some
prime $p$, and the usual absolute value $\absspc_\infty$.
\end{theorem}
\begin{remark}
Before giving the proof, we pause with a brief remark about
Ostrowski.  According to
\begin{verbatim}
   http://www-gap.dcs.st-and.ac.uk/~history/Mathematicians/Ostrowski.html
\end{verbatim}
\noindent{}Ostrowski was a Ukrainian mathematician who lived
1893--1986.  Gautschi writes about Ostrowski as follows: ``... you are
able, on the one hand, to emphasise the abstract and axiomatic side of
mathematics, as for example in your theory of general norms, or, on
the other hand, to concentrate on the concrete and constructive
aspects of mathematics, as in your study of numerical methods, and to
do both with equal ease. {\em You delight in finding short and
succinct proofs, of which you have given many examples} ...'' [italics mine]
\end{remark}
We will now give an example of one of these short and succinct proofs.
\begin{proof}                                     
Suppose $\absspc$ is a nontrivial valuation on $\Q$.
\par{\em Nonarchimedean case:} 
Suppose $\abs{c}\leq 1$ for all $c\in\Z$, so by
Lemma~\ref{lem:nonarch}, $\absspc$ is nonarchimedean.  
Since $\absspc$ is nontrivial, the set 
$$
  \p=\{a\in\Z : \abs{a}<1\}
$$
is nonzero.  Also $\p$ is an ideal and if $\abs{ab}<1$,
then $\abs{a}\abs{b}=\abs{ab}<1$, so $\abs{a}<1$ or $\abs{b}<1$,
so $\p$ is a prime ideal of~$\Z$.  Thus $\p=p\Z$, for some prime
number~$p$.  Since every element of $\Z$ has valuation at most
$1$, if  $u\in\Z$ with $\gcd(u,p)=1$, then $u\not\in\p$,
so $\abs{u}=1$.  Let $\alpha=\log_{\abs{p}}\frac{1}{p}$, so 
$\abs{p}^\alpha = \frac{1}{p}$.    Then for any $r$ and any $u\in\Z$
with $\gcd(u,p)=1$, we have
$$
\abs{up^r}^{\alpha} = \abs{u}^{\alpha}\abs{p}^{\alpha r}
   = \abs{p}^{\alpha r} = p^{-r} = \abs{up^r}_p.
$$
Thus $\absspc^{\alpha} =  \absspc_p$ on $\Z$, hence on $\Q$
by multiplicativity, so $\absspc$ is equivalent to $\absspc_p$,
as claimed.

{\em Archimedean case:} By replacing $\absspc$ by a power of
$\absspc$, we may assume without loss that $\absspc$ satisfies the
triangle inequality.  We first make some general remarks about any
valuation that satisfies the triangle inequality.  
Suppose $a\in\Z$ is greater than $1$.  Consider, for any $b\in\Z$
the base-$a$ expansion of $b$:
$$
  b = b_m a^m + b_{m-1} a^{m-1} + \cdots + b_0,
$$
where 
$$
  0 \leq b_j < a \qquad (0\leq j \leq m),
$$  
and $b_m\neq 0$.
Since $a^m\leq b$, taking logs we see that
$m\log(a)\leq \log(b)$, so 
$$m \leq \frac{\log(b)}{\log(a)}.$$
Let $\ds M=\max_{1\leq d<a}\abs{d}$.  Then by the triangle
inequality for $\absspc$, we have
\begin{align*}
\abs{b}&\leq \abs{b_m}\abs{a}^m + \cdots + \abs{b_1}\abs{a} + \abs{b_0}\\
    & \leq M\cdot (\abs{a}^m + \cdots + \abs{a} + 1)\\
   & \leq M\cdot (m+1)\cdot \max(1,\abs{a}^m)\\
   & \leq M\cdot\left(\frac{\log(b)}{\log(a)} + 1\right)
        \cdot \max\left(1,\abs{a}^{\log(b)/\log(a)}\right),
\end{align*}
where in the last step we use that $m\leq \frac{\log(b)}{\log(a)}$.
Setting $b=c^n$, for $c\in\Z$, in the above inequality and
taking $n$th roots, we have 
\begin{align*}
\abs{c} &\leq \left( M\cdot 
\left(\frac{\log(c^n)}{\log(a)}+1\right)\cdot
  \max(1,\abs{a}^{log(c^n)/\log(a)})\right)^{1/n}\\
  & = M^{1/n}\cdot\left(
       \frac{\log(c^n)}{\log(a)}+1\right)^{1/n}\cdot
      \max\left(1,\abs{a}^{\log(c^n)/\log(a)}\right)^{1/n}.
\end{align*}
The first factor $M^{1/n}$ converges to~$1$ as $n\to\infty$,
since $M\geq 1$ (because $\abs{1}=1$).  The second factor
is
$$
  \left(\frac{\log(c^n)}{\log(a)}+1\right)^{1/n}
   = 
\left(n \cdot \frac{\log(c)}{\log(a)}+1\right)^{1/n}
$$
which also converges to $1$, for the
same reason that $n^{1/n}\to 1$ 
(because $\log(n^{1/n})=\frac{1}{n}\log(n)\to 0$ as
$n\to\infty$).
The third factor is
$$
\max\left(1,\abs{a}^{\log(c^n)/\log(a)}\right)^{1/n}
  = \begin{cases}
   1 & \text{if }\abs{a}<1,\\
\abs{a}^{\log(c)/\log(a)} & \text{if }\abs{a}\geq 1.
\end{cases}
$$
Putting this all together, we see that
$$
\abs{c} \leq \max\left(1,\abs{a}^{\frac{\log(c)}{\log(a)}}\right).
$$

Our assumption that $\absspc$ is archimedean implies
that there is $c\in\Z$ with $c>1$ and $\abs{c}>1$.
Then for all $a\in\Z$ with $a>1$ we have
\begin{equation}\label{eqn:pow}
  1 < \abs{c} \leq
  \max\left(1,\abs{a}^{\frac{\log(c)}{\log(a)}}\right),
\end{equation}
so $1<\abs{a}^{\log(c)/\log(a)}$, so 
$1<\abs{a}$ as well (i.e., any $a\in\Z$ with
$a>1$ automatically satisfies $\abs{a}>1$).  Also, taking the
$1/\log(c)$ power on both sides of (\ref{eqn:pow})
we see that 
\begin{equation}\label{eqn:ineqac}
  \abs{c}^{\frac{1}{\log(c)}}
    \leq   \abs{a}^{\frac{1}{\log(a)}}.
\end{equation}
Because, as mentioned above, $\abs{a}>1$, we can interchange the roll
of $a$ and $c$ to obtain the reverse inequality of (\ref{eqn:ineqac}).
We thus have 
$$
  \abs{c}
    =   \abs{a}^{\frac{\log(c)}{\log(a)}}.
$$ Letting $\alpha=\log(2)\cdot \log_{\abs{2}}(e)$ and setting $a=2$,
we have
$$
  \abs{c}^{\alpha} = \abs{2}^{\frac{\alpha}{\log(2)}\cdot \log(c)}
      = \left(\abs{2}^{\log_{\abs{2}}(e)}\right)^{\log(c)} =
   e^{\log(c)} = c = \abs{c}_\infty.
$$
Thus for all integers $c\in\Z$ with $c>1$ we have
$\abs{c}^{\alpha} = \abs{c}_{\infty}$, which implies
that $\absspc$ is equivalent to $\absspc_\infty$.
\end{proof}

Let $k$ be any field and let $K=k(t)$, where $t$
is transcendental.  Fix a real number $c>1$.
If $p=p(t)$ is an irreducible
polynomial in the ring $k[t]$, we define a valuation
by 
\begin{equation}\label{eqn:ffabsp}
  \abs{p^a \cdot \frac{u}{v}}_p = c^{-\deg(p)\cdot a},
\end{equation}
where $a\in\Z$ and $u,v\in k[t]$ with
$p\nmid u$ and $p\nmid v$.
\begin{remark}
This definition differs from the one page 46 of [Cassels-Frohlich,
Ch. 2] in two ways.   First, we assume that $c>1$ instead
of $c<1$, since otherwise $\absspc_p$ does not satisfy
Axiom 3 of a valuation.  Also, we write $c^{-\deg(p)\cdot a}$
instead of $c^{-a}$, so that the product formula will
hold.  (For more about the product formula, see
Section~\ref{sec:global_fields}.)
\end{remark}
In addition there is a a non-archimedean valuation
$\absspc_\infty$ defined by 
\begin{equation}\label{eqn:ffabsoo}
  \abs{\frac{u}{v}}_\infty = c^{\deg(u)-\deg(v)}.
\end{equation}


This definition differs from the one in \cite[pg.~46]{cassels:global}
in two ways.  First, we assume that $c>1$ instead of $c<1$, since
otherwise $\absspc_p$ does not satisfy Axiom 3 of a valuation.  Here's
why: Recall that Axiom 3 for a non-archimedean valuation on $K$
asserts that whenever $a\in K$ and $\abs{a}\leq 1$, then
$\abs{a+1}\leq 1$.  Set $a=p-1$, where $p=p(t)\in K[t]$ is an
irreducible polynomial.  Then $\abs{a}=c^0 = 1$, since $\ord_p(p-1) =
0$.  However, $\abs{a+1} = \abs{p-1+1} = \abs{p}=c^{-1}>1$, since
$\ord_p(p) = 1$.  If we take $c>1$ instead of $c<1$, as I propose,
then $\abs{p}=c^{-1}<1$, as required.


Note the (albeit imperfect) analogy between $K=k(t)$ and $\Q$.
If $s=t^{-1}$, so $k(t)=k(s)$, the valuation $\absspc_{\infty}$
is of the type (\ref{eqn:ffabsp}) belonging to the irreducible
polynomial $p(s)=s$.

The reader is urged to prove the following lemma as a homework
problem.
\begin{lemma}\ilem{valuations on $k(t)$}
The only nontrivial valuations on $k(t)$ which are trivial
on $k$ are equivalent to the valuation (\ref{eqn:ffabsp})
or (\ref{eqn:ffabsoo}).
\end{lemma}
For example, if $k$ is a finite field, there are no
nontrivial valuations on $k$, so the only
nontrivial valuations on $k(t)$ are equivalent to 
(\ref{eqn:ffabsp}) or (\ref{eqn:ffabsoo}).

\begin{exercise} \label{ex:valuations2}
Let $k$ be any field. Prove that the only nontrivial valuations
on $k(t)$ which are trivial on $k$ are equivalent to the valuation
(\ref{eqn:ffabsp}) or (\ref{eqn:ffabsoo}) of page~\pageref{eqn:ffabsp}.
\end{exercise} 


%%% Local Variables: 
%%% mode: latex
%%% TeX-master: "ant"
%%% End: 
