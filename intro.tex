\chapter{Introduction}

\section{Mathematical background}
In addition to general mathematical maturity, 
this book assumes you have the following background:
\begin{itemize}\setlength{\itemsep}{-.7ex}
\item Basics of finite group theory
\item Commutative rings, ideals, quotient rings
\item Some elementary number theory
\item Basic Galois theory of fields
\item Point set topology
\item Basics of topological rings, groups, and measure theory
\end{itemize}
For example, if you have never worked with finite groups before, you
should read another book first. If you haven't seen much elementary
ring theory, there is still hope, but you will have to do some
additional reading and exercises.  We will briefly review the basics of
the Galois theory of number fields.

Some of the homework problems involve using a computer, but there
are examples which you can build on.  We will not assume that you have
a programming background or know much about algorithms. Most
of the book uses Sage (\url{http://sagemath.org}), which is
free open source mathematical software.  The following is an example
Sage session:
\begin{sagecell}
2 + 2
\end{sagecell}
\begin{sageout}
4
\end{sageout}
\begin{sagecell} 
k.<a> = NumberField(x^2 + 1); k
\end{sagecell}
\begin{sageout}
Number Field in a with defining polynomial x^2 + 1
\end{sageout}

\section{What is algebraic number theory?}
A number field $K$ is a finite degree algebraic extension of the rational
numbers $\Q$.  The primitive element theorem from Galois theory
asserts that every such extension can be represented as the set of all
polynomials of degree at most $d = [K:\Q] = \dim_{\Q} K$ in 
a single root $\alpha$ of some polynomial with coefficients in $\Q$:
$$
 K = \Q(\alpha) = \left\{ \sum_{n=0}^{m} a_n \alpha^n : a_n\in\Q \right\}.
$$ 

Note that
$\Q(\alpha)$ is non-canonically isomorphic to $\Q[x]/(f)$, where $f$
is the minimal polynomial of~$\alpha$.
The homomorphism $\Q[x]\to\Q(\alpha)$ that sends~$x$ to~$\alpha$,
which has kernel~$(f)$, hence it descends to an isomorphism between
$\Q[x]/(f)$ and $\Q(\alpha)$.
It is not canonical, since $\Q(\alpha)$ could
have nontrivial automorphisms.  For example, if $\alpha=\sqrt{2}$, then
$\Q(\sqrt{2})$ is isomorphic as a field to $\Q(-\sqrt{2})$ via
$\sqrt{2}\mapsto -\sqrt{2}$.  There are two isomorphisms
$\Q[x]/(x^2-2)\to \Q(\sqrt{2})$.

\defn{Algebraic number theory} involves using techniques from (mostly
commutative) algebra and finite group theory to gain a deeper
understanding of the arithmetic of number fields and related objects
(e.g., functions fields, elliptic curves, etc.).  The main objects that we
study in this book are number fields, rings of integers
of number fields, unit groups, ideal class groups, norms, traces,
discriminants, prime ideals, Hilbert and other class fields and
associated reciprocity laws, zeta and $L$-functions, and algorithms
for computing with each of the above.

\subsection{Topics in this book}
These are some of the main topics that are discussed in this book:
\begin{itemize}\setlength{\itemsep}{-.7ex}
\item Rings of integers of number fields
\item Unique factorization of nonzero ideals in Dedekind domains
\item Structure of the group of units of the ring of integers
\item Finiteness of the abelian group of equivalence classes
of nonzero ideals of the ring of integers (the ``class group'')
\item Decomposition and inertia groups, Frobenius elements
\item Ramification
\item Discriminant and different
\item Quadratic and biquadratic fields
\item Cyclotomic fields (and applications)
\item How to use a computer to compute with many of the above objects
\item Valuations on fields
\item Completions ($p$-adic fields)
\item Adeles and Ideles
\end{itemize}
Note that we will not do anything nontrivial with zeta functions or
$L$-functions. \edit{This is for another course -- add this to the
book later.}



\section{Some applications of algebraic number theory}
The following examples illustrate some of the power, depth and
importance of algebraic number theory.

\begin{enumerate}
\item {\bf Integer factorization} using the number field sieve.  The
number field sieve is the asymptotically fastest known algorithm for
factoring general large integers (that don't have too special of a
form).  On December 12, 2009, the number field sieve was used to
factor the RSA-768 challenge, which is a 232 digit number that is a product of two primes:\\
\begin{sagecell}
rsa768 = 123018668453011775513049495838496272077285356959533479\
219732245215172640050726365751874520219978646938995647494277406384592\
519255732630345373154826850791702612214291346167042921431160222124047\
9274737794080665351419597459856902143413
n = 33478071698956898786044169848212690817704794983713768568912\ %purple!
431388982883793878002287614711652531743087737814467999489
m = 36746043666799590428244633799627952632279158164343087642676\
032283815739666511279233373417143396810270092798736308917
n*m == rsa768
\end{sagecell}
\begin{sageout}
True
\end{sageout}

This record integer factorization cracked a certain 768-bit public key
cryptosystem (see \url{http://eprint.iacr.org/2010/006}), thus
establishing a lower bound on one's choice of key size:
\begin{sagecell}[language=bash]
$ man ssh-keygen   # in ubuntu-12.04
...
     -b bits
             Specifies the number of bits in the key to create.  
             For RSA keys, the minimum size is 768 bits ...
\end{sagecell}


\item {\bf Primality testing:} Agrawal and his students Saxena and Kayal from
India found in 2002 the first ever deterministic
polynomial-time (in the number of digits) primality test.  Their
methods involve arithmetic in quotients of $(\Z/n\Z)[x]$, which are
best understood in the context of algebraic number theory.  

\item {\bf Deeper point of view} on questions in number theory:
\begin{enumerate}
\item Pell's Equation $x^2-dy^2=1$ can be reinterpreted in terms of units in real quadratic fields, which
leads to a study of unit groups of number fields.
\item Integer factorization leads to factorization of nonzero ideals in rings of integers of number fields.
\item The Riemann hypothesis about the zeros of $\zeta(s)$ generalizes to zeta functions of number fields.
\item Reinterpreting Gauss's quadratic reciprocity law in terms of the
  arithmetic of cyclotomic fields $\Q(e^{2\pi i/n})$ leads to class
  field theory, which in turn leads to the Langlands program.
\end{enumerate}
\item Wiles's proof of {\bf Fermat's Last Theorem}, i.e., that the
equation $x^n+y^n=z^n$ has no solutions with $x,y,z,n$ all positive
integers and $n\geq 3$, uses methods from
algebraic number theory extensively, in addition to many other deep
techniques.  Attempts to prove Fermat's Last Theorem long ago were
hugely influential in the development of algebraic number theory
by Dedekind, Hilbert, Kummer, Kronecker, and others.
\item {\bf Arithmetic geometry:} This is a huge field that studies
solutions to polynomial equations that lie in arithmetically
interesting rings, such as the integers or number fields.  A famous
major triumph of arithmetic geometry is Faltings's proof of Mordell's
Conjecture.
\begin{theorem}[Faltings] \label{thm:faltings}\ithm{Faltings}
Let $X$ be a nonsingular plane algebraic curve over a number
field $K$.  Assume that the manifold $X(\C)$ of complex solutions to
$X$ has genus at least $2$ (i.e., $X(\C)$ is topologically a donut
with at least two holes).  Then the set $X(K)$ of points on $X$ with
coordinates in~$K$ is finite.
\end{theorem}  
For example, Theorem~\ref{thm:faltings} implies that for any $n\geq 4$
and any number field~$K$, there are only finitely many solutions
in~$K$ to $x^n+y^n=1$.  

A major open problem in arithmetic geometry is the {\em Birch
  and Swinnerton-Dyer conjecture}. 
An \defn{elliptic curves} $E$ is an algebraic curve with at least one point
with coordinates in $K$ such that the set of complex points
$E(\C)$ is a topological torus.
The Birch and Swinnerton-Dyer conjecture gives a
criterion for whether or not $E(K)$ is infinite in
terms of analytic properties of the $L$-function $L(E,s)$.
See \url{http://www.claymath.org/millennium/Birch_and_Swinnerton-Dyer_Conjecture/}.

\end{enumerate}


%%% Local Variables: 
%%% mode: latex
%%% TeX-master: "ant"
%%% End: 
