\chapter{Ideles and Ideals}
In this chapter, we introduce the ideles $\II_K$, and relate ideles to
ideals, and use what we've done so far to give an alternative
interpretation of class groups and their finiteness, thus linking the
adelic point of view with the classical point of view of the first
part of this course.



\section{The Idele Group}
The invertible elements of any commutative
topological ring~$R$ are a group $R^*$ under multiplication.
In general $R^*$ is not a topological group if it is
endowed with the subset topology because inversion need
not be continuous (only multiplication and addition on 
$R$ are required to be continuous).  It is usual therefore
to give $R^*$ the following topology.
There is an injection 
\begin{equation}\label{eqn:prod_embed}
  x\mapsto \left( x, \,\,\,\frac{1}{x} \right)
\end{equation}
of $R^*$ into the topological product $R\times R$.  We give $R^*$ the
corresponding subset topology.  Then $R^*$ with this topology is a
topological group and the inclusion map $R^*\hra R$ is continous.  To
see continuity of inclusion, note that this topology is finer (has at
least as many open sets) than the subset topology induced by
$R^*\subset R$, since the projection maps $R\times R\to R$ are
continuous.

\begin{example}\label{ex:cont}
This is a ``non-example''. The inverse map on $\Z_p^*$ is continuous with
respect to the $p$-adic topology.  If $a,b\in \Z_p^*$,
then $\abs{a}=\abs{b}=1$, so if $\abs{a-b}<\eps$, then
$$
  \abs{\frac{1}{a} - \frac{1}{b}}
   = \abs{\frac{b-a}{ab}} = \frac{\abs{b-a}}{\abs{ab}} < \frac{\eps}{1}=\eps.
$$
\end{example}

\begin{definition}[Idele Group]
  The \defn{idele group} $\II_K$ of $K$ is the group $\AA_K^*$ of invertible
  elements of the adele ring $\AA_K$.
\end{definition}
We shall usually speak of $\II_K$ as a subset of $\AA_K$, and will
have to distinguish between the $\II_K$ and $\AA_K$-topologies.
\begin{example}
For a rational prime $p$, let $\bx_p\in \AA_\Q$ be the adele whose $p$th
component is $p$ and whose $v$th component, for $v\neq p$, is $1$.
Then $\bx_p \to 1$ as $p\to\infty$ in $\AA_\Q$, for the following reason.
We must show that if $U$ is a basic open set that contains the
adele $1=\{1\}_v$, the $\bx_p$ for all sufficiently large $p$
are contained in $U$.  Since $U$ contains $1$ and is a basic
open set, it is of the form 
$$\prod_{v\in S} U_v \times \prod_{v\not\in S} \Z_v,$$
where $S$ is a finite set, and the $U_v$, for $v\in S$, are
arbitrary open subsets of $\Q_v$ that contain $1$.  
If $q$ is a prime larger than any prime in $S$, then 
$\bx_p$ for $p\geq q$, is in $U$.   This proves
convergence.
If the inverse map were continuous on $\II_K$, then 
the sequence of $\bx_p^{-1}$ would converge to $1^{-1}=1$.  
However, if $U$ is an open set as above about $1$, then
for sufficiently large $p$, {\em none} of the adeles $\bx_p$ are
contained in $U$.
\end{example}


\begin{lemma}\label{lem:idelesprod}\ilem{ideles are a restricted product}
The group of ideles $\II_K$ is the restricted topological project
of the $K_v^*$ with respect to the units $U_v=\O_v^*\subset K_v$,
with the restricted product topology.
\end{lemma}
We omit the proof of Lemma~\ref{lem:idelesprod}, which is a
matter of thinking carefully about the definitions.  The main
point is that inversion is continuous on $\O_v^*$ for each~$v$.
(See Example~\ref{ex:cont}.)


We have seen that $K$ is naturally embedded in $\AA_K$, so
$K^*$ is naturally embedded in~$\II_K$.  
\begin{definition}[Principal Ideles]
  We call $K^*$, considered as a subgroup of $\II_K$, the 
\defn{principal ideles}.
\end{definition}

\begin{lemma}\ilem{principal ideles are discrete}
The principal ideles $K^*$ are discrete as a subgroup of $\II_K$.
\end{lemma}
\begin{proof}
  For $K$ is discrete in $\AA_K$, so $K^*$ is embedded in $\AA_K\times
  \AA_K$ by (\ref{eqn:prod_embed}) as a discrete subset.
  (Alternatively, the subgroup topology on $\II_K$ is finer than the
  topology coming from $\II_K$ being a subset of $\AA_K$, and $K$ is
  already discrete in $\AA_K$.)
\end{proof}

\begin{definition}[Content of an Idele]
The \defn{content} of $\bx=\{x_v\}_v \in \II_K$ is
$$
  c(\bx) = \prod_{\text{all }v} \abs{x_v}_v \in \R_{>0}.
$$
\end{definition}

\begin{lemma}\ilem{content map is continuous}
The map $\bx\to c(\bx)$ is a continuous  homomorphism of
the topological group $\II_K$ into $\R_{>0}$, where
we view $\R_{>0}$ as a topological group under multiplication.
If~$K$ is a number field, then $c$ is surjective.
\end{lemma}
\begin{proof}
That the content map~$c$ satisfies the axioms of a homomorphisms
follows from the multiplicative nature of the defining formula
for~$c$.  For continuity, suppose $(a,b)$ is an open interval
in $\R_{>0}$.  Suppose $\bx\in\II_K$ is such that $c(\bx)\in (a,b)$.
By considering small intervals about each non-unit component of 
$\bx$, we find an open neighborhood $U\subset \II_K$ of $\bx$ 
such that $c(U)\subset (a,b)$.  It follows the $c^{-1}((a,b))$
is open.

For surjectivity, use that each archimedean valuation is surjective,
and choose an idele that is~$1$ at all but one archimedean valuation.
\end{proof}
\begin{remark}
Note also that the $\II_K$-topology is that appropriate to a
group of operators on $\AA_K^+$: a basis of open sets
is the $S(C,U)$, where $C,U\subset \AA_K^+$
are, respectively, $\AA_K$-compact and $\AA_K$-open, and
$S$ consists of the $\bx\in \II_J$ such that 
$(1-\bx) C\subset U$ and 
$(1-\bx^{-1}) C\subset U$.
\end{remark}

\begin{definition}[$1$-Ideles]
  The subgroup $\II_K^1$ of \defn{$1$-ideles} is the subgroup of ideles
  $\bx=\{x_v\}$ such that $c(\bx)=1$.  Thus $\II_K^1$
is the kernel of $c$, so we have an exact sequence
$$
1 \to \II_K^1 \to \II_K \xra{c} \R_{>0}\to 1,
$$
where the surjectivity on the right is only if $K$
is a number field.
\end{definition}

\begin{lemma}\ilem{subset topology on $1$-ideles $\II_K^1$}
The subset $\II_K^1$ of $\AA_K$ is closed as a subset,
and the $\AA_K$-subset topology on $\II_K^1$ coincides
with the $\II_K$-subset topology on $\II_K^1$.
\end{lemma}
\begin{proof}
  Let $\bx\in\AA_K$ with $\bx\not\in\II_K^1$. To prove that $\II_K^1$
  is closed in $\AA_K$, we find an $\AA_K$-neighborhood $W$ of $\bx$
  that does not meet $\II_K^1$.

{\em 1st Case.}  Suppose that $\prod_v \abs{x_v}_v<1$ (possibly $=0$).
Then there is a finite set~$S$ of~$v$ such that
\begin{enumerate}
\item $S$ contains all the $v$ with $\abs{x_v}_v>1$, and
\item $\prod_{v\in S}\abs{x_v}_v < 1$.
\end{enumerate}
Then the set $W$ can be
defined by 
\begin{align*}
  \abs{w_v - x_v}_v &< \eps \qquad v\in S\\
  \abs{w_v}_v &\leq 1 \qquad v\not\in S
\end{align*}
for sufficiently small $\eps$.

{\em 2nd Case.} Suppose that $C:=\prod_v \abs{x_v}_v>1$.
Then there is a finite set $S$ of $v$ such that
\begin{enumerate}
\item $S$ contains all the $v$ with $\abs{x_v}_v>1$, and
\item if $v\not\in S$ an inequality 
$\abs{w_v}_v<1$ implies $\abs{w_v}_v < \frac{1}{2C}.$
(This is because for a non-archimedean valuation, the
largest absolute value less than $1$ is $1/p$, where $p$ is
the residue characteristic.  Also, the upper bound in
Cassels's article is $\frac{1}{2}C$ instead of $\frac{1}{2C}$,
but I think he got it wrong.)
\end{enumerate}
We can choose $\eps$ so small that 
$\abs{w_v-x_v}_v<\eps$ (for $v\in S$) implies
$1<\prod_{v\in S} \abs{w_v}_v < 2C.$  Then $W$ may be defined
by 
\begin{align*}
  \abs{w_v - x_v}_v &< \eps \qquad v\in S\\
  \abs{w_v}_v &\leq 1 \qquad v\not\in S.
\end{align*}
This works because if $\bw\in W$, then either
$\abs{w_v}_v=1$ for all $v\not\in S$, in 
which case $1 < c(\bw)  < 2c$, so $\bw\not\in \II_K^1$,
or $\abs{w_{v_0}}_{v_0}<1$ for some $v_0\not\in S$, in
which case $$c(\bw) = \left(\prod_{v\in S} \abs{w_v}_v \right)\cdot 
\abs{w_{v_0}} \cdots < 2C \cdot \frac{1}{2C} \cdots < 1,$$
so again $\bw\not\in\II_K^1$.

We next show that the $\II_K$- and $\AA_K$-topologies on $\II_K^1$
are the same.  If $\bx\in \II_K^1$, we must show that every
$\AA_K$-neighborhood of $\bx$ contains an $\II_K$-neighborhood
and vice-versa.

Let $W\subset \II_K^1$ be an $\AA_K$-neighborhood of $\bx$.  Then it
contains an $\AA_K$-neighborhood of the type
\begin{align}\label{eqn:adelicnbhd}
  \abs{w_v - x_v}_v &< \eps \qquad v\in S\\
  \abs{w_v}_v &\leq 1 \qquad v\not\in S
\end{align}
where $S$ is a finite set of valuations $v$.  This contains
the $\II_K$-neighborhood in which $\leq$ in (\ref{eqn:adelicnbhd})
is replaced by $=$.

Next let $H\subset \II_K^1$ be an $\II_K$-neighborhood.  Then it contains
an $\II_K$-neighborhood of the form
\begin{align}\label{eqn:adelicnbhd2}
  \abs{w_v - x_v}_v &< \eps \qquad v\in S\\
  \abs{w_v}_v &= 1 \qquad v\not\in S,
\end{align}
where the finite set~$S$ contains at least all archimedean
valuations~$v$ and all valuations~$v$ with 
$\abs{x_v}_v\neq 1$.  Since $\prod\abs{x_v}_v=1$, we may also
suppose that $\eps$ is so small that (\ref{eqn:adelicnbhd2})
implies 
$$
  \prod_v\abs{w_v}_v < 2.
$$
Then the intersection of  (\ref{eqn:adelicnbhd2}) with
$\II_K^1$ is the same as that of  (\ref{eqn:adelicnbhd})
with $\II_K^1$, i.e.,  (\ref{eqn:adelicnbhd2})
defines an $\AA_K$-neighborhood.
\end{proof}

By the product formula we have that $K^*\subset \II_K^1$.
The following result is of vital importance in class
field theory.
\begin{theorem}\label{thm:compquo}\ithm{compact quotient of ideles}
The quotient $\II_K^1/K^*$ with the quotient topology
is compact.
\end{theorem}
\begin{proof}
After the preceeding lemma, it is enough to find
an $\AA_K$-compact set $W\subset \AA_K$ such that the map
$$
  W\meet \II_K^1 \to \II_K^1/K^*
$$
is surjective.  We take for $W$ the set of 
$\bw=\{w_v\}_v$ with
$$
  \abs{w_v}_v\leq \abs{x_v}_v,
$$
where $\bx=\{x_v\}_v$ is any idele of content greater than
the $C$ of Lemma~\ref{lem:bignorm}.

Let $\by=\{y_v\}_v\in \II_K^1$.  Then the content of $\bx/\by$ equals
the content of $\bx$, so by Lemma~\ref{lem:bignorm}
there is an $a\in K^*$ such that 
$$
  \abs{a}_v \leq \abs{\frac{x_v}{y_v}}_v \qquad\text{all $v$}.
$$
Then $a\by\in W$, as required.
\end{proof}

\begin{remark}
  The quotient $\II_K^1/K^*$ is totally disconnected in the function
  field case.  For the structure of its connected component in the
  number field case, see papers of Artin and Weil in the ``Proceedings
  of the Tokyo Symposium on Algebraic Number Theory, 1955'' (Science
  Council of Japan) or \cite{artin-tate:cft}. The determination of the
  character group of $\II_K/K^*$ is global class field theory.
\end{remark}

\section{Ideals and Divisors}
Suppose that $K$ is a finite extension of $\Q$.  Let $F_K$ be the the
free abelian group on a set of symbols in bijection with the
non-archimedean valuation~$v$ of $K$.  Thus an element of $F_K$
is a formal linear combination 
$$
  \sum_{v\text{ non arch.}} n_v \cdot v
$$
where $n_v\in\Z$ and all but finitely many $n_v$ are $0$. 

\begin{lemma}\ilem{fractional ideals and formal sums of valuations}
  There is a natural bijection between $F_K$ and the group of nonzero
  fractional ideals of $\O_K$.  The correspondence is induced by
  $$ v\mapsto \wp_v = \{x \in \O_K : v(x)<1\},$$
where $v$ is a non-archimedean valuation.
\end{lemma}


Endow $F_K$ with the discrete topology.  Then there is a natural
continuous map $\pi:\II_K \to F_K$ given by
$$
\bx = \{x_v\}_v \mapsto \sum_v \ord_v(x_v)\cdot v.
$$
This map is continuous since the inverse image of 
a valuation $v$ (a point) is the product 
$$
\pi^{-1}(v) = \pi \O_v^* \quad \times 
\prod_{w \text{ archimedean}} K_w^*\quad
\times \prod_{w\neq v \text{ non-arch.}} \O_w^*,
$$
which is an open set in the restricted product
topology on $\II_K$.
Moreover, the image of $K^*$ in $F_K$ is the group of nonzero
principal fractional ideals.

Recall that the \defn{class group} $C_K$ of the number field $K$
is by definition the quotient of $F_K$ by the image of $K^*$.

\begin{theorem}\label{thm:classgrpfin}\ithm{finiteness of class group}
The class group $C_K$ of a number field $K$ is finite.
\end{theorem}
\begin{proof}
  We first prove that the map $\II_K^1\to F_K$ is surjective.  Let
  $\infty$ be an archimedean valuation on $K$.  If $v$ is a
  non-archimedean valuation, let $\bx\in \II_K^1$ be a $1$-idele such
  that $x_w=1$ at ever valuation~$w$ except~$v$ and~$\infty$.  At~$v$,
  choose $x_v = \pi$ to be a generator for the maximal ideal of
  $\O_v$, and choose $x_\infty$ to be such that $\abs{x_\infty}_\infty
  = 1/\abs{x_v}_v$.  Then $\bx\in \II_K$ and $\prod_{w}\abs{x_w}_w =
  1$, so $\bx\in\II_K^1$.  Also $\bx$ maps to $v \in F_K$.
  
  Thus the group of ideal classes is the continuous image of the
  compact group $\II_K^1/K^*$ (see Theorem~\ref{thm:compquo}), hence
  compact.  But a compact discrete group is finite.
\end{proof}

\subsection{The Function Field Case}

When $K$ is a finite separable extension of $\F(t)$, we define the
divisor group $D_K$ of $K$ to be the free abelian group on all the
valuations~$v$.  For each $v$ the number of elements of the residue class
field $\F_v = \O_v/\wp_v$ of $v$ is a power, say $q^{n_v}$, of the number~$q$
of elements in $\F$.  We call $n_v$ the degree of $v$, and similarly
define $\sum n_v d_v$ to be the degree of the divisor $\sum n_v\cdot v$.
The divisors of degree $0$ form a group $D_K^0$.  
As before, the principal divisor attached to $a\in K^*$  is 
$\sum \ord_v(a) \cdot v \in D_K$.
The following theorem is proved in the same way as Theorem~\ref{thm:classgrpfin}.
\begin{theorem}\label{thm:finclassgrpff}\ithm{finiteness of function field class group}
The quotient of $D_K^0$ modulo the principal divisors is
a finite group.
\end{theorem}

\subsection{Jacobians of Curves}
For those familiar with algebraic geometry and algebraic curves, one
can prove Theorem~\ref{thm:finclassgrpff} from an alternative point of
view.  There is a bijection between nonsingular geometrically
irreducible projective curves over $\F$ and function fields $K$ over
$\F$ (which we assume are finite separable extensions of $\F(t)$ such
that $\Fbar\meet K = \F$).  Let $X$ be the curve corresponding to $K$.
The group $D_K^0$ is in bijection with the divisors of degree $0$ on
$X$, a group typically denoted $\Div^0(X)$.  The quotient of
$\Div^0(X)$ by principal divisors is denoted $\Pic^0(X)$.  The {\em
  Jacobian} of $X$ is an abelian variety $J=\Jac(X)$ over the finite
field $\F$ whose dimension is equal to the genus of $X$.  Moreover,
assuming $X$ has an $\F$-rational point, the elements of $\Pic^0(X)$
are in natural bijection with the $\F$-rational points on~$J$.  In
particular, with these hypothesis, the class group of $K$, which is
isomorphic to $\Pic^0(X)$, is in bijection with the group of
$\F$-rational points on an algebraic variety over a finite field.
This gives an alternative more complicated proof of finiteness of the
degree $0$ class group of a function field.  

Without the degree $0$ condition, the divisor class group won't be finite.  It
is an extension of~$\Z$ by a finite group.
$$
  0 \to \Pic^0(X) \to \Pic(X) \xra{\deg} n\Z \to 0,
$$
where~$n$ is the greatest common divisor of the degrees of 
elements of $\Pic(X)$, which is $1$ when $X$ has a rational 
point.



%%% Local Variables: 
%%% mode: latex
%%% TeX-master: "ant"
%%% End: 
