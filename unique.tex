\chapter{Unique Factorization of Ideals}
Unique factorization into irreducible elements frequently fails for
rings of integers of number fields.  In this chapter we will deduce a
central property of the ring of integers $\O_K$ of an algebraic number
field, namely that every nonzero {\em ideal} factors uniquely as a
products of prime ideals.  Along the way, we will introduce fractional
ideals and prove that they form a free abelian group under
multiplication.  Factorization of {\em elements} of $\O_K$ (and much
more!) is governed by the class group of $\O_K$, which is the quotient
of the group of fractional ideals by the principal fractional ideals
(see Chapter~\ref{ch:classgroup}).

\section{Dedekind Domains}
Recall (Corollary~\ref{prop:intnoetherian}) that we proved that the
ring of integers $\O_K$ of a number field is noetherian as follows.
As we saw before using norms, the ring $\O_K$ is finitely generated as
a module over~$\Z$, so it is certainly finitely generated as a ring
over~$\Z$.  By the Hilbert Basis Theorem (Theorem~\ref{thm:hilbert}),
$\O_K$ is noetherian.

If $R$ is an integral domain, the \defn{field of fractions} $\Frac(R)$
of~$R$ is the field of all equivalence classes of formal quotients
$a/b$, where $a,b \in R$ with $b\neq 0$, and $a/b\sim c/d$ if $ad=bc$.
For example, the field of fractions of~$\Z$ is (canonically isomorphic
to) $\Q$ and the field of fractions of $\Z[(1+\sqrt{5})/2]$ is
$\Q(\sqrt{5})$.  The field of fractions of the ring $\O_K$ of integers
of a number field~$K$ is just the number field~$K$ (see Lemma~\ref{lem:intq}).

\begin{example}
We compute the fraction fields mentioned above.
\begin{sagecode}
\begin{sagecell}
Frac(ZZ)
\end{sagecell}
\begin{sageout}
Rational Field
\end{sageout}
\end{sagecode}
In Sage the {\tt Frac} command usually returns a field canonically
isomorphic to the fraction field (not a formal construction).
\begin{sagecode}
\begin{sagecell}
K.<a> = QuadraticField(5)
OK = K.ring_of_integers(); OK
\end{sagecell}
\begin{sageout}
Maximal Order in Number Field in a with defining polynomial x^2 - 5
\end{sageout}
\begin{sagecell}
OK.basis()
\end{sagecell}
\begin{sageout}
[1/2*a + 1/2, a]
\end{sageout}
\begin{sagecell}
Frac(OK)
\end{sagecell}
\begin{sageout}
Number Field in a with defining polynomial x^2 - 5
\end{sageout}
\end{sagecode} %link

Note that in computers {\tt 1/2 * x} means the same as {\tt (1/2)*x}. For more information about the order of operations in programming see
\url{https://docs.python.org/2/reference/expressions.html#operator-precedence}.
\noindent{}The fraction field of an {\em order} -- i.e., a subring of $\O_K$ of
finite index -- is also the number field again.
%link
\begin{sagecode}
\begin{sagecell}
O2 = K.order(2*a); O2
\end{sagecell}
\begin{sageout}
Order in Number Field in a with defining polynomial x^2 - 5
\end{sageout}
\begin{sagecell}
Frac(O2)
\end{sagecell}
\begin{sageout}
Number Field in a with defining polynomial x^2 - 5
\end{sageout}
\end{sagecode}
\end{example}

\begin{definition}[Integrally Closed]
An integral domain $R$ is \defn{integrally closed in its field of
fractions} if whenever $\alpha$ is in the field of fractions of $R$
and $\alpha$ satisfies a monic polynomial $f\in R[x]$, then $\alpha
\in R$.
\end{definition}

For example, every field is integrally closed in its field of
fractions, as is the ring $\Z$ of integers.  However, $\Z[\sqrt{5}]$
is not integrally closed in its field of fractions, since
$(1+\sqrt{5})/2$ is integrally over $\Z$ and lies in $\Q(\sqrt{5})$,
but not in $\Z[\sqrt{5}]$

\begin{proposition}\label{prop:integrallyclosed}\iprop{$\O_K$ is integrally closed}
If $K$ is any number field, then $\O_K$ is integrally closed.  Also,
the ring $\Zbar$ of all algebraic integers (in a fixed choice of $\Qbar$)
is integrally closed.
\end{proposition}
\begin{proof}
We first prove that~$\Zbar$ is integrally closed.  Suppose $\alpha\in\Qbar$
is integral over~$\Zbar$, so there is a monic polynomial $f(x)=x^n +
a_{n-1}x^{n-1} + \cdots + a_1 x + a_0$ with $a_i\in\Zbar$ and
$f(\alpha)=0$.  The $a_i$ all lie in the ring of integers $\O_K$ of the
number field $K=\Q(a_0,a_1,\ldots a_{n-1})$, and $\O_K$ is finitely
generated as a $\Z$-module, so $\Z[a_0,\ldots, a_{n-1}]$ is finitely
generated as a $\Z$-module.  Since $f(\alpha)=0$, we can write $\alpha^n$ as a
$\Z[a_0,\ldots, a_{n-1}]$-linear combination of $\alpha^i$ for $i<n$, so
the ring $\Z[a_0,\ldots, a_{n-1},\alpha]$ is also finitely generated as a
$\Z$-module.  Thus $\Z[\alpha]$ is finitely generated as a $\Z$-module
because it is a submodule of a finitely generated $\Z$-module, which
implies that~$\alpha$ is integral over~$\Z$.

Without loss we may assume that $K\subset \Qbar$, so that
$\O_K = \Zbar \cap K$.
Suppose $\alpha\in K$ is integral over $\O_K$.  Then since $\Zbar$ is
integrally closed,~$\alpha$ is an element of $\Zbar$, so
$\alpha\in K \cap \Zbar=\O_K$, as required.
\end{proof}

\begin{exercise}\label{ex:Zbarnotnoetherian}
	Prove that $\Zbar$ is not noetherian. Hint: Consider an ideal
	generated by fractional powers of a prime.
\end{exercise}

\begin{definition}[Dedekind Domain]
An integral domain~$R$ is a \defn{Dedekind domain} if it is noetherian,
integrally closed in its field of fractions, and every nonzero prime
ideal of $R$ is maximal.
\end{definition}

\begin{exercise}
Let $K$ be a field.
\begin{enumerate}
\item[(a)] Prove that the polynomial ring $K[x]$ is a Dedekind domain.
\item[(b)] Is $\Z[x]$ a Dedekind domain?
\end{enumerate}
\end{exercise}

The ring $\Z\oplus \Z$ is not a Dedekind domain because it is not an
integral domain.  The ring $\Z[\sqrt{5}]$ is not a Dedekind domain
because it is not integrally closed in its field of fractions.  The
ring~$\Z$ is a Dedekind domain, as is any ring of integers $\O_K$ of a
number field, as we will see below.  Also, any field $K$ is a Dedekind
domain, since it is an integral domain, it is trivially integrally
closed in itself, and there are no nonzero prime ideals so the
condition that they be maximal is empty.

\begin{exercise}
	In Proposition~\ref{prop:integrallyclosed} we showed
	that $\Zbar$ is integrally closed in its field of fractions.
	Prove that and every nonzero prime ideal of $\Zbar$
	is maximal. Together with Exercise~\ref{ex:Zbarnotnoetherian},
	this shows $\Zbar$ is not a Dedekind domain only because it
	is not noetherian.
\end{exercise}

\begin{proposition}\iprop{$\O_K$ is Dedekind}
The ring of integers $\O_K$ of a number field is a Dedekind domain.
\end{proposition}
\begin{proof}
By Proposition~\ref{prop:integrallyclosed}, the ring $\O_K$ is
integrally closed, and by Proposition~\ref{prop:intnoetherian} it is
noetherian.  Suppose that~$\p$ is a nonzero prime ideal of $\O_K$.
Let $\alpha\in \p$ be a nonzero element, and let $f(x)\in\Z[x]$ be the
minimal polynomial of~$\alpha$.  Then
$$f(\alpha)=\alpha^n+a_{n-1}\alpha^{n-1}+\cdots+a_1\alpha+a_0=0,$$ so
$a_0 = -(\alpha^n+a_{n-1}\alpha^{n-1}+\cdots+a_1\alpha)\in \p$.  Since
$f$ is irreducible, $a_0$ is a nonzero element of $\Z$ that lies
in~$\p$.  Every element of the finitely generated abelian group
$\O_K/\p$ is killed by~$a_0$, so $\O_K/\p$ is a finite set.
Since~$\p$ is prime, $\O_K/\p$ is an integral domain.  Every finite
integral domain is a field (see Exercise~\ref{ex:finitedomain}), so
$\p$ is maximal, which completes the proof.
\end{proof}

\section{Factorization of Ideals}

If $I$ and $J$ are ideals in a ring $R$, the product $IJ$ is the ideal
{\em generated by} all products of elements in $I$ with elements in $J$:
$$
  IJ = (ab : a\in I, b\in J) \subset R.
$$
Note that the set of all products $ab$, with $a\in I$ and $b\in J$,
need not be an ideal, so it is important to take the ideal generated
by that set (see Exercise~\ref{ex:idealprod}).

\begin{definition}[Fractional Ideal]\label{def:fracideal}
A \defn{fractional ideal} is a nonzero $\O_K$-submodule~$I$ of~$K$ that
is finitely generated as an $\O_K$-module.
\end{definition}
We will sometimes call a genuine ideal $I\subset \O_K$ an
\defn{integral ideal}.  The notion of fractional ideal makes
sense for an arbitrary Dedekind domain $R$ -- it is an
$R$-module $I\subset K=\Frac(R)$ that is finitely
generated as an $R$-module.

\begin{example}
We multiply two fractional ideals in \sage:
\begin{sagecode}
\begin{sagecell}
K.<a> = NumberField(x^2 + 23)
I = K.fractional_ideal(2, 1/2*a - 1/2)
J = I^2
I
\end{sagecell}
\begin{sageout}
Fractional ideal (2, 1/2*a - 1/2)
\end{sageout}
\begin{sagecell}
J
\end{sagecell}
\begin{sageout}
Fractional ideal (4, 1/2*a + 3/2)
\end{sageout}
\begin{sagecell}
I*J
\end{sagecell}
\begin{sageout}
Fractional ideal (1/2*a + 3/2)
\end{sageout}
\end{sagecode}
\end{example}

Since fractional ideals $I$ are finitely generated, we can clear
denominators of a generating set to see that there exists some nonzero
$\alpha\in K$ such that
$$
\alpha I = J \subset \O_K,
$$
with $J$ an integral ideal.  Thus dividing by $\alpha$, we see
that every fractional ideal is
of the form
$$a J = \{a b : b \in J\}$$
for some $a\in K$ and integral ideal $J\subset \O_K$.

For example, the set $\frac{1}{2}\Z$ of rational numbers with
denominator $1$ or $2$ is a fractional ideal of $\Z$.

\begin{theorem}\label{thm:ddabgrp}
\ithm{fractional ideals}
The set of fractional ideals of a Dedekind domain~$R$ is an
abelian group under ideal multiplication with identity element $R$.
\end{theorem}
Note that fractional ideals are nonzero by definition, so it is not
necessary to write ``nonzero fractional ideals'' in the statement of
the theorem. We will {\em only} prove Theorem~\ref{thm:ddabgrp} in the
case when $R=\O_K$ is the ring of integers of a number field $K$. The
general case can be found in many algebraic number theory books such
as \cite[Chapter 3]{marcus1977number}.
Before proving Theorem~\ref{thm:ddabgrp} we prove a lemma.  For the
rest of this section $\O_K$ is the ring of integers of a number
field~$K$.


\begin{definition}[Divides for Ideals]
Suppose that $I,J$ are ideals of $\O_K$.
Then we say that~$I$ \defn{divides}~$J$ if $I\supset J$.
\end{definition}
To see that this notion of divides is sensible, suppose $K=\Q$, so
$\O_K=\Z$.  Then $I=(n)$ and $J=(m)$ for some integer $n$ and $m$, and
$I$ divides $J$ means that $(n)\supset (m)$, i.e., that there exists
an integer~$c$ such that $m=cn$, which exactly means that $n$ divides
$m$, as expected.

\begin{lemma}\label{lem:divprod}\ilem{$I$ divides product of primes}
Suppose~$I$ is a nonzero ideal of $\O_K$.  Then there exist prime ideals
$\p_1,\ldots, \p_n$ such that $\p_1\cdot \p_2\cdots \p_n \subset{}I$,
i.e.,~$I$ divides a product of prime ideals.
\end{lemma}
\begin{proof}
Let $S$ be the set of nonzero ideals of $\O_K$ that do not
satisfy the conclusion
of the lemma.  The key idea is to use that $\O_K$ is noetherian to show that
$S$ is the empty set.   If~$S$ is
nonempty, then since $\O_K$ is noetherian, there is an ideal
$I\in S$ that is maximal as an element of~$S$.  If~$I$ were prime, then~$I$
would trivially contain a product of primes, so we may assume that~$I$
is not prime.  Thus there exists $a,b\in \O_K$ such that $ab\in
I$ but $a\not\in I$ and $b\not\in I$.  Let $J_1 = I+(a)$ and
$J_2=I+(b)$.  Then neither $J_1$ nor $J_2$ is in $S$, since~$I$ is
maximal, so both $J_1$ and $J_2$ contain a product of prime ideals,
say $\p_1 \cdots \p_r \subset J_1$ and $\q_1\cdots \q_s \subset J_2$.
Then
$$
\p_1  \cdots \p_r \cdot \q_1\cdots \q_s \subset
J_1 J_2 = I^2 + I(b)+  (a)I+ (ab) \subset I,$$
so $I$ contains a product of primes.  This is a contradiction,
since we assumed $I\in S$.   Thus~$S$ is empty, which completes
the proof.
\end{proof}
We are now ready to prove the theorem.

\begin{proof}[Proof of Theorem~\ref{thm:ddabgrp}]
  Note that we will {\em only} prove Theorem~\ref{thm:ddabgrp} in the
  case when $R=\O_K$ is the ring of integers of a number field $K$.

The product of two fractional ideals is again finitely generated, so
it is a fractional ideal, and $I\O_K=I$ for any ideal~$I$,
so to prove that the set of fractional ideals under multiplication is
a group it suffices to show the existence of inverses.  We will first
prove that if~$\p$ is a prime ideal, then~$\p$ has an inverse, then we
will prove that all nonzero integral ideals have inverses, and finally
observe that every fractional ideal has an inverse.  (Note: Once we
know that the set of fractional ideals is a group, it will follows
that inverses are unique; until then we will be careful to write
``an'' instead of ``the''.)

Suppose $\p$ is a nonzero prime ideal of $\O_K$.   We will show that
the $\O_K$-module
$$
  I = \{a \in K : a\p \subset \O_K \}
$$
is a fractional ideal of $\O_K$ such that $I\p = \O_K$, so that
$I$ is an inverse of $\p$.

For the rest of the proof, fix a nonzero element $b\in \p$.  Since~$I$
is an $\O_K$-module, $bI\subset \O_K$ is an $\O_K$ ideal, hence~$I$ is
a fractional ideal.  Since $\O_K \subset I$ we have $\p \subset I \p
\subset \O_K$, hence since~$\p$ is maximal, either $\p = I\p$ or $I\p
= \O_K$.  If $I\p=\O_K$, we are done since then~$I$ is an inverse
of~$\p$.  Thus suppose that $I\p=\p$.  Our strategy is to show that
there is some $d\in I$, with $d\not\in\O_K$.  Since $I\p=\p$, such
a~$d$ would leave~$\p$ invariant, i.e., $d \p \subset \p$.  Since $\p$
is a finitely generated $\O_K$-module we will see that it will follow that $d\in \O_K$,
a contradiction.

By Lemma~\ref{lem:divprod}, we can choose a product $\p_1,\ldots, \p_m$,
with~$m$ minimal, with
$$
\p_1\p_2\cdots \p_m \subset (b) \subset \p.
$$ If no $\p_i$ is contained in $\p$, then we can choose for each $i$
an $a_i \in \p_i$ with $a_i\not\in \p$; but then $\prod a_i\in \p$,
which contradicts that $\p$ is a prime ideal.  Thus some $\p_i$, say
$\p_1$, is contained in $\p$, which implies that $\p_1 = \p$ since
every nonzero prime ideal is maximal.  Because~$m$ is minimal,
$\p_2\cdots \p_m$ is not a subset of $(b)$, so there exists $c\in
\p_2\cdots \p_m$ that does not lie in $(b)$. Then $\p(c) \subset (b)$,
so by definition of $I$ we have $d=c/b\in I$.  However, $d\not\in
\O_K$, since if it were then $c$ would be in $(b)$.  We have thus
found our element $d\in I$ that does not lie in $\O_K$.

To finish the proof that $\p$ has an inverse, we observe that $d$
preserves the finitely generated $\O_K$-module $\p$, and is hence in $\O_K$, a
contradiction.  More precisely, if $b_1,\ldots, b_n$ is a basis for
$\p$ as a $\Z$-module, then the action of~$d$ on~$\p$ is given by a
matrix with entries in~$\Z$, so the minimal polynomial of~$d$ has
coefficients in~$\Z$ (because $d$ satisfies the minimal polynomial of
$\ell_d$, by the Cayley-Hamilton theorem -- here we also use that
$\QQ\tensor\p = K$, since $\O_K/\p$ is a finite set).  This implies
that~$d$ is integral over~$\Z$, so $d\in \O_K$, since~$\O_K$ is
integrally closed by Proposition~\ref{prop:integrallyclosed}.  (Note
how this argument depends strongly on the fact that $\O_K$ is
integrally closed!)

So far we have proved that if $\p$ is a prime ideal of $\O_K$, then
$$
  \p^{-1} = \{a \in K : a\p \subset \O_K\}
$$
is the inverse of $\p$ in
the monoid of nonzero fractional ideals of $\O_K$.  As mentioned after
Definition~\ref{def:fracideal}, every nonzero fractional
ideal is of the form $aI$ for $a\in K$ and $I$ an integral ideal, so
since $(a)$ has inverse $(1/a)$, it suffices to show that every
integral ideal~$I$ has an inverse.  If not, then there is a nonzero
integral ideal~$I$ that is maximal among all nonzero integral ideals
that do not have an inverse.  Every ideal is contained in a maximal
ideal, so there is a nonzero prime ideal $\p$ such that $I\subset \p$.
Multiplying both sides of this inclusion by $\p^{-1}$ and using that
$\O_K\subset \p^{-1}$, we see that
$$
  I \subset \p^{-1} I \subset \p^{-1}\p = \O_K.
$$
If $I = \p^{-1} I$, then arguing as in the proof that $\p^{-1}$ is an
inverse of~$\p$, we see that each element of $\p^{-1}$ preserves the
finitely generated $\Z$-module~$I$ and is hence integral.  But then
$\p^{-1}\subset \O_K$, which, upon multiplying both sides by $\p$,
implies that $\O_K = \p \p^{-1} \subset \p$, a contradiction.
Thus $I \neq \p^{-1} I$.  Because $I$ is
maximal among ideals that do not have an inverse, the ideal $\p^{-1}
I$ does have an inverse~$J$.
Then $\p{}^{-1}J$ is an inverse of~$I$, since
$
(J\p^{-1})I = J(\p^{-1}I) = \O_K.
$
\end{proof}

We can finally deduce the crucial Theorem~\ref{thm:intuniqfac}, which
will allow us to show that any nonzero ideal of a Dedekind domain can
be expressed uniquely as a product of primes (up to order).  Thus
unique factorization holds for ideals in a Dedekind domain, and it is
this unique factorization that initially motivated the introduction of
ideals to mathematics over a century ago.

\begin{theorem}\label{thm:intuniqfac}\ithm{unique ideal factorization}
Suppose $I$ is a nonzero integral ideal of $\O_K$.  Then $I$ can
be written as a product
$$
  I = \p_1\cdots \p_n
$$ of prime ideals of $\O_K$, and this representation is unique up to
order.
\end{theorem}
\begin{proof}
Suppose~$I$ is an ideal that is maximal among the set of all ideals in
$\O_K$ that cannot be written as a product of primes.  Every ideal is
contained in a maximal ideal, so $I$ is contained in a nonzero prime
ideal $\p$.  If $I\p^{-1} = I$, then by Theorem~\ref{thm:ddabgrp} we
can cancel $I$ from both sides of this equation to see that
$\p^{-1}=\O_K$, a contradiction.  Since $\O_K\subset \p^{-1}$,
we have $I\subset I\p^{-1}$, and by the above observation~$I$
is strictly contained in
$I\p^{-1}$.  By our maximality assumption on~$I$, there are maximal
ideals $\p_1,\ldots, \p_n$ such that $I\p^{-1} = \p_1\cdots \p_n$.
Then $I=\p\cdot \p_1\cdots \p_n$, a contradiction.  Thus every ideal
can be written as a product of primes.

Suppose $\p_1\cdots \p_n=\q_1\cdots \q_m$. If no $\q_i$ is contained in
$\p_1$, then for each $i$ there is an $a_i\in \q_i$ such that
$a_i\not\in\p_1$.  But the product of the $a_i$ is in $\p_1\cdots
\p_n$, which is a subset of $\p_1$, which contradicts that
$\p_1$ is a prime ideal.  Thus $\q_i=\p_1$ for some~$i$.  We can thus
cancel $\q_i$ and $\p_1$ from both sides of the equation by multiplying
both sides by the inverse.  Repeating
this argument finishes the proof of uniqueness.
\end{proof}

\begin{theorem}\label{thm:uniqfac}\icor{factorization
of fractional ideals}
If $I$ is a fractional ideal of $\O_K$ then there exists
prime ideals $\p_1,\ldots, \p_n$ and $\q_1,\ldots, \q_m$,
unique up to order, such that
$$
  I = (\p_1\cdots \p_n)(\q_1\cdots \q_m)^{-1}.
$$
\end{theorem}
\begin{proof}
We have $I=(a/b)J$ for some $a,b\in\O_K$ and integral ideal $J$.
Applying Theorem~\ref{thm:intuniqfac} to $(a)$, $(b)$, and $J$ gives
an expression as claimed.  For uniqueness, if one has two such product
expressions, multiply through by the denominators and use the
uniqueness part of Theorem~\ref{thm:intuniqfac}
\end{proof}

\begin{example}
The ring of integers of $K=\Q(\sqrt{-6})$ is $\O_K=\Z[\sqrt{-6}]$.
We have
$$
  6 = -\sqrt{-6}\sqrt{-6} = 2 \cdot 3.
$$ If $ab=\sqrt{-6}$, with $a,b\in \O_K$ and neither a unit, then
$\Norm(a)\Norm(b) = 6$, so without loss $\Norm(a)=2$ and
$\Norm(b)=3$. If $a=c + d\sqrt{-6}$, then $\Norm(a) = c^2 + 6d^2$;
since the equation $c^2 + 6d^2 = 2$ has no solution with $c,d\in\Z$,
there is no element in $\O_K$ with norm~$2$, so $\sqrt{-6}$ is
irreducible.  Also, $\sqrt{-6}$ is not a unit times~$2$ or times $3$,
since again the norms would not match up.  Thus $6$ cannot be written
uniquely as a product of irreducibles in $\O_K$.
Theorem~\ref{thm:uniqfac}, however, implies that the principal
ideal $(6)$ can, however, be
written uniquely as a product of prime ideals.
An explicit decomposition is
\begin{equation}\label{eqn:fac6}
(6) = (2, 2+\sqrt{-6})^2 \cdot (3,3+\sqrt{-6})^2,
\end{equation}
where each of the ideals $(2, 2+\sqrt{-6})$ and $(3, 3+\sqrt{-6})$ is
prime.  We will discuss algorithms for computing such a decomposition
in detail in Chapter~\ref{ch:factoring_primes}.  The first idea is to
write $(6)=(2)(3)$, and hence reduce to the case of writing the $(p)$,
for $p\in\Z$ prime, as a product of primes.  Next one decomposes the
finite (as a set) ring $\O_K/p\O_K$.

The factorization (\ref{eqn:fac6}) can be compute using
\sage{} as follows:
\begin{sagecode}
\begin{sagecell}
K.<a> = NumberField(x^2 + 6); K
\end{sagecell}
\begin{sageout}
Number Field in a with defining polynomial x^2 + 6
\end{sageout}
\begin{sagecell}
K.factor(6)
\end{sagecell}
\begin{sageout}
(Fractional ideal (2, a))^2 * \
(Fractional ideal (3, a))^2
\end{sageout}
\end{sagecode}

%{\small
%\begin{lstlisting}
%   > R<x> := PolynomialRing(RationalField());
%   > K := NumberField(x^2+6);
%   > OK := RingOfIntegers(K);
%   > [K!b : b in Basis(OK)];
%   [ 1,
%     K.1]    // this is sqrt(-6)
%   > Factorization(6*OK);
%   [
%       <Prime Ideal of OK
%       Two element generators:
%           [2, 0]
%           [2, 1], 2>,
%       <Prime Ideal of OK
%       Two element generators:
%           [3, 0]
%           [3, 1], 2>
%   ]
%\end{lstlisting}
%}
%
%The factorization (\ref{eqn:fac6}) can also be computed
%using PARI (see \cite{pari}).
%{\small
%\begin{lstlisting}
%   ? k=nfinit(x^2+6);
%   ? idealfactor(k, 6)
%   [[2, [0, 1]~, 2, 1, [0, 1]~] 2]
%   [[3, [0, 1]~, 2, 1, [0, 1]~] 2]
%   ? k.zk
%   [1, x]
%\end{lstlisting}
%}
%The output of PARI is a list of two prime ideals with exponent $2$.
%A prime ideal is represented by a $5$-tuple
%$[p,a,e,f,b]$, where the ideal is $p\O_K + \alpha\O_K$,
%where $\alpha = \sum a_i \omega_i$, where $\omega_1,\ldots, \omega_n$
%are a basis for $\O_K$ (as output by {\tt k.zk}).

\end{example}



%%% Local Variables:
%%% mode: latex
%%% TeX-master: "ant"
%%% End: