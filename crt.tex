\chapter{The Chinese Remainder Theorem}\label{ch:crt}
In this chapter, we prove the Chinese Remainder Theorem (CRT) for
arbitrary commutative rings, then apply CRT to prove that every ideal
in a Dedekind domain $R$ is generated by at most two elements.  We
also prove that $\p^n/\p^{n+1}$ is (noncanonically) isomorphic to
$R/\p$ as an $R$-module, for any nonzero prime ideal $\p$ of $R$.  The
tools we develop in this chapter will be used frequently to prove
other results later.

\section{The Chinese Remainder Theorem}

\subsection{CRT in the Integers}
The classical CRT asserts that if $n_1,\ldots, n_r$ are integers that are coprime
in pairs, and $a_1,\ldots, a_r$ are integers, then there exists an
integer~$a$ such that $a\con a_i\pmod{n_i}$ for each $i=1,\ldots,r$.
Here ``coprime in pairs'' means that $\gcd(n_i,n_j)=1$ whenever
$i\neq j$; it does {\em not} mean that $\gcd(n_1,\ldots, n_r)=1$,
though it implies this. 
In terms of rings, CRT asserts that the
natural map
\begin{equation}\label{eqn:crt}
\Z/(n_1\cdots n_r)\Z \to (\Z/n_1\Z)\oplus \cdots \oplus (\Z/n_r\Z)
\end{equation}
that sends $a \in \Z$ to its reduction modulo each $n_i$,
is an isomorphism.  

This map is {\em never} an isomorphism if the $n_i$ are not coprime.
Indeed, the cardinality of the image of the left hand side of
(\ref{eqn:crt}) is $\lcm(n_1,\ldots, n_r)$, since it is the image of a
cyclic group and $\lcm(n_1,\ldots, n_r)$ is the largest order of an
element of the right hand side, whereas the cardinality of the right
hand side is $n_1\cdots n_r$.

The isomorphism (\ref{eqn:crt}) can alternatively be viewed as
asserting that any system of linear congruences
$$
  x \con a_1 \pmod{n_1}, \quad  
x \con a_2 \pmod{n_2}, \quad
\ldots, \quad
x \con a_r \pmod{n_r}
$$
with pairwise coprime moduli has a unique solution modulo $n_1\cdots n_r$.

Before proving the CRT in more generality, we prove
(\ref{eqn:crt}).
There is a natural map 
$$
  \phi: \Z \to (\Z/n_1\Z)\oplus \cdots \oplus (\Z/n_r\Z)
$$
given by projection onto each factor.  Its kernel is
$$
 n_1 \Z \cap \cdots \cap n_r \Z.
$$ 
If $n$ and $m$ are integers, then $n\Z\cap m\Z$ is the
set of multiples of both $n$ and $m$, so $n\Z\cap m\Z = \lcm(n,m)\Z$.
Since the $n_i$ are coprime, 
$$
 n_1 \Z \cap \cdots \cap n_r \Z = n_1 \cdots n_r \Z.
$$ 
Thus we have proved there is an inclusion 
\begin{equation}\label{eqn:crt_inj}
 i: \Z/(n_1\cdots n_r)\Z \hra (\Z/n_1\Z)\oplus \cdots \oplus (\Z/n_r\Z).
\end{equation}
This is half of the CRT; the other half is to prove that this map is
surjective.  In this case, it is clear that $i$ is also surjective,
because $i$ is an injective map between finite sets of the same cardinality.
We will, however, give a proof of surjectivity that doesn't use
finiteness of the above two sets.

To prove surjectivity of $i$, note that since the $n_i$ are coprime in
pairs, $$\gcd(n_1, n_2\cdots n_r)=1,$$ so there exists integers $x,y$
such that
$$
   x n_1 + y n_2\cdots n_r = 1.
$$
To complete the proof, observe that 
$ y n_2\cdots n_r = 1 - x n_1$
is congruent to~$1$ modulo $n_1$ and~$0$ modulo $n_2\cdots n_r$.
Thus $(1,0,\ldots,0) = i (y n_2\cdots n_r)$ is in the image of~$i$.  
By a similar argument, we see that $(0,1,\ldots,0)$ and the
other similar elements are all in the image of~$i$, so $i$
is surjective, which proves CRT.  

\subsection{CRT in General}
Recall that {\em all rings in this book are commutative with unity}.
Let $R$ be such a ring.
\begin{definition}[Coprime]
Ideals $I$ and $J$ of $R$ are \defn{coprime} if $I+J=(1)$.
\end{definition}
For example, if~$I$ and~$J$ are nonzero ideals in a Dedekind domain,
then they are coprime precisely when the prime ideals that appear in
their two (unique) factorizations are disjoint.

\begin{lemma}\label{lem:prodint}\ilem{$I\cap{}J = IJ$}
If $I$ and $J$ are coprime ideals in a ring $R$, then
$I\cap{}J = IJ$.
\end{lemma}
\begin{proof}
Choose $x\in I$ and $y\in J$
such that $x+y=1$.  If $c\in{} I\cap{} J$ then 
$$c=c\cdot 1=c\cdot (x+y) = cx + cy \in IJ + IJ = IJ,$$
so $I\cap{} J\subset IJ$.  
The other inclusion is obvious by the definition of an ideal.
\end{proof}

\begin{lemma}\label{lem:coprime_prod}
Suppose $I_1,\ldots, I_s$ are pairwise coprime ideals.
Then $I_1$ is coprime to the product $I_2\cdots I_s$.
\end{lemma}
\begin{proof}
In the special case of a Dedekind domain, we could easily
prove this lemma using unique factorization of ideals as
products of primes (Theorem~\ref{thm:intuniqfac}); instead,
we give a direct general argument.

It suffices to prove the lemma in the case $s=3$, since the
general case then follows from induction.
By assumption, there
are $x_1 \in I_1, y_2 \in I_2$ and $a_1 \in I_1, b_3 \in I_3$
such
$$ 
x_1 + y_2 = 1 \qquad\text{and}\qquad a_1 + b_3 = 1.
$$
Multiplying these two relations yields
$$
x_1 a_1 + x_1 b_3 + y_2 a_1 + y_2 b_3 = 1 \cdot 1 = 1.
$$
The first three terms are in $I_1$ and the last term is in 
$I_2 I_3 = I_2 \cap I_3$ (by Lemma~\ref{lem:prodint}), 
so $I_1$ is coprime to $I_2 I_3$.
\end{proof}

Next we prove the general Chinese Remainder Theorem.
We will apply this result with $R=\O_K$ in the rest of this chapter.
\begin{theorem}[Chinese Remainder Theorem]\label{thm:crt}
\ithm{chinese remainder}
Suppose $I_1,\ldots, I_r$ are nonzero ideals of a ring~$R$ such 
$I_m$ and $I_n$ are coprime for any $m\neq n$.  Then the natural
homomorphism $R \to \bigoplus_{n=1}^r R/I_n$ induces an isomorphism
$$
\psi: R/\prod_{n=1}^r I_n \to \bigoplus_{n=1}^r R/I_n.
$$
Thus given any $a_n \in R$, for $n=1,\ldots,r$, there exists some $a\in R$
such that $a\con a_n\pmod{I_n}$ for $n=1,\ldots, r$; moreover,~$a$ 
is unique modulo $\prod_{n=1}^r I_n$.
\end{theorem}

\begin{proof}
Let 
$
  \vphi:R \to \bigoplus_{n=1}^r R/I_n
$
be the natural map induced by reduction modulo
the $I_n$.  
An inductive application of Lemma~\ref{lem:prodint} 
implies that
the kernel $\cap_{n=1}^r I_n$ of~$\vphi$ 
is equal to 
$\prod_{n=1}^r I_n$, so the map~$\psi$ of the theorem is injective.  

Each projection $R\to R/I_n$ is  surjective, so to prove
that $\psi$ is surjective, it suffices 
to show that $(1,0,\ldots,0)$
is in the image of~$\vphi$, and similarly for the other
factors.  By Lemma~\ref{lem:coprime_prod}, 
$J=\prod_{n=2}^rI_n$ is coprime to~$I_1$, so
there exists $x\in I_1$ and $y \in J$ such that
$x+y=1$.  Then $y = 1-x$ maps to~$1$ in 
$R/I_1$ and to~$0$ in $R/J$, hence to~$0$ in $R/I_n$
for each $n\geq 2$, since $J\subset I_n$. 
\end{proof}

\section{Structural Applications of the CRT}
Let $\O_K$ be the ring of integers of some number field $K$, and
suppose~$I$ is a nonzero ideal of $\O_K$.  As an abelian group $\O_K$
is free of rank $[K:\Q]$, and~$I$ is of finite index in $\O_K$, so~$I$
is generated by $[K:\Q]$ generators as an abelian group, so as an
$R$-ideal $I$ requires at most $[K:\Q]$ generators.  The main result
of this section asserts something better, namely that~$I$ can be
generated {\em as an ideal} by at most two elements.  Moreover, our
result is more general, since it applies to an arbitrary Dedekind
domain $R$. Thus, for the rest of this section, $R$ is any Dedekind
domain, e.g., the ring of integers of either a number field or
function field.  We use CRT to prove that every ideal of $R$ can be
generated by two elements.  

\begin{remark}
Caution -- If we replace $R$ by an order in a Dedekind domain, i.e.,
by a subring of finite index,then there may be ideals that require far more than $2$ generators. 
\end{remark}

Suppose that~$I$ is a nonzero integral ideal of
$R$.  If $a\in I$, then $(a)\subset I$, so~$I$ divides~$(a)$ and
the quotient $(a)I^{-1}$ is an integral ideal.  The following lemma
asserts that~$(a)$ can be chosen so the quotient $(a)I^{-1}$ is coprime to
any given ideal.
\begin{lemma}\label{lem:magica}
If $I$ and $J$ are nonzero integral ideals in $R$, then there exists
an $a\in I$ such that the integral ideal $(a)I^{-1}$ is coprime to~$J$.
\end{lemma}
Before we give the proof in general, note that the lemma is trivial
when $I$ is principal, since if $I=(b)$, just take $a=b$, and 
then $(a)I^{-1} = (a)(a^{-1})= (1)$ is coprime to every ideal.
\begin{proof}
Let $\p_1,\ldots, \p_r$ be the prime divisors of~$J$.
For each $n$, let $v_n$ be the largest power of $\p_n$
that divides~$I$.  
Since $\p_n^{v_n}\neq \p_n^{v_n+1}$, 
we can choose an element $a_n\in \p_n^{v_n}$
that is not in $\p_n^{v_n+1}$.
By Theorem~\ref{thm:crt} applied to
the $r+1$ coprime integral ideals
$$
  \p_1^{v_1+1}, \ldots, \p_r^{v_r+1}, \, I\cdot \left(\prod \p_n^{v_n}\right)^{-1},
$$
there exists $a\in R$
such that
$$
   a \con a_n \pmod{\p_n^{v_n+1}}
$$
for all $n=1,\ldots, r$ and
also 
$$
   a \con 0 \quad \left(\text{mod}\,\,\, I\cdot \left(\prod \p_n^{v_n}\right)^{-1}\right).
$$

To complete the proof we show that $(a)I^{-1}$ is not
divisible by any $\p_n$, or equivalently, that each 
$\p_n^{v_n}$ exactly divides $(a)$. 
First we show that $\p_n^{v_n}$ divides $(a)$. Because
$a\con a_n \pmod{\p_n^{v_n+1}}$, there exists 
$b \in \p_n^{v_n+1}$ such that $a = a_n + b$.  Since
$a_n\in \p_n^{v_n}$ and $b \in \p_n^{v_n + 1} \subset \p_n^{v_n}$, 
it follows that $a\in \p_n^{v_n}$,
so $\p_n^{v_n}$ divides~$(a)$.  
Now assume for the sake of contradiction that
$\p_n^{v_n+1}$ divides $(a$); then $a_n=a-b\in \p_n^{v_n+1}$, which
contradicts that we chose $a_n \not\in\p_n^{v_n+1}$.
Thus
$\p_n^{v_n+1}$ does not divide $(a)$, as claimed. 
\end{proof}



\begin{proposition}\iprop{ideals generated by two elements}\label{prop:2gen}
Suppose $I$ is a fractional ideal in a Dedekind domain $R$.  Then there exist $a,b\in{}K$ such that 
$I=(a,b)=\{\alpha a + \beta b : \alpha,\beta \in R\}$.
\end{proposition}
\begin{proof}
If $I=(0)$, then $I$ is generated by $1$ element and we are done.  If
$I$ is not an integral ideal, then there is an $x\in K$ such that $xI$ is
an integral ideal, and the number of generators of $xI$ is the same as
the number of generators of $I$, so we may assume that $I$ is an
integral ideal.  

Let $a$ be {\em any} nonzero element of the integral ideal~$I$.  We
will show that there is some $b\in I$ such that $I=(a,b)$.  Let
$J=(a)$.  By Lemma~\ref{lem:magica}, there exists $b\in I$ such that
$(b)I^{-1}$ is coprime to $(a)$.  Since $a,b\in I$, we have $I\mid
(a)$ and $I\mid (b)$, so $I\mid (a,b)$.  Suppose $\p^n\mid (a,b)$ with
$\p$ prime and $n\geq 1$.  Then $\p^n\mid (a)$ and $\p^n\mid (b)$, so
$\p\nmid (b)I^{-1}$, since $(b)I^{-1}$ is coprime to $(a)$.  We have
$\p^n\mid(b) = I\cdot (b)I^{-1}$ and $\p\nmid (b)I^{-1}$, so $\p^n
\mid I$.  Thus by unique factorization of ideals in $R$ we
have that $(a,b)\mid I$.  Since $I \mid (a,b)$ we conclude
that  $I=(a,b)$, as claimed.
\end{proof}

We can also use Theorem~\ref{thm:crt} to determine the
$R$-module structure of $\p^n/\p^{n+1}$.
\begin{proposition}\label{prop:quopow}\iprop{structure of $\p^n/\p^{n+1}$}
Let $\p$ be a nonzero prime ideal of $R$, and let $n\geq 0$ be an
integer.  Then $\p^n/\p^{n+1} \isom R/\p$ as $R$-modules.
\end{proposition}
\begin{proof}[Proof~\footnote{Proof from \cite[pg.~13]{sd:brief}.}]
Since $\p^n\neq \p^{n+1}$, by unique factorization, 
there is an element $b\in
\p^n$ such that $b\not\in \p^{n+1}$.  Let
$\vphi:R\to\p^n/\p^{n+1}$ be the $R$-module morphism defined by
$\vphi(a)=ab$.  The kernel of $\vphi$ is $\p$ since clearly
$\vphi(\p)=0$ and if $\vphi(a)=0$ then $ab\in\p^{n+1}$, so
$\p^{n+1}\mid (a)(b)$, so $\p\mid (a)$, since $\p^{n+1}$ does not
divide~$(b)$.  Thus~$\vphi$ induces an injective $R$-module
homomorphism $R/\p\hra \p^{n}/\p^{n+1}$.

It remains to show that $\vphi$ is surjective, and this is where we
will use Theorem~\ref{thm:crt}.   Suppose $c\in \p^{n}$.
By Theorem~\ref{thm:crt} there exists $d\in R$
such that
$$
  d \con c\pmod{\p^{n+1}}
\qquad\text{and}\qquad
  d \con 0\pmod{(b)/\p^{n}}.
$$
We have $\p^n\mid (d)$ since $d\in\p^n$ and $(b)/\p^n\mid (d)$
by the second displayed condition, so 
since $\p\nmid(b)/\p^n$, we have $(b)=\p^n\cdot(b)/\p^n\mid (d)$, hence 
$d/b\in R$.   Finally
\[
 \vphi\left(\frac{d}{b}\right) \quad \equiv \quad \frac{d}{b}\cdot b \pmod{\p^{n+1}} 
 \quad \equiv \quad d\pmod{\p^{n+1}} \quad\equiv \quad c\pmod{\p^{n+1}},
\]
so $\vphi$ is surjective.
\end{proof}

\begin{exercise}\label{ex:residuefieldofpower}(See \cite[Thm.~22(a)]{marcus1977number}) %pg 67
	Let $R$ be a Dedekind domain and $\p$ a nonzero prime ideal in $R$.
	Show that $\#(R/\p^m) = \#(R/\p)^m$.
	
	Note: $\#(R/\p)$ is not finite in general! For example,
	The ring of formal power series $k[[t]]$ for some field $k$
	is a Dedekind domain and the residue field at the prime $(t)$
	is $k$.
	
	\begin{hint}
		Consider the exact sequence
		$$
			0\to \p/\p^{m} \to R/\p^{m} \to R/\p^{m-1} \to 0
		$$
		and the chain
		$$
			\p^m \subseteq \p^{m-1}
			\subseteq \cdots \subseteq \p^2 \subseteq \p.
		$$
	\end{hint}
\end{exercise}

\begin{remark}
	There is one special case of the previous exercise that you probably
	have seen before: the size of $\Z/4\Z$ is the same as
	$(\Z/2\Z)^2$. In fact you might have seen a proof of
	the fact that $\Z/n^m\Z$ has the same cardinality as $\left(\Z/n\Z\right)^m$
	in a standard group theory or abstract algebra course.
\end{remark}

\section{Computing Using the CRT}
In order to explicitly compute an $a$ as given by Theorem~\ref{thm:crt},
usually one first precomputes elements $v_1, \ldots, v_r \in R$ such that
$v_1\mapsto (1,0,\ldots, 0)$, 
$v_2\mapsto (0,1,\ldots, 0)$, etc.
Then given any $a_n \in R$, for $n=1,\ldots, r$, we obtain an $a \in R$
with $a_n\con a\pmod{I_n}$ by taking
$$
  a = a_1 v_1 + \cdots + a_r v_r.
$$
How to compute the $v_i$ depends on the ring~$R$.   It reduces to
the following problem: Given coprimes ideals $I,J\subset R$, find
$x\in I$ and $y\in J$ such that $x+y=1$.   If $R$ is torsion free and
of finite rank 
as a $\Z$-module, so $R\ncisom \Z^n$,
then $I, J$ can be represented by giving a basis in terms of a basis
for~$R$, and finding $x,y$ such  that $x+y=1$ can then be reduced to 
a problem in linear algebra over~$\Z$.  
More precisely, let~$A$
be the matrix whose columns are the concatenation of a basis for~$I$
with a basis for~$J$.  
Suppose $v\in \Z^n$ corresponds to $1\in\Z^n$.
Then finding $x,y$ such that $x+y=1$ is equivalent to
finding a solution $z\in \Z^n$ to the matrix equation
$Az = v$. This latter linear algebra problem
can be solved using Hermite normal form 
(see \cite[\S4.7.1]{cohen:course_ant}),
which is a generalization over $\Z$
of reduced row echelon form.

%We next describe how to use \magma{} and PARI to do CRT computations.

\subsection{\SAGE{}}
[[TODO]]

\subsection{\magma{}}
The \magma{} command {\tt ChineseRemainderTheorem} implements the
algorithm suggested by Theorem~\ref{thm:crt}.  In the following example,
we compute a prime over~$(3)$ and a prime over~$(5)$ of the ring of
integers of $\Q(\sqrt[3]{2})$, and find an element of $\O_K$ that is
congruent to $\sqrt[3]{2}$ modulo one prime and~$1$ modulo the other.
\begin{verbatim}
   > R<x> := PolynomialRing(RationalField());
   > K<a> := NumberField(x^3-2);
   > OK := RingOfIntegers(K);
   > I := Factorization(3*OK)[1][1];
   > J := Factorization(5*OK)[1][1];
   > I;
   Prime Ideal of OK
   Two element generators:
       [3, 0, 0]
       [4, 1, 0]
   > J;
   Prime Ideal of OK
   Two element generators:
       [5, 0, 0]
       [7, 1, 0]
   > b := ChineseRemainderTheorem(I, J, OK!a, OK!1);
   > K!b;
   -4
   > b - a in I;
   true
   > b - 1 in J;
   true
\end{verbatim}

\subsection{PARI}
There is also a CRT algorithm for number fields in PARI, but it
is more cumbersome to use.  First we defined $\Q(\sqrt[3]{2})$
and factor the ideals $(3)$ and $(5)$.
\begin{verbatim}
   ? f = x^3 - 2;
   ? k = nfinit(f);
   ? i = idealfactor(k,3);
   ? j = idealfactor(k,5);
\end{verbatim}

Next we form matrix whose rows correspond to a product of two primes,
one dividing $3$ and one dividing $5$:
\begin{verbatim}
   ? m = matrix(2,2);       
   ? m[1,] = i[1,];  
   ? m[1,2] = 1;
   ? m[2,] = j[1,];
\end{verbatim}
Note that we set {\tt m[1,2] = 1}, so the exponent is 1
instead of $3$.
We apply the CRT to obtain a lift in terms
of the basis for $\O_K$. 
\begin{verbatim}
   ? ?idealchinese
   idealchinese(nf,x,y): x being a prime ideal factorization and y 
   a vector of elements, gives an element b such that 
   v_p(b-y_p)>=v_p(x) for all prime ideals p dividing x, 
   and v_p(b)>=0 for all other p.
   ? idealchinese(k, m, [x,1])
   [0, 0, -1]~
   ? nfbasis(f)
   [1, x, x^2]
\end{verbatim}
Thus PARI finds the lift $-(\sqrt[3]{2})^2$, and we finish by
verifying that this lift is correct. The
{\tt idealval} function returns the number of times
a prime appears in the factorization of an ideal. We will use it
to check that $-(\sqrt[3]{2})^2 - \sqrt[3]{2}$ is contained in
the prime above $3$ and $-(\sqrt[3]{2})^2 - 1$ is contained in
the prime above $5$.
\begin{verbatim}
   ? idealval(k,-x^2 - x,i[1,1])
   1
   ? idealval(k,-x^2 - 1,j[1,1])
   1
\end{verbatim}


%%% Local Variables: 
%%% mode: latex
%%% TeX-master: "ant"
%%% End: 
