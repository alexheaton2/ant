\chapter{Extensions and Normalizations of Valuations}
\section{Extensions of Valuations}
In this section we continue to tacitly assume that all valuations are
nontrivial.  We do not assume all our valuations satisfy the triangle


Suppose $K\subset L$ is a finite extension of fields, and that $\absspc{}$ and $\normspc{}$
are valuations on~$K$ and~$L$, respectively.  
\begin{definition}[Extends]
We say that $\normspc{}$ \defn{extends}
$\absspc$ if $\abs{a} = \norm{a}$ for all $a\in K$.   
\end{definition}
\begin{theorem}\label{thm:extunique}\ithm{uniqueness of valuation extension}
Suppose that $K$ is a field that is complete with respect to $\absspc$
and that~$L$ is a finite extension of~$K$ of degree $N=[L:K]$.   
Then there is precisely
one extension of $\absspc{}$ to $K$, namely
\begin{equation}
  \norm{a} = \abs{\Norm_{L/K}(a)}^{1/N},
\label{eqn:normdef}\end{equation}
where the $N$th root is the non-negative real $N$th root of the
nonnegative real number $\abs{\Norm_{L/K}(a)}$.
\end{theorem}
\begin{proof}
We may assume that $\absspc$ is normalized so as
to satisfy the triangle inequality.  Otherwise, normalize
$\absspc$ so that it does, prove the theorem for the normalized 
valuation $\absspc^c$, then raise both sides of (\ref{eqn:normdef})
to the power $1/c$.  In the uniqueness proof, by the same
argument we may assume that $\normspc$ also satisfies the triangle
inequality.

\vspace{1ex}\noindent{\em Uniqueness.}  View $L$ as a
finite-dimensional vector space over~$K$. Then $\normspc$ is a norm in
the sense defined earlier (Definition~\ref{defn:norm}).  Hence any two
extensions $\normspc_1$ and $\normspc_2$ of $\absspc$ are equivalent
as norms, so induce the same topology on $K$.  But as we have
seen (Proposition~\ref{prop:same_topo}), two valuations which induce the same topology are
equivalent valuations, i.e., $\normspc_1 = \normspc_2^c$, for some
positive real $c$.  Finally $c=1$ since $\norm{a}_1 = \abs{a} =
\norm{a}_2$ for all $a\in K$.

\vspace{1ex}\noindent{\em Existence.}  We do not give a proof of
existence in the general case.  Instead we give a proof, which was
suggested by Dr. Geyer at the conference out of which
\cite{cassels:global} arose. It is valid when~$K$ is locally
compact, which is the only case we will use later.

We see at once that the function defined in (\ref{eqn:normdef})
satisfies the condition (i) that $\norm{a}\geq 0$ with equality only
for $a=0$, and (ii) $\norm{ab}=\norm{a}\cdot \norm{b}$ for all $a,b\in
L$.  The difficult part of the proof is to show that there is a
constant $C>0$ such that $$\norm{a}\leq 1 \implies \norm{1+a}\leq C.$$
Note that we do not know (and will not show) that $\normspc$ as
defined by (\ref{eqn:normdef}) is a norm as in
Definition~\ref{defn:norm}, since showing that $\normspc$ is a norm
would entail showing that it satisfies the triangle inequality, which
is not obvious.

Choose a basis $b_1,\ldots, b_N$ for~$L$ over~$K$.  Let $\normspc_0$
be the max norm on $L$, so for $a=\sum_{i=1}^N c_i b_i$ with $c_i\in K$ we have
$$
\norm{a}_0 = \norm{\sum_{i=1}^N c_i b_i}_0 = \max \{\abs{c_i} : i=1,\ldots, N\}.
$$
(Note: in Cassels's original article he let $\normspc_0$ be {\em
  any} norm, but we don't because the rest of the proof does not work,
since we can't use homogeneity as he claims to do.  This is because it need not
be possible to find, for any nonzero $a\in L$ some element $c\in K$ such that
$\norm{ac}_0=1$.  This would fail, e.g., if $\norm{a}_0\neq \abs{c}$
for any $c\in K$.)
The rest of the argument is very similar to our proof from
Lemma~\ref{lem:ext_unique} of uniqueness of norms on vector spaces
over complete fields.

With respect to the $\normspc_0$-topology, $L$ has the product topology
as a product of copies of $K$.  The
function $a\mapsto \norm{a}$ is a composition of continuous functions on $L$
with respect to this topology (e.g., $\Norm_{L/K}$ is the determinant, hence
polynomial),
hence $\normspc$ defines nonzero continuous function on the compact set 
$$
 S = \{a \in L : \norm{a}_0 = 1\}.
$$
By compactness, there are  real numbers $\delta,\Delta\in\R_{>0}$ such that 
$$
0 < \delta \leq \norm{a} \leq \Delta \qquad\text{for all $a\in S$}.
$$
For any nonzero $a\in L$ there exists $c\in K$ such that
$\norm{a}_0 = \abs{c}$; to see this take $c$ to be a $c_i$
in the expression $a=\sum_{i=1}^N c_i b_i$ with $\abs{c_i}\geq \abs{c_j}$
for any~$j$.  Hence $\norm{a/c}_0 = 1$, so $a/c\in S$ and
$$
0 \leq \delta \leq \frac{\norm{a/c}}{\norm{a/c}_0} \leq \Delta.
$$
Then by homogeneity
$$
0 \leq \delta \leq \frac{\norm{a}}{\norm{a}_0} \leq \Delta.
$$
Suppose now that $\norm{a}\leq 1$.  Then $\norm{a}_0\leq \delta^{-1}$, so 
\begin{align*}
 \norm{1+a} &\leq \Delta\cdot \norm{1+a}_0 \\
  &\leq \Delta\cdot \left( \norm{1}_0 + \norm{a}_0\right)\\
  &\leq \Delta\cdot \left( \norm{1}_0 + \delta^{-1}\right)\\
  &=C \quad\text{(say)},
\end{align*}
as required.
\end{proof}

\begin{example}
Consider the extension $\C$ of $\R$ equipped with the archimedean valuation.
The unique extension is the ordinary absolute value on $\C$:
$$\norm{x+iy} = \left(x^2 + y^2\right)^{1/2}.$$
\end{example}

\begin{example}
Consider the extension $\Q_2(\sqrt{2})$ of $\Q_2$ 
equipped with the $2$-adic absolute value.  
Since $x^2-2$ is irreducible over $\Q_2$ we can do
some computations by working in the subfield $\Q(\sqrt{2})$
of $\Q_2(\sqrt{2})$.
\begin{lstlisting}
sage: K.<a> = NumberField(x^2 - 2); K
Number Field in a with defining polynomial x^2 - 2
sage: norm = lambda z: math.sqrt(2^(-z.norm().valuation(2)))
sage: norm(1 + a)
1.0
sage: norm(1 + a + 1)
0.70710678118654757
sage: z = 3 + 2*a
sage: norm(z)
1.0
sage: norm(z + 1)
0.35355339059327379
\end{lstlisting}

%\begin{verbatim}
%> K<a> := NumberField(x^2-2);
%> K;
%Number Field with defining polynomial x^2 - 2 over the Rational Field
%> function norm(x) return Sqrt(2^(-Valuation(Norm(x),2))); end function;
%> norm(1+a);
%1.0000000000000000000000000000
%> norm(1+a+1);
%0.70710678118654752440084436209
%> z := 3+2*a;
%> norm(z);
%1.0000000000000000000000000000
%> norm(z+1);
%0.353553390593273762200422181049
%\end{verbatim}
\end{example}

\begin{remark}
  Geyer's existence proof gives (\ref{eqn:normdef}).  But it is
  perhaps worth noting that in any case (\ref{eqn:normdef}) is a
  consequence of unique existence, as follows.  Suppose $L/K$ is as
  above.  Suppose $M$ is a finite Galois extension of $K$ that
  contains~$L$.  Then by assumption there is a unique extension of
  $\absspc{}$ to $M$, which we shall also denote by $\normspc$.  If
  $\sigma\in\Gal(M/K)$, then
$$\norm{a}_\sigma := \norm{\sigma(a)}
$$
is also an extension of $\absspc{}$ to $M$, so $\normspc_\sigma = \normspc$,
i.e.,
$$ 
  \norm{\sigma(a)} = \norm{a}\qquad\text{for all $a\in M$}.
$$
But now 
$$
\Norm_{L/K}(a) = \sigma_1(a) \cdot \sigma_2(a) \cdots \sigma_N(a)
$$
for $a\in K$, where $\sigma_1,\ldots, \sigma_N\in \Gal(M/K)$ extend the embeddings
of $L$ into $M$.
Hence 
\begin{align*}
 \abs{\Norm_{L/K}(a)} &= \norm{\Norm_{L/K}(a)} \\
     &= \prod_{1\leq n \leq N} \norm{\sigma_n(a)}\\
     &= \norm{a}^N,
\end{align*}
as required.
\end{remark}

\begin{corollary} 
Let $w_1,\ldots, w_N$ be a basis for $L$ over $K$.  Then
there are positive constants $c_1$ and $c_2$ such that 
$$
   c_1 \leq \frac{\norm{\ds\sum_{n=1}^{N} b_n w_n}}{\max \{ \abs{b_n} : n = 1,\ldots, N\}} \leq c_2
$$
for any $b_1,\ldots, b_N\in K$ not all $0$.
\end{corollary}
\begin{proof}
  For $\abs{\sum_{n=1}^N b_n w_n}$ and $\max{\abs{b_n}}$ are two norms
  on $L$ considered as a vector space over $K$.  

I don't believe this
  proof, which I copied from Cassels's article.  My problem with it is
  that the proof of Theorem~\ref{thm:extunique} does not give that
  $C\leq 2$, i.e., that the triangle inequality holds for $\normspc$.  By
changing the basis for $L/K$ one can make any nonzero vector $a\in L$
have $\norm{a}_0=1$, so if we choose $a$ such that $\abs{a}$ is very large,
then the $\Delta$ in the proof will also be very large.  One way to fix
the corollary is to only claim that there are positive
constants $c_1, c_2,c_3, c_4$
such that 
$$
   c_1 \leq \frac{\norm{\ds\sum_{n=1}^{N} b_n w_n}^{c_3}}{\max \{ \abs{b_n}^{c_4} : n = 1,\ldots, N\}} \leq c_2.
$$
Then choose $c_3, c_4$ such that $\normspc^{c_3}$ and $\absspc^{c_4}$ satisfies the triangle inequality, and prove the modified corollary
using the proof suggested by Cassels.
\end{proof}

\begin{corollary}\label{cor:iscomp}\icor{extension of complete field is complete}
A finite extension of a completely valued field $K$ is complete
with respect to the extended valuation.
\end{corollary}
\begin{proof}
By the proceeding corollary it has the topology of a finite-dimensional
vector space over $K$. (The problem with the proof of the previous
corollary is not an issue, because we can replace the extended valuation
by an inequivalent one that satisfies the triangle inequality and
induces the same topology.)
\end{proof}

When $K$ is no longer complete under $\absspc$ the position is more complicated:
\begin{theorem}\label{thm:extensions}\ithm{valuation extensions}
Let $L$ be a separable extension of $K$ of finite degree
  $N=[L:K]$.  Then there are at most $N$ extensions of a valuation
  $\absspc$ on $K$ to $L$, say $\normspc_j$, for $1\leq j \leq J$.
  Let $K_v$ be the completion of $K$ with respect to $\absspc$, and for
  each~$j$ let $L_j$ be the completion of $L$ with respect to
  $\normspc_j$.  Then
\begin{equation}\label{eqn:tenslocal}
  K_v \tensor_K L \isom \bigoplus_{1\leq j\leq J} L_j
\end{equation}
algebraically and topologically, where the right hand side is given
the product topology.
\end{theorem}
\begin{proof}
  We already know (Lemma~\ref{lem:tensor_prod}) that $K_v\tensor_K L$
  is of the shape (\ref{eqn:tenslocal}), where the $L_j$ are finite
  extensions of $K_v$.  Hence there is a unique extension
  $\absspc_j^*$ of $\absspc$ to the $L_j$, and by
  Corollary~\ref{cor:iscomp} the $L_j$ are complete with respect to
  the extended valuation.  Further,  the
  ring homomorphisms
  $$
  \lambda_j : L \to K_v\tensor_K L \to L_j
  $$
  are injections.   Hence we get an extension $\normspc_j$ of $\absspc$ to $L$ by putting 
$$
\norm{b}_j = \abs{\lambda_j(b)}_j^*.
$$
Further, $L\isom \lambda_j(L)$ is dense in $L_j$ with respect to $\normspc_j$ because
$L = K\tensor_K L$ is dense in $K_v\tensor_K L$ (since $K$ is dense
in $K_v$).  Hence $L_j$ is exactly the completion of $L$.

It remains to show that the $\normspc_j$ are distinct and that they
are the only extensions of $\absspc{}$ to~$L$.

Suppose $\normspc$ is any valuation of $L$ that extends $\absspc$.  Then
$\normspc$ extends by continuity to a real-valued function on $K_v\tensor_K L$,
which we also denote by $\normspc$. (We are again using that $L$ is dense
in $K_v\tensor_K L$.)  By continuity we have for all $a,b\in K_v\tensor_K L$,
$$
  \norm{ab} = \norm{a}\cdot \norm{b}
$$
and if $C$ is the constant in axiom (iii) for $L$ and $\normspc$, then
$$
 \norm{a}\leq 1 \implies \norm{1+a}\leq C.
$$
(In Cassels, he inexplicable assume that $C=1$ at this point in the proof.)

We consider the restriction of $\normspc$ to one of the $L_j$.  If $\norm{a}\neq 0$
for some $a\in L_j$, then $\norm{a} = \norm{b}\cdot \norm{ab^{-1}}$ for every
$b\neq 0$ in $L_j$ so $\norm{b} \neq 0$.  Hence either $\normspc$ is identically
$0$ on $L_j$ or it induces a valuation on $L_j$.

Further, $\normspc$ cannot induce a valuation on two of the $L_j$.  For
$$
  (a_1,0,\ldots, 0)\cdot (0,a_2,0,\ldots,0) = (0,0,0,\ldots,0),
$$
so for any $a_1\in L_1$, $a_2\in L_2$,
$$
  \norm{a_1}\cdot \norm{a_2} = 0.
$$
Hence $\normspc{}$ induces a valuation in precisely one of the $L_j$,
and it extends the given valuation $\absspc$ of $K_v$.  Hence $\normspc = \normspc_j$
for precisely one $j$.

It remains only to show that (\ref{eqn:tenslocal}) is a topological homomorphism.
For $$(b_1,\ldots, b_J)\in L_1\oplus \cdots \oplus L_J$$ put
$$
\norm{(b_1,\ldots, b_J)}_0 = \max_{1\leq j \leq J} \norm{b_j}_j.
$$
Then $\normspc_0$ is a norm on the right hand side of (\ref{eqn:tenslocal}),
considered as a vector space over $K_v$ and it induces the product topology.
On the other hand, any two norms are equivalent, since $K_v$ is complete,
so $\normspc_0$ induces the tensor product topology on the left hand side of
(\ref{eqn:tenslocal}).
\end{proof}

\begin{corollary}
Suppose $L=K(a)$, and let $f(x)\in K[x]$ be the minimal polynomial of~$a$. 
Suppose that 
$$
f(x) = \prod_{1\leq j\leq J} g_j(x)
$$
in $K_v[x]$, where the $g_j$ are irreducible.  Then $L_j = K_v(b_j)$, where
$b_j$ is a root of $g_j$.
\end{corollary}

\section{Extensions of Normalized Valuations}
Let $K$ be a complete field with valuation $\absspc$.  
We consider the following three cases:
\begin{itemize}
\item[(1)] $\absspc$ is discrete non-archimedean and the 
residue class field is finite.
\item[(2i)] The completion of $K$ with respect to $\absspc$ is $\R$.
\item[(2ii)] The completion of $K$ with respect to $\absspc$ is $\C$.
\end{itemize}
(Alternatively, these cases can be subsumed by the hypothesis that
the completion of $K$ is locally compact.)

In case (1) we defined the normalized valuation to 
be the one such that if Haar measure of the ring of integers $\O$ is $1$,
then $\mu(a\O) = \abs{a}$ (see Definition~\ref{defn:normalized}).
In case (2i) we say that $\absspc$ is normalized if it is the ordinary
absolute value, and in (2ii) if it is the {\em square} of the ordinary
absolute value:
$$\abs{x+iy} = x^2 + y^2 \qquad \text{(normalized)}.$$
In every case, for every $a\in K$,  the map 
$$
   a: x \mapsto a x
$$
on $K^+$ multiplies any choice of Haar measure by $\abs{a}$, and this characterizes
the normalized valuations among equivalent ones. 

We have already verified the above characterization for
non-archimedean valuations, and it is clear for the ordinary absolute
value on $\R$, so it remains to verify it for $\C$.  The additive
group $\C^+$ is topologically isomorphic to $\R^+ \oplus \R^+$, so a
choice of Haar measure of $\C^+$ is the usual area measure on the
Euclidean plane.  Multiplication by $x+iy\in \C$ is the same as
rotation followed by scaling by a factor of $\sqrt{x^2+y^2}$, so if we
rescale a region by a factor of $x+iy$, the area of the region changes
by a factor of the square of $\sqrt{x^2+y^2}$. This explains why the
normalized valuation on $\C$ is the square of the usual absolute
value.  Note that the normalized valuation on $\C$ does not satisfy
the triangle inequality:
$$
\abs{1 + (1+i)} = \abs{2+i} = 2^2 + 1^2 = 5 \not\leq 
 3= 1^2 + (1^2 + 1^2) =  \abs{1} + \abs{1+i}.
$$
The constant $C$ in axiom (3) of a valuation for the ordinary
absolute value on $\C$ is $2$, so the constant for the normalized
valuation $\absspc$ is $C\leq 4$:
$$
 \abs{x+iy} \leq 1 \implies \abs{x+iy+1} \leq 4.
$$
Note that $x^2 +y^2 \leq 1$ implies $$(x+1)^2 + y^2 
 = x^2 + 2x + 1 + y^2 \leq 1 + 2x + 1 \leq 4$$ since
$x\leq 1$.

\begin{lemma}\ilem{extension of normalized valuation}
  Suppose~$K$ is a field that is complete with respect to a normalized
  valuation~$\absspc$ and let~$L$ be a finite extension of~$K$ of
  degree $N=[L:K]$.  Then the normalized valuation $\normspc$ on~$L$
  which is equivalent to the unique extension of $\absspc$ to~$L$ is
  given by the formula
\begin{equation}\label{eqn:normdef2}
 \norm{a} = \abs{\Norm_{L/K}(a)}\qquad \text{all } a\in L.
\end{equation}
\end{lemma}
\begin{proof}
Let $\normspc$ be the normalized valuation on $L$ that extends $\absspc$.
Our goal is to identify $\normspc$, and in particular to show
that it is given by (\ref{eqn:normdef2}).

By the preceding section there is a positive real number~$c$ 
such that for all $a\in L$ we have
$$\norm{a} = \abs{\Norm_{L/K}(a)}^c.$$
Thus all we have to do is prove that $c=1$.
In case 2 the only nontrivial situation is $L=\C$ and $K=\R$,
in which case $\abs{\Norm_{\C/\R}(x+iy)} = \abs{x^2+y^2}$,
which is the normalized valuation on $\C$ defined above.

One can argue in a unified way in all cases as follows.
Let $w_1,\ldots, w_N$ be a basis for $L/K$. Then the map
$$
  \vphi:L^+ \to \bigoplus_{n=1}^N K^+, \qquad 
\sum a_n w_n \mapsto (a_1,\ldots, a_N)
$$
is an isomorphism between the additive group $L^+$
and the direct sum $\oplus_{n=1}^N K^+$, and
this is a homeomorphism if the right hand side
is given the product topology.  In particular, the
Haar measures on $L^+$ and on $\oplus_{n=1}^N K^+$
are the same up to a multiplicative constant in $\Q^*$.  

Let $b\in K$.  Then the left-multiplication-by-$b$ map
$$
   b : \sum a_n w_n \mapsto \sum b a_n w_n
$$
on $L^+$ is the same as the map
$$
  (a_1,\ldots, a_N) \mapsto (ba_1,\ldots, ba_N)
$$
on $\oplus_{n=1}^N K^+$, so it multiplies the Haar
measure by $\abs{b}^N$, since $\absspc$ on~$K$
is assumed normalized (the measure of each factor
is multiplied by $\abs{b}$, so the measure on
the product is multiplied by $\abs{b}^N$).
Since $\normspc$ is assumed normalized, so
multiplication by~$b$ rescales by $\norm{b}$, we
have 
$$
  \norm{b} = \abs{b}^N.
$$
But $b\in K$, so $\Norm_{L/K}(b) = b^N$.
Since $\absspc$ is nontrivial and for $a\in K$ we 
have $$\norm{a} = \abs{a}^N = \abs{a^N} = \abs{\Norm_{L/K}(a)},$$
so we must have $c=1$ in (\ref{eqn:normdef2}), as claimed.
\end{proof}

In the case when~$K$ need not be complete with respect
to the valuation~$\absspc$ on~$K$, we have the following
theorem.
\begin{theorem}\label{thm:normprod}\ithm{product of extensions}
Suppose $\absspc$ is a (nontrivial as always) normalized valuation
of a field~$K$ and let~$L$ be a finite extension of~$K$.
Then for any $a\in L$,
$$  
   \prod_{1\leq j \leq J} \norm{a}_j = \abs{\Norm_{L/K}(a)}
$$
where the $\normspc_j$ are the normalized valuations equivalent
to the extensions of~$\absspc$ to~$K$.
\end{theorem}
\begin{proof}
Let $K_v$ denote the completion of $K$ with respect to 
$\absspc$.  Write
$$
  K_v\tensor_K L = \bigoplus_{1\leq j \leq J} L_j.
$$
Then Theorem~\ref{thm:normprod} asserts that
\begin{equation}\label{eqn:normprod}
 \Norm_{L/K}(a) = \prod_{1\leq j\leq J} \Norm_{L_j/K_v}(a).
\end{equation}
By Theorem~\ref{thm:extensions}, the $\normspc_j$ are exactly the
normalizations of the extensions of $\absspc$ to the $L_j$ (i.e., the
$L_j$ are in bijection with the extensions of valuations, so there are
no other valuations missed).  By Lemma~\ref{eqn:normdef}, the
normalized valuation $\normspc_j$ on $L_j$ is $\abs{a} =
\abs{\Norm_{L_J/K_v}(a)}$.  The theorem now follows by taking absolute
values of both sides of (\ref{eqn:normprod}).
\end{proof}

What next?!  We'll building up to giving a new proof of finiteness of
the class group that uses that the class group naturally has the
discrete topology and is the continuous image of a compact group.


%%% Local Variables: 
%%% mode: latex
%%% TeX-master: "ant"
%%% End: 
